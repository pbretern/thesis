\documentclass[10pt,a4paper,twocolumn,conference]{IEEEtran}

\usepackage[german]{babel}    % Deutsche Sprache in automatisch generiertem
\usepackage{latexsym}         % Fuer recht seltene Zeichen
\usepackage[utf8]{inputenc}   % =E4 =F6 =FC =DF; danach  geht auch das ß richtig
\usepackage{caption}          % Figure-Captions formatieren
\usepackage{sectsty}          % Section headings formatieren
\usepackage[fixlanguage]{babelbib}
\usepackage[a4paper,lmargin={2.5cm},rmargin={2.5cm},tmargin={3cm},bmargin={2.5cm}]{geometry}
\usepackage{blindtext}
\usepackage{bookmark}
\usepackage{verbatim}
\usepackage[numbers]{natbib}

\pdfinfo{
	/Title		(Titel)
	/Subject	(Expose zur Bachelor-Studienjahres-/Master-/Diplomarbeit)
	/Author		(Max Mustermann)
}
\selectbiblanguage{german}
\allsectionsfont{\sffamily}
\captionsetup{margin=1cm,font=small,labelfont=bf}
%\setlength\parindent{24pt}

\newcommand\notice[1]{}
\newcommand\seppar{ \vspace{2ex} \noindent }
\usepackage{hyperref}

%\renewcommand*{\thesection}{\Roman{section}.}
\begin{document}
\nocite{*}
\title{{\bf Visualisierung von bibliografischen Metadaten als zusätzlichen Einstieg in die
        Recherche nach Medien in einem Bibliothekskatalog} \\ 
    \begin{large}
        Exposé zur Masterarbeit                                                                             
    \end{large}}
\author{
	Peter Breternitz \\
	TH Wildau, Wildau Institute of Technology\\ Bibliotheksinformatik \\
	peter.breternitz@th-wildau.de
}


\date{\today}

\maketitle

\section{Einführung}
Gedächtnisinstitutionen wie Bibliotheken verfügen über einen Schatz 
hochqualitativer bibliografischer Metadaten. Diese werden in klassischen 
Online-Katalogen oder in Discovery-Systemen zur Anzeige gebracht. 
Neben einer Facettierung, mit der die gefundenen Ergebnisse einer Suchanfrage fein justiert werden 
können, scheinen Discovery-Systeme hinsichtlich der Darstellung von Suchergebnissen 
nicht mehr zu bieten als die klassischen Online-Kataloge.\footnote{Visualisierung bezeichnet Disziplinen
wie Informationsvisualisierung, Datenvisualisierung und visuelle Analyse. Dabei wird sich konzentriert 
auf Ansätze die Visualisierungen mittels Visualisierungstechniken algorithmisch aus den Metadaten
der Dokumente oder aus den Dokumenten selber erzeugen.} 
Es wäre hierbei zu fragen, ob Bibliotheken die Potentiale ihrer Metadaten für die Darstellung von Suchergebnissen 
vollständig ausschöpfen. Gibt es Alternativen in der Human-Computer-Kommunikation, die dass explorative Suchen 
der Nutzer:Innen fördern und Ergebnisse zu neuen Wissensräume zusammenschließen?

Visualisierungen von semantischen Informationsräumen können die Aussagekraft von Bibliotheksmetadaten erhöhen. Es werden
neue wissenschaftliche Zusammenhänge und Trends durch die Visualisierung
sichtbar \cite{RN1}. Durch das Aufzeigen von visuellen Suchpfaden sowie dem
Anzeigen davon, wie die einzelnen Informationsressourcen über Gemeinsamkeiten ihrer
Metadaten verbunden sind, können sich neue Informationsräume ergeben, die der
Informationsgewinnung dienen. Visuelle Übergänge zwischen den
Informationsressourcen können helfen, den wissenschaftlichen Kontext leichter
zu überblicken, der bei der klassischen Darstellung in Katalogen verloren gehen kann. 
Diese These und die oben aufgeführten Fragestellungen werden in der zu
entstehenden Arbeit vor dem Hintergrund des Semantic Webs und
anderer Technologien in einem ersten Schritt theoretisch diskutiert.
In einem nachfolgend praktischen Teil wird eine Applikation als proof-of-concept anhand 
bibliografischer Metadaten einer Gedächntnisinstitution entwickelt. 
Die Applikation soll die Suche nach Informationsressourcen mit geeigneten
Visualisierungstechniken wie zum Beispiel einem Network Diagram u.ä. visualisieren. 
Letztlich geht es um ein zusätzliches Angebot an die Wissenschaftler:Innen. 
Darüberhinaus könnte die Applikation einerseits den Bestand ansprechend visuell
präsentieren und ihn dadurch sichtbarer machen (What is trending?
Neuerwerbungslghisten visuell präsentieren). Andererseits könnte sie
auch als Werkzeug eingesetzt werden, um zum Beispiel den
Bestandsaufbau zu unterstützen und ihn kontrolliert zu entirwickeln. 



\section{Problemstellung}
Die Recherche nach Ressourcen in einem Bibliothekskatalog oder Discovery-System\footnote{Discovery-Sytemen operieren als 
One-Stop-Shop-Systeme und basieren auf der Suchmaschinentechnologie. Sie zeichnen sich unter anderem durch einen großen 
Suchraum, eine intuitive Bedienbarkeit und einem Ranking der Treffer nach
Relevanz aus.} kann über mehrere Sucheinstiege wie Titelstichwörter, 
Sachschlagwörter, AutorInnen oder über den gesamten Index passieren. 
Discovery-Systeme bieten Facettierungen an, um die Ergebnissmengen im nachhinein
einzuschränken. Die zurückgelieferte Ergebnissmenge einer
Suchanfrage entspricht meist einer textuellen Liste, die nach voreingestellten Kriterien
sortiert ist. Abgesehen von graphischen Darstellungen von Erscheinungsjahre wie
zum Beispiel im SLUB-Katalog\footnote{\url{https://www.slub-dresden.de/recherche/} Zugriff:
\today} sind Optionen der Visualisierungen nicht vorhanden.
%Meist stehen sie scheinbar unverbunden nebeneinander.
%Informationsräume\\Searching-Browsing Concept\\Interlinked-Ressources\\Visuelle
%Pfade\\Verknüpfen von Informationen\\Leichter fällt die Suchergebnisse zu
%überblicken\\Schlagwortesuche über das Highlighten der Worte hinaus geht
%Discovery-System. Darstellung der Ergebnisse - Grenzen der Darstellung.
%Sacherschließung, Stöbern nach ähnlichen Medien, das Gruppieren nach
%inhaltlichen Gesichtspunkten.
\section{Literaturdiskussion}
Visualisierungen und Visualisierungstechniken zur Analyse von Daten und zum Erkennen von
Informationszusammenhängen sind Bereiche, die in den letzten drei Jahrzehnten
gewachsen sind und eine Vielzahl an Anwendungen hervorgebracht haben \cite{RN14}. Im
bibliothekarischen bzw. informationswissenschaftlichen Bereich sind hier
\textit{VisualBib}, \textit{PivotPath} und \textit{SeeCollections} zu nennen.
\textit{VisualBib} wertet Buchkapitel, Papers, Bücher und Conference sowie Workshop Article 
von verschiedenen Datenbanken aus, stellt diese graphisch in
\textit{narrativ views} dar und hilft den WissenschaftlerInnen dabei Bibliographien zu erstellen \cite{RN17}. \textit{PivotPath} ist eine interaktive
Visualisierung zur Erkundung von Informationsressourcen \cite{RN12}. 
\textit{SeeCollections} ist eine Web-Application zur visuellen
Darstellung von Bibliotheksteilbetänden nach Klassifikation, Thema und nach
Erscheinungsjahr \cite{RN9}.
%Während die ersten beiden Anwendungen Metadaten
%Suchfragen herstellt. Wie stelle ich mir die Visualisierung von Ergebnismengen
% von Suchanfragen vor? Welche Visualisierungstechniken gibt es da, um das
% umsetzen zu können?:update




\begin{comment}
\section{Theoretischer Rahmen}
Hier sollen theoretische Annahmen und Konzepte kurz eingeführt und erklärt
werden. Aus bibliothekarischer Perspektive werden in einem Teil wichtige
Konzepte wie das Semantic Web und dessen Technologien erklärt.\cite{RN8}  Stichpunkte und
Fragen für diesen \textit{Theorie-I-Teil} sind:\\
Semantic Web, FRBR, LRM, RDF, LOD, Warum sind Metadaten von
Bibliotheksbeständen besonders für Visualisierungen geeignet? Catalogue
Enrichment.Für den zweiten Theorieteil werden Schlaglichter auf Data
Science, Digital Humanities geworfen. Dabei spielen auch die verwendeten
Technologien eine Rolle. Stichpunkte sind für diesen \textit{Theorie-II-Teil:}\\
Visualisierungen, Grenzen und Möglichkeiten, Interaktive Grafiken,
Data Science Life Cycle
Kapitel in den \textit{Theorie-III-Teil} zusammen und diskutiert anhand 
von Literatur aus der bibliothekarischen Praxis bereits vorhandene Beispiele. 
Dabei wird hierbei auch die Frage der technischen Umsetzung eine Rolle spielen,
aber auch auf insbesondere eventuelle erstellen hingewiesen \cite{RN9, RN3}.
RVK-Visual \url{https://github.com/bvb-kobv-allianz/RVK-VISUAL}
\end{comment}
\section{Eigener Ansatz und berücksichtigte Methoden}
Entwicklung Proof-of-Conceptes
\section{Abriß}
Ein grober Abriß der Masterarbeit bezüglich des Inhaltes und des Umfangs ist im
Folgendem angegeben:\\
\textit{Einführung (ca. 5 S.)}:\\Tool mit dem man Suchergebnisse visualisieren kann\\
Tool für die kontrollierte Bestandsentwicklung\\Übersicht über die Kapitel - 1-
4 Sätze\\
\textit{Theoretische Teil I (ca.10 S.)}:\\
Semantic Web Technoliogien\\
FRBR, LRM, RDF, Linked Open Data, Catalogue Enrichment\\
Warum eignen sich Metadaten aus Bibliotheken und anderen Kultureinrichtungen besonders\\
\textit{Theoretische Teil II (ca. 15 S.)}:\\
Grenzen und Möglichkeiten von Datenvisualisierungen vor dem
Hintergrund Data Science und Digital Humanities\\
Interaktive Visualisierungen\\ 
\textit{Theoretische Teil III (ca. 15 S.)}:\\
Literaturdiskussion Bibliotheken - Metadaten - Visualisierungen\\
Beispiele - Technologien\\
\textit{Praktischer Teil (ca. 15 S.)}:\\
Entwicklung eines proof-of-Conceptes anhand von Metadaten einer
Bibliothek oder einer Gedächtnisinstitution\\
Aufzeigen des Workflows\\
Theoretisches Modell der Data Science anwenden auf das Projekt\\
praktische Methoden - Wie und mit was programmieren\\
Auswertung welcher Metadatenfelder\\
Catalogue Enrichment\\
\textit{Schluß (ca. 5 S.)}:\\
Resultat\\
Zusammenfassung\\
zukünftige Forschungfragen
            
\bibliographystyle{plain}
\bibliography{literature/lit} 


\end{document}
