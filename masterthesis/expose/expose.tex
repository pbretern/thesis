\documentclass[10pt,a4paper,twocolumn,conference]{IEEEtran}

\usepackage[german]{babel}    % Deutsche Sprache in automatisch generiertem
\usepackage{latexsym}         % Fuer recht seltene Zeichen
\usepackage[utf8]{inputenc}   % =E4 =F6 =FC =DF; danach  geht auch das ß richtig
\usepackage{caption}          % Figure-Captions formatieren
\usepackage{sectsty}          % Section headings formatieren
\usepackage[fixlanguage]{babelbib}
\usepackage[a4paper,lmargin={2.5cm},rmargin={2.5cm},tmargin={3cm},bmargin={2.5cm}]{geometry}
\usepackage{blindtext}
\usepackage{bookmark}
\usepackage{verbatim}
\usepackage[numbers]{natbib}

\pdfinfo{
	/Title		(Titel)
	/Subject	(Expose zur Bachelor-Studienjahres-/Master-/Diplomarbeit)
	/Author		(Max Mustermann)
}
\selectbiblanguage{german}
\allsectionsfont{\sffamily}
\captionsetup{margin=1cm,font=small,labelfont=bf}
%\setlength\parindent{24pt}

\newcommand\notice[1]{}
\newcommand\seppar{ \vspace{2ex} \noindent }
\usepackage{hyperref}

%\renewcommand*{\thesection}{\Roman{section}.}
\begin{document}
\nocite{*}
\title{{\bf Datengetriebenes Unterstützungssystem für zukünftige
        Erwerbungspolitik, Etatplanung und Mittelallokation einer hypriden
        Spezislbibliotek} \\ 
    \begin{large}
        Exposé zur Masterarbeit                                                                             
    \end{large}}
\author{
	Peter Breternitz \\
	TH Wildau, Wildau Institute of Technology\\ Bibliotheksinformatik \\
	peter.breternitz@th-wildau.de
}


\date{\today}

\maketitle

\section{Einführung}
Ziel der zu entstehenden Arbeit ist die Entwicklung einer Applikation,
mit der zentral Nutzungsdaten bibliothekarischer Servicedienstleistungen einer
hypriden Spezialbibliothek gesammelt, statistisch
analysiert und mit geeigneten Datenvisualisierungstechniken dargestellt werden können. 
% Ziel dieses Tool

Mit diesen statistischen Datenanalysen sollen zukünftige
Entscheidungen im Bibliotheksmanagement wie Erwerbungspolitik, Budgetplanung und 
Mittelallokation hinsichtlich der weiteren Entwicklung der 
Servicedienstleistungen evidenz basiert und datengetrieben unterstützt werden. 

Darüberhinaus soll die Applikation eine Funktion beinhalten, ausgewählte
Resultate automatisiert als \textit{factsheet} zu exportieren, um diese 
als Rechenschaftsbericht gegenüber stakeholdern der Bibliothek präsentieren zu
können.
%predictive analysis\\
%evidence based stock management\\

%Erschaffen eines Reports

\section{Problemstellung}
\blindtext
\section{Literaturdiskussion}
\blindtext
Untersuchungen zu library assessement finden sich in der Literatur zahlreiche.
Es werden Bestandsgruppen, Bibliotheksservices evaluiert. Auch gibt es dass
schon für Bestandsuaftbau. Diese einzelnen Stränge werden untersucht, aber ein
zusammenbringen ist nirgendwo zu finden
Meist sind es große Bibliotheken. Ein auf Spezialbibliotheken ausgerichtetes
Tool fehlt -> Beziehen der Daten aus heterogenen Quellen
Welche Daten sind sinnvoll zu erheben\\
Analyse der zu betrachtenden Quellen an welche Daten komme ich ran?\\
Wie sind diese aufgebaut?\\
Welche statistischen Methoden sollen angewendet werden?\\
Wie sieht eine adäquate Visualisierung aus?
Ziel: Beobachtung der zu erhebenden Daten um daraus Rückschlüsse auf weitere
bibliothekarische Service-Angebote zu ziehen. Budgetplanung, Übersicht über
Bestandswachstum, Diversifikation in den einzelnen Bestandsgruppen
Einbettung der Statistiken der MPDL\\
internes Tool zur Dienstleistungssteuerung
Report für Stackholders
um eigene Arbeit zu evaluieren.
Tätigkeitsbericht zu den stakeholdern


Ziel ist es ein Tool zu erschaffen, dass 
\section{Theoretischer Rahmen}
\blindtext
\section{Eigener Ansatz und berücksichtigte Methoden}
Entwicklung Proof-of-Conceptes
\section{Abriß}
Ein grober Abriß der Masterarbeit bezüglich des Inhaltes und des Umfangs ist im
Folgendem angegeben:\\
\textit{Einführung (ca. 5 S.)}:\\Tool mit dem man Suchergebnisse visualisieren kann\\
Tool für die kontrollierte Bestandsentwicklung\\Übersicht über die Kapitel - 1-
4 Sätze\\
\textit{Schluß (ca. 5 S.)}:\\
Resultat\\
Zusammenfassung\\
zukünftige Forschungfragen
            
\bibliographystyle{plain}
\bibliography{literature/lit} 


\end{document}
