\documentclass[10pt,a4paper,twocolumn,conference]{IEEEtran}

\usepackage[german]{babel}    % Deutsche Sprache in automatisch generiertem
\usepackage{latexsym}         % Fuer recht seltene Zeichen
\usepackage[utf8]{inputenc}   % =E4 =F6 =FC =DF; danach  geht auch das ß richtig
\usepackage{caption}          % Figure-Captions formatieren
\usepackage{sectsty}          % Section headings formatieren
\usepackage[fixlanguage]{babelbib}
\usepackage[a4paper,lmargin={2.5cm},rmargin={2.5cm},tmargin={3cm},bmargin={2.5cm}]{geometry}
\usepackage{blindtext}
\usepackage{bookmark}
\usepackage{verbatim}
\usepackage[numbers]{natbib}

\pdfinfo{
	/Title		(Titel)
	/Subject	(Expose zur Bachelor-Studienjahres-/Master-/Diplomarbeit)
	/Author		(Max Mustermann)
}
\selectbiblanguage{german}
\allsectionsfont{\sffamily}
\captionsetup{margin=1cm,font=small,labelfont=bf}
%\setlength\parindent{24pt}

\newcommand\notice[1]{}
\newcommand\seppar{ \vspace{2ex} \noindent }
\usepackage{hyperref}

%\renewcommand*{\thesection}{\Roman{section}.}
\begin{document}
\nocite{*}
\title{{\bf Datengetriebenes Unterstützungssystem für zukünftige
        Erwerbungspolitik, Etatplanung und Mittelallokation einer hypriden
        Spezislbibliotek} \\ 
    \begin{large}
        Exposé zur Masterarbeit                                                                             
    \end{large}}
\author{
	Peter Breternitz \\
	TH Wildau, Wildau Institute of Technology\\ Bibliotheksinformatik \\
	peter.breternitz@th-wildau.de
}


\date{\today}

\maketitle

\section{Einführung}
Ausgehend von ökonomischen, informationstechnologischen und marktpolitischen Einschnitten in den
letzten zwei Jahrzehnten\footnote{Zu nennen wären hier die Explosion der Zeitschriftenpreise im 
STM-Bereich,das Aufkommen von Epublishing und die Konzentration auf wenige Verlage}, 
sind Bibliotheken dazu veranlasst, ihr Budget hinsichtlich der Informationsbedarfe
ihrer Nutzer:Innen behutsamer zu planen und sich in zunehmenden Maß gegenüber ihren Unterhaltsträgern zu rechtfertigen.

Die Relevanz von bibliothekarischen Kennzahlen wie zum Beispiel Nutzungszahlen 
des Bestandes oder der Fernleihe ist in diesem Zusammenhang größer geworden.
Deswegen ist es wichtig Nutzungsdaten zu sammeln, zu erheben und statistisch
auszuwerten, um auf Basis der daraus erzielten Erkenntnisse handeln zu können.

Ziel der zu entstehenden Arbeit ist die Entwicklung einer Applikation als proof-of-concept,
mit der systematisch Nutzungsdaten bibliothekarischer Servicedienstleistungen einer
hypriden Spezialbibliothek aggregiert, statistisch
analysiert und mit geeigneten Datenvisualisierungstechniken dargestellt werden können. 
% Ziel dieses Tool

Mit diesen statistischen Datenanalysen sollen zukünftige
Entscheidungen im Bibliotheksmanagement wie Erwerbungspolitik, Budgetplanung und 
Mittelallokation hinsichtlich der weiteren Entwicklung der 
Servicedienstleistungen evidenzbasiert und datengetrieben unterstützt werden. 

Um zukünftigen Anforderungen gewachsen zu sein, soll die Applikation
modulbasiert programmiert werden und dadurch leicht erweiterbar sein.

Darüberhinaus soll sie eine Funktion beinhalten, ausgewählte
Resultate automatisiert als \textit{factsheet} zu exportieren, um diese 
als Rechenschaftsbericht gegenüber stakeholdern der Bibliothek präsentieren und
verhandeln zu können.

Die Notwendigkeit für diese Applikation ist durch das
Nichtvorhandensein eines zentralen Nachweisortes für bibliothekarische
Statistiken in der Bibliothek gegeben. Da sie zudem verschiedene Recherche-Systeme den
Wissenschaftler:innen anbietet, wäre eine Engführung bezüglich der statistischen
Datenerhebung auf eine Plattform begrüßenswert.
Des Weiteren ist die Notwendigkeit bibliothekarische Geschäftsprozesse zu evaluieren und die
Servicedienstleistungen bezüglich der Ziele der Institution noch weiter zu
optimieren von großer Relevanz.

Die Applikation soll zudem helfen systematisches Controlling einzuführen und das
Bibliotheksmanagement weiter zu professionalisieren.
%predictive analysis\\
%evidence based stock management\\

%Erschaffen eines Reports

\section{Literaturdiskussion}
%Bibliotheksmanagement - Evaluation von Beständen -> Verweis auf BIX
%Herausstellung das grafische Unterstützung gut ankommt bei Vorgesetzten und
%einem selber hilft Zsh. gut zu verstehen
Tools zur Unterstützung der Datenerhebung von Nutzungsdaten in Bibliotheken
gibt es einige. Von kommerziellen Anbietern sind solhe zu nennen wie Tableau,
CollectionHq insbesondere für öffentliche Bibliotheken, AlmaAnalytics von
ExLibris und BibControl von OCLC.
oder schnödes Excel
Daneben gibt es 
Versuche einzelne Geschäftsprozesse bzw. Fragestellungen mit eigens
programmierten Tools zu beantworten. So werden Fragen nach Weeding versucht zu
beantworten, oder nach dem Erfolg von Lerngruppen in der Bibliothek...
Shiny, R -> sollen leicht erweiterbar sein
Versuche einzelner Fragestellungen werden verknüoft mit grafischen
Darstellungen -> Weeding, ... Erfolg Lerngruppen in Bibliotheken
Fachref aber nur für einen Anbieter geschriebenn
meistens für große Bibliotheken und bisher für keine wissenschaftliche
Spezialbibliothek -> personelle Ressourcen knapp -> Erleichterung bei der
Erfassung von Daten da automatisiert

%Untersuchungen zu library assessement finden sich in der Literatur zahlreiche.
%Es werden Bestandsgruppen, Bibliotheksservices evaluiert. Auch gibt es dass
%schon für Bestandsuaftbau. Diese einzelnen Stränge werden untersucht, aber ein
%zusammenbringen ist nirgendwo zu finden
%Meist sind es große Bibliotheken. Ein auf Spezialbibliotheken ausgerichtetes
%Tool fehlt -> Beziehen der Daten aus heterogenen Quellen
%Welche Daten sind sinnvoll zu erheben\\
%Analyse der zu betrachtenden Quellen an welche Daten komme ich ran?\\
%Wie sind diese aufgebaut?\\
%Welche statistischen Methoden sollen angewendet werden?\\
%Wie sieht eine adäquate Visualisierung aus?
%Ziel: Beobachtung der zu erhebenden Daten um daraus Rückschlüsse auf weitere
%bibliothekarische Service-Angebote zu ziehen. Budgetplanung, Übersicht über
%Bestandswachstum, Diversifikation in den einzelnen Bestandsgruppen
%Einbettung der Statistiken der MPDL\\
%internes Tool zur Dienstleistungssteuerung
%Report für Stackholders
%um eigene Arbeit zu evaluieren.
%Tätigkeitsbericht zu den stakeholdern



\section{Problemstellung}
Die Bibliothek des \textit{Max-Planck-Institutes für empirische Ästhetik}
ist Teil des \textit{HeBis-Verbundes}. Die Geschäftsprozesse der Erwerbung und
der Katalogisierung passieren im CBS und im LBS4 der Firma \textit{OCLC}. Das  
lokale Bibliothekssystem wird gehostet und betreut von dem Lokalsystem-Team Frankfurt.
Über die Bereitstellung besonderer Funktionalitäten seitens des
Lokalsystem-Teams, erhält die Bibliothek u.a. Ausleih- und Lieferantenstatistiken in
Form von automatisch generierten Textdateien von diesem. 


Neben der Verankerung in der deutschen Bibliotheksverbundlandschaft
wird die Bibliothek in ihren Aufgaben von der
\textit{Max-Planck-Digital-Library} (mpdl)
unterstützt. Deren Portfolio umfasst vorrangig die zentrale Lizenzierung
von relevanten elektronischen Informationsressourcen, die Bereitstellung
von Softwarelösungen, das Betreiben eines Publikationsreposituriums und 
das Vorantreiben der OpenAccess-Initiative. Zudem stellt sie noch
COUNTER-Statistiken, die sie von den Verlagen bekommt zur Verfügung.

Außer den bereitgestellten Daten, erhebt die Bibliothek u.a. ebenfalls Daten über
die Frequentierung des Lesesaals, die Nutzung des nehmenden Fernleihservices, des
Dokumentenlieferdienstes \textit{subito} und des Bestandswachstums vor Ort.
Nach den Verantwortlichkeiten aufgeteilt, passiert dieses an verschiedenen virtuellen Orten. 
Eine systematische Auswertung der erhobenen Daten findet nur unzureichend statt. 
Daher regt sich der Wunsch seitens der Mitarbeiter:innen nach einem gemeinsamen Tool, 
womit übersichtlich und klar alle notwendigen nutzungs- und sammlungsbezogenen Statistiken für eine 
Spezialbibliothek erfasst und dargestellt werden können.\footnote{Zwar führt HeBis eine Bestandstatistik, diese ist aber insbesondere für die
Evaluation und Optimierung von Geschäftsprozessen einer Spezialbibliothek
insuffizient \url{https://www.hebis.de/de/1ueber_uns/statistik/cbs_statistik.php}}
Ferner wird sowohl von \textit{HeBis} bzw. von der
LokalBibliotheksSystem-Betreuung als auch von der \textit{mpdl} keine Applikation
in dieser Richtung angeboten.
Ebenso ist es ungewiss, wann die Ablösung des schon betagten CBS/LBS hinzu
einem neuen nextGenerationLibrary-System im Hebis-Verbund stattfindet und
welche zusätzlichen Module neben den klassischen wie Erwerbung, Katalogisirung
umfassen wird, erfolgt.



%Identifizieren von ... Punkten, die für die Datenerhebung interessant sind
%1... -> Warum ? Ziel?

%Rücken der Bestände -> Überprüfen wie Bestandsgruppen in letzten Jahren
%gewachsen sind um daraus abzuschätzen wieviel Platz sie zukünftig brauchen
%Das könnte man da stellen und hätte es auf einen Blick, -> herkömmlich: Abzug
%aus System -> mit OpenRefine Teile der Signatur, die für die Aufstellung
%relevant sind herausfiltern im Zsh. mit Zugangsnummer -> wird das ersichtlich

%-> Darstellung wie wächst jede Bestandsgruppe monatlich (NE), quartalsweise,
%halbjährlich, im jahr -> Balanxing out the stock


%Ziel ist es ein Tool zu erschaffen, dass 
\section{Eigener Ansatz und berücksichtigte Methoden}
Entwicklung Proof-of-Conceptes - Identifizierung relevanter Daten im Rahmen der
Geschäftsgänge und Service-Dienstleistungen einer wissenschaftlichen Spezialbibliothek\\
Analyse der vorliegenden Daten - in welchem Format liegen sie vor - welches
Format is gut für die Weiterverarbeitung\\

\section{Abriß}
Ein grober Abriß der Masterarbeit bezüglich des Inhaltes und des Umfangs ist im
Folgendem angegeben:\\
\textit{Einführung (ca. 5 S.)}:\\Tool mit dem man Suchergebnisse visualisieren kann\\
Tool für die kontrollierte Bestandsentwicklung\\Übersicht über die Kapitel - 1-
4 Sätze\\
\textit{Schluß (ca. 5 S.)}:\\
Resultat\\
Zusammenfassung\\
zukünftige Forschungfragen
            
\bibliographystyle{plain}
\bibliography{literature/lit} 


\end{document}
