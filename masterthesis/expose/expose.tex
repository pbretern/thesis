% Diese Vorlage wurde mit freundlicher Unterstützung von Felix Neumann angefertigt.

\documentclass[10pt,a4paper,twocolumn]{article}

\usepackage[german]{babel}    % Deutsche Sprache in automatisch generiertem
\usepackage{latexsym}         % Fuer recht seltene Zeichen
\usepackage[utf8]{inputenc}   % =E4 =F6 =FC =DF; danach  geht auch das ß richtig
\usepackage{caption}          % Figure-Captions formatieren
\usepackage{sectsty}          % Section headings formatieren
\usepackage[fixlanguage]{babelbib}
\usepackage[a4paper,lmargin={2.5cm},rmargin={2.5cm},tmargin={3cm},bmargin={2.5cm}]{geometry}
\usepackage{blindtext}
\pdfinfo{
	/Title		(Titel)
	/Subject	(Expose zur Bachelor-Studienjahres-/Master-/Diplomarbeit)
	/Author		(Max Mustermann)
}

\selectbiblanguage{german}
\allsectionsfont{\sffamily}
\captionsetup{margin=1cm,font=small,labelfont=bf}

\newcommand\notice[1]{}
\newcommand\seppar{ \vspace{2ex} \noindent }

\renewcommand*{\thesection}{\Roman{section}.}
\begin{document}

\title{{\bf Visualisierung von Metadaten als alternativer Sucheinstieg in den
        Katalog } \\ \begin{large}Exposé zur Masterarbeit                                                                             \end{large}}
\author{
	Peter Breternitz \\
	TH Wildau, Wildau Institute of Technology\\ Bibliotheksinformatik \\
	peter.breternitz@th-wildau.de
}
\date{\today}

\maketitle

\section{Einführung} 
Bibliotheken verfügen über einen Schatz hochqualitativer Metadaten. Diese
werden in klassischen Online-Katalogen oder in neuartigeren Discovery-Systemen
zur Anzeige gebracht. 
wäre die Frage, ob Bibliotheken diesen Schatz und dem hierarchisierten
Sucheinstieg (nochmal überprüfen) nicht unterschätzen und eventuelles
Potential für Nutzer und NutzerInnen verschenken. Neben einer 
einer Facettierung, mit der die gefundenen Ergebnisse einer Suchanfrage fein
justiert werden können, scheinen Discovery-Systeme hinsichtlich der Visualisierung von Suchergebnissen
nicht mehr zu bieten als die klassischen Kataloge. Zweierlei Ziel
verfolgt diese Arbeit: einerseits die Visualisierung der Metadaten des Bestandes
der Bibliothek des Max-Planck-Institutes für empirische Ästhetik als
explorative Alternative zum herkömmlichen Sucheinstieg und dadurch
andererseits die Gewinnung eines Management-Tools zur kontrollierten Steuerung
des Bestandwachstums. 

\section{Problemstellung} 
\blindtext
Zitate: \cite{Harrison75a, Harrison78a}.

\section{Literaturübersicht}

\blindtext
\section{Eigener Ansatz und berücksichtigte Methoden} 
\blindtext


\bibliographystyle{alpha}
\bibliography{literature/lit} 


\end{document}
