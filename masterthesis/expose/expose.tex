% Diese Vorlage wurde mit freundlicher Unterstützung von Felix Neumann angefertigt.

\documentclass[10pt,a4paper,twocolumn]{article}

\usepackage[german]{babel}    % Deutsche Sprache in automatisch generiertem
\usepackage{latexsym}         % Fuer recht seltene Zeichen
\usepackage[utf8]{inputenc}   % =E4 =F6 =FC =DF; danach  geht auch das ß richtig
\usepackage{caption}          % Figure-Captions formatieren
\usepackage{sectsty}          % Section headings formatieren
\usepackage[fixlanguage]{babelbib}
\usepackage[a4paper,lmargin={2.5cm},rmargin={2.5cm},tmargin={3cm},bmargin={2.5cm}]{geometry}
\usepackage{blindtext}
\pdfinfo{
	/Title		(Titel)
	/Subject	(Expose zur Bachelor-Studienjahres-/Master-/Diplomarbeit)
	/Author		(Max Mustermann)
}

\selectbiblanguage{german}
\allsectionsfont{\sffamily}
\captionsetup{margin=1cm,font=small,labelfont=bf}

\newcommand\notice[1]{}
\newcommand\seppar{ \vspace{2ex} \noindent }

\renewcommand*{\thesection}{\Roman{section}.}
\begin{document}

\title{{\bf Visualisierung von Metadaten als alternativer Sucheinstieg in den
        Katalog } \\ \begin{large}Exposé zur Masterarbeit                                                                             \end{large}}
\author{
	peter breternitz \\
	TH Wildau, Wildau Institute of Technology\\ Bibliotheksinformatik \\
	peter.breternitz@th-wildau.de
}
\date{\today}

\maketitle

\section{Einführung} 
\blindtext

\section{Problemstellung} 
\blindtext
Zitate: \cite{Harrison75a, Harrison78a}.

\section{Literaturübersicht}

\blindtext
\section{Eigener Ansatz und berücksichtigte Methoden} 
\blindtext


\bibliographystyle{alpha}
\bibliography{literature/lit} 


\end{document}
