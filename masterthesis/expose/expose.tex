\documentclass[10pt,a4paper,twocolumn,conference]{IEEEtran}

%\documentclass[12pt,a4paper]{article}
\usepackage[german]{babel}    % Deutsche Sprache in automatisch generiertem
\usepackage{latexsym}         % Fuer recht seltene Zeichen
\usepackage[utf8]{inputenc}   % =E4 =F6 =FC =DF; danach  geht auch das ß richtig
\usepackage{caption}          % Figure-Captions formatieren
\usepackage{sectsty}          % Section headings formatieren
\usepackage[fixlanguage]{babelbib}
\usepackage[a4paper,lmargin={2.5cm},rmargin={2.5cm},tmargin={3cm},bmargin={2.5cm}]{geometry}
\usepackage{blindtext}
\usepackage{bookmark}
\usepackage{verbatim}
\usepackage[numbers]{natbib}

\pdfinfo{
	/Title		(Titel)
	/Subject	(Expose zur Bachelor-Studienjahres-/Master-/Diplomarbeit)
	/Author		(Max Mustermann)
}
\selectbiblanguage{german}
\allsectionsfont{\sffamily}
\captionsetup{margin=1cm,font=small,labelfont=bf}
%\setlength\parindent{24pt}

\newcommand\notice[1]{}
\newcommand\seppar{ \vspace{2ex} \noindent }
\usepackage{hyperref}

%\renewcommand*{\thesection}{\Roman{section}.}
\begin{document}
%\nocite{*}
\title{{\bf Datengetriebenes Unterstützungssystem für zukünftige
        Etatplanung und Mittelallokation einer hypriden
        Spezialbibliotek} \\ 
    \begin{large}
        Exposé zur Masterarbeit                                                                             
    \end{large}}
\author{
	Peter Breternitz \\
	TH Wildau, Wildau Institute of Technology\\ Bibliotheksinformatik \\
	peter.breternitz@th-wildau.de
}


\date{\today}

\maketitle

\section{Einführung}
Ausgehend von ökonomischen, informationstechnologischen und marktpolitischen Einschnitten in den
letzten zwei Jahrzehnten\footnote{Zu nennen wären hier die Explosion der Zeitschriftenpreise im 
STM-Bereich,das Aufkommen von Epublishing und die Konzentration auf wenige
Verlage},
sind Bibliotheken dazu veranlasst, ihr Budget hinsichtlich der Informationsbedarfe
ihrer Nutzer:innen behutsamer zu planen und sich in zunehmenden Maß gegenüber ihren Unterhaltsträgern zu rechtfertigen.

Die Relevanz von bibliothekarischen Kennzahlen ist in diesem Zusammenhang größer geworden. 
Deswegen ist es wichtig Daten aus bibliothekarischen Servicedienstleistungen
und Geschäftsprozessen zu aggregieren, zu erheben und statistisch
auszuwerten, um auf Basis der daraus erzielten Erkenntnisse handeln zu können.

Ziel der zu entstehenden Arbeit ist die Entwicklung einer
interaktiven Business-Intelligence-Applikation als proof-of-concept,
mit der systematisch die relevanten Daten einer hypriden Spezialbibliothek aggregiert, statistisch
analysiert und mit geeigneten und modernen Datenvisualisierungstechniken\footnote{Visualisierung bezeichnet Disziplinen
wie Informationsvisualisierung, Datenvisualisierung und visuelle Analyse. Dabei wird sich konzentriert 
auf Ansätze die Visualisierungen mittels Visualisierungstechniken algorithmisch aus 
Daten erzeugen. dargestellt werden können. 
Visualisierungen können komplexe Sachverhalte herunterbrechen und können
so große Datenmengen im Gegensatz zu großen Tabellen leicht verständlich
darstellen.\cite{RN100}} ausgewertet und ausgegeben werden.
Vor allem soll sich hier auch auf automatisierte Prozesse zur Gewinnung der Ergebnisse konzentriert werden.
% Ziel dieses Tool

Mit diesen automatisch angefertigten statistischen Datenanalysen sollen zukünftige
Entscheidungen im Bibliotheksmanagement wie Erwerbungspolitik, Budgetplanung und 
Mittelallokation hinsichtlich der weiteren Entwicklung der 
Servicedienstleistungen evidenzbasiert und datengetrieben unterstützt werden.

Darüberhinaus soll die Applikation  eine Funktion beinhalten, ausgewählte
Resultate automatisiert als \textit{factsheet} zu exportieren, um diese 
als Rechenschaftsbericht gegenüber stakeholdern der Bibliothek präsentieren und
verhandeln zu können.

Die Notwendigkeit für diese Applikation ist durch das
Fehlen eines zentralen Nachweisortes für bibliothekarische
Statistiken in der Bibliothek gegeben. Da sie zudem verschiedene Recherche-Systeme den
Wissenschaftler:innen anbietet, wäre eine Engführung bezüglich der statistischen
Datenerhebung auf eine Plattform begrüßenswert.
Des Weiteren ist das Erforderniss bibliothekarische Geschäftsprozesse zu evaluieren und die
Servicedienstleistungen bezüglich der Ziele der Institution noch weiter zu
optimieren von großer Relevanz. Mit dem Ende der Konsolidierungsphase der
Bibliothek, die im Zuge des \textit{Max-Planck-Institutes für empirische
Ästhetik} 2014 gegründet wurde, tritt sie ein in eine Phase, in der ab dem Jahr
2021 Budgetplanungen eine größere Rolle spielen werden. 

Die zu entstehende Applikation soll hierbei helfen systematisches Controlling einzuführen und das
Bibliotheksmanagement weiter zu professionalisieren.
%predictive analysis\\
Um künftigen Anforderungen gewachsen zu sein, soll sie 
modulbasiert programmiert werden und dadurch leicht erweiterbar und/oder für 
andere Bibliotheken nachnutzbar sein.
%evidence based stock management\\

%Erschaffen eines Reports

\section{Literaturdiskussion}
Ein gutes Beispiel für ein datengetriebenes Unterstützungssystem findet sich in
der Literatur bei Spielberg, der sich mit dem Thema der Bestandspflege an der
Universitätsbibliothek Essen befasst und eine Applikation (weiter)entwickelt hat, die
die Fachreferent:innen bei der Aussonderung und Erwerbung
unterstützt.\cite{RN48}
Ebenso finden sich andere Beispiele, die vorrangig anhand einzelner
Fragestellungen hinsichtlich der Bestandsentwicklung\cite{RN28} oder anderer
bibliothekarischer Servicedienstleistungen\cite{RN43,RN41,RN45} verschiedene statistische Analysen
vollzogen und diese visualisiert haben.
Eine Ausnahme bildet die Entwicklung eines Dashboards an der \textit{New York
University Health Sciences Libraries}, das versucht möglichst viele Metriken
aus bibliothekarischen Dienstleistungen aufzunehmen.\cite{RN34}
Alle Projekte sind meist an größeren akademischen
Universitätsbibliotheken mit ganz unterschiedlichen softwaretechnischen
Herangehensweisen\cite{RN31,RN42} und Zielen\cite{RN1} entstanden.

Dennoch fehlen in der gesichenten Literatur Teile die sich mit den Budgetierung
explizit befassen und Auskunft über Mittelallokation geben. 

Zudem fehlt ein Beispiel in der Literatur, das holistisch alle relevanten Daten, die aus den
Geschäftsgängen und Servicedienstleistungen insbesondere einer
Spezialbibliothek entstehen,
in einem Dashboard aggregiert und auf diesen automatisch 
statistische Analysen ausführt und diese mit modernen Visualisierungstechniken
interaktiv darstellt.

\begin{comment}
\section{Literaturdiskussion1}
%Literatur zum Bibliotheksmanagement
%Tools

Office-Lösungen zum Einsatz. Diese Applikationen sind entweder 
an bestimmte Bibliothekssysteme zurückgebunden, limitiert in ihren
Funktionen\cite{RN47} oder nicht mehr zeitgemaß.
Trotz dem Vorhandenseins von Open-Source-Software in dem Bereich von Business
Intelligence sind solche Lösungen in Bibliotheken scheinbar noch nicht so
häufig vertreten.
Vielmehr scheint mit dem Einvernehmen wie wichtig datengetriebene
Unterstützungssysteme sind, ein nur punktuelles Ausprobieren
der Möglichkeiten einherzugehen.\cite{RN31} 
%Gründe dafür sind knappe Personalressourcen, das Fehlende Know-How in Bibliotheken und die
%Unattraktivität des öffentlichen Dienstes für Informatiker:innen.


%Bibliotheksmanagement - Evaluation von Beständen -> Verweis auf BIX
%Herausstellung das grafische Unterstützung gut ankommt bei Vorgesetzten und
%einem selber hilft Zsh. gut zu verstehen
programmierten Tools zu beantworten. So werden Fragen nach Weeding versucht zu
beantworten, oder nach de:update
m Erfolg von Lerngruppen in der Bibliothek...
Shiny, R -> sollen leicht erweiterbar sein
Versuche einzelner Fragestellungen werden verknüoft mit grafischen
Darstellungen -> Weeding, ... Erfolg Lerngruppen in Bibliotheken
Fachref aber nur für einen Anbieter geschriebenn
meistens für große Bibliotheken und bisher für keine wissenschaftliche
Spezialbibliothek -> personelle Ressourcen knapp -> Erleichterung bei der
Erfassung von Daten da automatisiert

%Untersuchungen zu library assessement finden sich in der Literatur zahlreiche.
%Es werden Bestandsgruppen, Bibliotheksservices evaluiert. Auch gibt es dass
%schon für Bestandsuaftbau. Diese einzelnen Stränge werden untersucht, aber ein
%zusammenbringen ist nirgendwo zu finden
%Meist sind es große Bibliotheken. Ein auf Spezialbibliotheken ausgerichtetes
%Tool fehlt -> Beziehen der Daten aus heterogenen Quellen
%Welche Daten sind sinnvoll zu erheben\\
%Analyse der zu betrachtenden Quellen an welche Daten komme ich ran?\\
%Wie sind diese aufgebaut?\\
%Welche statistischen Methoden sollen angewendet werden?\\
%Wie sieht eine adäquate Visualisierung aus?
%Ziel: Beobachtung der zu erhebenden Daten um daraus Rückschlüsse auf weitere
%bibliothekarische Service-Angebote zu ziehen. Budgetplanung, Übersicht über
%Bestandswachstum, Diversifikation in den einzelnen Bestandsgruppen
%Einbettung der Statistiken der MPDL\\
%internes Tool zur Dienstleistungssteuerung
%Report für Stackholders
%um eigene Arbeit zu evaluieren.
%Tätigkeitsbericht zu den stakeholdern
\end{comment}


\section{Problemstellung}
Die Bibliothek des \textit{Max-Planck-Institutes für empirische Ästhetik}
ist Teil des \textit{HeBis-Verbundes}. Die Geschäftsprozesse der Erwerbung und
der Katalogisierung findet im \textit{CBS} und im \textit{LBS4} von
\textit{OCLC} statt. Das  
lokale Bibliothekssystem wird gehostet und betreut vom Lokalsystem-Team Frankfurt.
Über die Bereitstellung besonderer Funktionalitäten seitens des
Lokalsystem-Teams, erhält die Bibliothek unter anderem Ausleih-, Budget- und
Umsatzübersichten per email als Text von diesem. 


Neben der Verankerung in der deutschen Bibliotheksverbundlandschaft
wird die Bibliothek in ihren Aufgaben von der
\textit{Max-Planck-Digital-Library} (mpdl)
unterstützt. Deren Portfolio umfasst vorrangig die zentrale Lizenzierung
von relevanten elektronischen Informationsressourcen, die Bereitstellung
von Softwarelösungen, das Betreiben eines Publikationsreposituriums und 
das Vorantreiben der OpenAccess-Initiative. Zudem stellt sie noch
COUNTER-Statistiken, die sie von den Verlagen bekommt zur Verfügung.

Außer den bereitgestellten Daten, erhebt die Bibliothek u.a. ebenfalls Daten über
die Frequentierung des Lesesaals, die Nutzung des nehmenden Fernleihservices, des
Dokumentenlieferdienstes \textit{subito} und des Bestandswachstums vor Ort.
Nach den Verantwortlichkeiten aufgeteilt, passiert dieses an verschiedenen virtuellen Orten. 
Eine systematische Auswertung der erhobenen Daten findet nur unzureichend statt. 
Daher regt sich der Wunsch seitens der Mitarbeiter:innen nach einem gemeinsamen Tool, 
womit übersichtlich und klar alle notwendigen nutzungs- und sammlungsbezogenen Statistiken für eine 
Spezialbibliothek erfasst und dargestellt werden können.\footnote{Zwar führt HeBis eine Bestandstatistik, diese ist aber insbesondere für die
Evaluation und Optimierung von Geschäftsprozessen einer Spezialbibliothek
insuffizient. \url{https://www.hebis.de/de/1ueber_uns/statistik/cbs_statistik.php}}

Es gibt eine Vielzahl von auf Business Intelligence Software basierenden
kommerziellen Lösungen, die auf den Bibliotheksbereich zu geschnitten sind. 
Zu nennen wären \textit{AlmaAnalytics} von \textit{ExLibris} für das
NextGenerationSystem \textit{Alma} 
\footnote{\url{https://www.exlibrisgroup.com/products/alma-library-services-platform/alma-analytics}
Stand: \today},
\textit{BibControl} von \textit{OCLC}\footnote{\url{https://www.oclc.org/de/bibcontrol.html} Stand: \today},
\textit{CollectionHq} von \textit{Baker \& Talor}\footnote{\url{https://www.collectionhq.com/} Stand: \today} von
oder \textit{Libinsight} von \textit{SpringShare}\footnote{\url{https://springshare.com/libinsight/} Stand: \today}.
Darüber hinaus gibt es Business Intelligence Applikationien, die von
Bibliotheken für Reporting, Datenanalyse und Datenvisualisierung adaptiert werden
wie zum Beispiel \textit{Tableau} von der Firma \textit{Tableau Software} oder
\textit{Crytal Reports} von \textit{SAP}.
Diese Applikationen sind entweder 
an bestimmte Bibliothekssysteme zurückgebunden, limitiert in ihren
Funktionen\cite{RN47} oder zu generisch.
Ferner wird sowohl von \textit{HeBis} bzw. von der
LokalBibliotheksSystem-Betreuung als auch von der \textit{mpdl} keine Applikation
in dieser Richtung angeboten.
Ebenso ist ungewiss, wann die Ablösung des schon betagten CBS/LBS hinzu
einem neuen Next-Generation-Library-System im \textit{Hebis-Verbund} stattfindet und ob
es Modul zur statistischen Datenerhebung liefert.



%Identifizieren von ... Punkten, die für die Datenerhebung interessant sind
%1... -> Warum ? Ziel?

%Rücken der Bestände -> Überprüfen wie Bestandsgruppen in letzten Jahren
%gewachsen sind um daraus abzuschätzen wieviel Platz sie zukünftig brauchen
%Das könnte man da stellen und hätte es auf einen Blick, -> herkömmlich: Abzug
%aus System -> mit OpenRefine Teile der Signatur, die für die Aufstellung
%relevant sind herausfiltern im Zsh. mit Zugangsnummer -> wird das ersichtlich

%-> Darstellung wie wächst jede Bestandsgruppe monatlich (NE), quartalsweise,
%halbjährlich, im jahr -> Balanxing out the stock


%Ziel ist es ein Tool zu erschaffen, dass 
\section{inhaltlicher Abriß der Masterarbeit}
Im Folgenden wird ein skizzenhafter Überblick gegeben, welche Punkte die
theoretische Konzeption und praktische Umsetzung beinhalten könnten.\\ 

\textbf{Theoretische Vorarbeit}\\
\begin{itemize}
    \item Identifizierung der sammlungs- und bestandsbezogenebn Geschäftsgänge und Servicediensleistungen und der in dem Zusammenhang stehenden Daten.
    \item Sind die Daten ausreichend, um auf ihnen sinnvolle Analysen vollführen zu
          können? In welchem Format sollen die Daten vorliegen für die statistischen
          Analysen? Wie müssen diese aufbereitet bzw. bereinigt werden?
    \item Wie sollen die Daten letztendlich vorgehalten werden?
    \item Wie sieht es mit Anbinung an LBS-Datenbank für die Abfrage von Live-Daten
          aus?
    \item Identifizierung der statitsischen Methoden. Welche Fragen werden an die Daten
          gestellt, welche Ergebnisse sollen diese liefern.
    \item Wahl des Frameworks für die Applikation/Dashboard. Begründung warum?
    \item Identifizierung geeigneter grafischer Visualisierungen. Welche
          Interaktionen soll die Applikation bieten.
    \item Identifizierung und Beschreibung der funktionalen und nichtfunktionalen
          Anforderungen der Applikation - Lastenheft
    \item praktische Überlegungen hinsichtlich der modularen Aufteilung in der
          Programmierung.\\
\end{itemize}


\textbf{Praktische Umsetzung}\\
\begin{itemize}
    \item Umsetzung des Lastenhefts
    \item modulares Programmieren für die einzelnen Geschäftsgangsdaten/Dienstleistungen
    \item Programmierung für die automatische Weiterverarbeitung der Daten
    \item Programmierung der Algorithmen für die statistischen Analysen
    \item Programmierung der Datenvisualisierungstechnologien
%Bibliothekarische Daten sind messy. Vor der Analyse bedürfen sie einer
%Aufbereitung.
\end{itemize}
\begin{comment}
\section{Abriß}
Ein grober Abriß der Masterarbeit bezüglich des Inhaltes und des Umfangs ist
unten stehend angegeben:\\
\textit{Einführung (ca. 5 S.)}:\\Tool mit dem man Suchergebnisse visualisieren kann\\
Tool für die kontrollierte Bestandsentwicklung\\Übersicht über die Kapitel - 1-
4 Sätze\\
\textit{Schluß (ca. 5 S.)}:\\
Resultat\\
Zusammenfassung\\
zukünftige Forschungfragen
\end{comment}            
\bibliographystyle{plain}
\bibliography{literature/lit} 


\end{document}
