\documentclass[10pt,a4paper,twocolumn,report]{IEEEtran}

%\documentclass[12pt,a4paper]{article}
\usepackage[german]{babel}    % Deutsche Sprache in automatisch generiertem
\usepackage{latexsym}         % Fuer recht seltene Zeichen
\usepackage[utf8]{inputenc}   % =E4 =F6 =FC =DF; danach  geht auch das ß richtig
\usepackage{caption}          % Figure-Captions formatieren
\usepackage{sectsty}          % Section headings formatieren
\usepackage[fixlanguage]{babelbib}
\usepackage[a4paper,lmargin={2.5cm},rmargin={2.5cm},tmargin={3cm},bmargin={2.5cm}]{geometry}
\usepackage{blindtext}
\usepackage{bookmark}
\usepackage{verbatim}
\usepackage[numbers]{natbib}

\pdfinfo{
	/Title		(Titel)
	/Subject	(Expose zur Bachelor-Studienjahres-/Master-/Diplomarbeit)
	/Author		(Max Mustermann)
}
\selectbiblanguage{german}
\allsectionsfont{\sffamily}
\captionsetup{margin=1cm,font=small,labelfont=bf}
%\setlength\parindent{24pt}

\newcommand\notice[1]{}
\newcommand\seppar{ \vspace{2ex} \noindent }
\usepackage{hyperref}

%\renewcommand*{\thesection}{\Roman{section}.}
\begin{document}
%\nocite{*}
\title{{Konzeption und Entwicklung eines datengetriebenen Unterstützungssystems für
        Etatplanung und Mittelallokation einer hybriden
        Spezialbibliothek}\\
    \begin{large}
        Exposé zur Masterarbeit
    \end{large}}
\author{
	Peter Breternitz \\
	TH Wildau, Wildau Institute of Technology\\ Bibliotheksinformatik \\
	peter.breternitz@th-wildau.de
}


\date{\today}

\maketitle

\section{Einführung}
Ausgehend von ökonomischen, informationstechnologischen und marktpolitischen Einschnitten in den
vergangenen Jahrzehnten\footnote{Als Gründe zu nennen wären hier: die Explosion der Zeitschriftenpreise im Bereich der
Science, Technology \& Medicine (STM), das Aufkommen von E-Publishing und die Konzentration auf wenige
Verlage},
sind Bibliotheken dazu veranlasst, ihr Budget hinsichtlich der Informationsbedarfe
ihrer Nutzer:innen behutsamer zu planen und sich in zunehmenden Maße gegenüber ihren Unterhaltsträgern zu rechtfertigen.

Die Relevanz von bibliothekarischen Kennzahlen ist in diesem Zusammenhang größer geworden.
Deswegen ist es wichtig, Daten aus bibliothekarischen Servicedienstleistungen
und Geschäftsprozessen zu aggregieren, zu erheben und statistisch
auszuwerten, um auf Basis der daraus erzielten Erkenntnisse handeln zu können.

Ziel der zu entstehenden Arbeit ist die Entwicklung einer
interaktiven Business-Intelligence-Applikation als proof-of-concept,
mit der systematisch die relevanten Daten einer hybriden Spezialbibliothek aggregiert, statistisch
analysiert und mit geeigneten und modernen Datenvisualisierungstechniken\footnote{Visualisierungen können komplexe Sachverhalte herunterbrechen und
so große Datenmengen - im Gegensatz zu großen Tabellen - leicht verständlich
darstellen. Im Kontext dieser Arbeit konzentriere ich mich auf Ansätze, die Visualisierungen mittels Visualisierungstechniken algorithmisch aus
Daten erzeugen (Informationsvisualisierung, Datenvisualisierung und visuelle Analyse).\cite{RN100}}
ausgegeben werden sollen.
Vor allem soll sich hier auf automatisierte Prozesse zur Gewinnung der Ergebnisse konzentriert werden.


Mit diesen automatisch angefertigten statistischen Datenanalysen sollen zukünftige
Entscheidungen im Bibliotheksmanagement wie Erwerbungspolitik, Budgetplanung und
Mittelallokation hinsichtlich der weiteren Entwicklung der
Servicedienstleistungen evidenzbasiert und datengetrieben unterstützt werden.

Darüber hinaus soll die Applikation  eine Funktion beinhalten, ausgewählte
Resultate automatisiert als \textit{factsheet} zu exportieren, um diese
als Rechenschaftsbericht gegenüber Stakeholdern der Bibliothek präsentieren zu können.

Die Notwendigkeit für diese Applikation ist durch das
Fehlen eines zentralen Nachweisortes für bibliothekarische
Statistiken in der Bibliothek gegeben. Da die Bibliothek zudem verschiedene Recherche-Systeme den
Wissenschaftler:innen anbietet, wäre eine Engführung der statistischen
Datenerhebung auf eine Plattform begrüßenswert.
Des Weiteren ist das Erfordernis, bibliothekarische Geschäftsprozesse zu evaluieren und die
Servicedienstleistungen bezüglich der Ziele der Institution noch weiter zu
optimieren, von großer Relevanz. Mit dem Ende der Konsolidierungsphase der
Bibliothek, die im Zuge des \textit{Max-Planck-Institutes für empirische
Ästhetik} 2014 gegründet wurde, tritt sie ein in eine Phase, in der ab dem Jahr
2021 Budgetplanungen eine größere Rolle spielen werden.

Die zu entstehende Applikation könnte hierbei helfen, systematisches Controlling einzuführen und das
Bibliotheksmanagement weiter zu professionalisieren.
%predictive analysis\\
Um künftigen Anforderungen gewachsen zu sein, soll sie
modulbasiert programmiert werden und dadurch leicht erweiterbar und eventuell von
anderen Bibliotheken nachnutzbar sein.
%evidence based stock management\\

%Erschaffen eines Reports

\section{Literaturdiskussion}
Ein gutes Beispiel für ein datengetriebenes Unterstützungssystem findet sich in
der Literatur bei Spielberg, der sich mit dem Thema der Bestandspflege an der
\textit{Universitätsbibliothek Essen} befasst und eine Applikation (weiter-)entwickelt hat, die
die Fachreferent:innen bei der Aussonderung und Erwerbung von Medien
unterstützt.\cite{RN48}
Ebenso finden sich in der Fachliteratur Ansätze, die vorrangig anhand einzelner
Fragestellungen hinsichtlich der Bestandsentwicklung\cite{RN28} oder anderer
bibliothekarischer Servicedienstleistungen\cite{RN43,RN41,RN45} verschiedene statistische Analysen
vollzogen und diese visualisiert haben.
Eine Ausnahme bildet die Entwicklung eines Dashboards an der \textit{New York
University Health Sciences Libraries}, das versucht, möglichst viele Metriken
aus bibliothekarischen Dienstleistungen aufzunehmen.\cite{RN34}
Fast alle Projekte sind an größeren
Universitätsbibliotheken mit ganz unterschiedlichen softwaretechnischen
Herangehensweisen\cite{RN31,RN42} und Zielen\cite{RN1} entstanden.

Dennoch fehlen in der gesichteten Literatur Teile, die sich mit der Budgetierung
befassen und Auskunft über Mittelallokation geben.

Zudem fehlt ein Beispiel in der Literatur, das holistisch alle relevanten Daten, die in den
Geschäftsgängen und Servicedienstleistungen insbesondere einer Spezialbibliothek entstehen,
aggregiert, auf diesen Daten automatisch statistische Analysen ausführt und diese mit modernen Visualisierungstechniken
interaktiv darstellt.


\section{Problemstellung}
Die Bibliothek des \textit{Max-Planck-Institutes für empirische Ästhetik}
ist Teil des \textit{hessischen Bibliotheksverbundes (HeBis)}. Die Geschäftsprozesse
der Katalogisierung und der Erwerbung finden im Zentralsystem \textit{CBS} und im im Lokalsystem \textit{LBS4} von
\textit{OCLC} statt. \textit{LBS4} wird gehostet und betreut vom Lokalsystem-Team Frankfurt. Als Service-Leistung werden der Bibliothek besondere Funktionalitäten
für das \textit{CBS} bereitgestellt. Außerdem erhält die Bibliothek unter anderem Ausleih-, Budget- und
Umsatzübersichten als Text per E-mail zugesandt.


Neben der Verankerung in der deutschen Bibliotheksverbundlandschaft
wird die Bibliothek in ihren Aufgaben von der
\textit{Max Planck Digital Library (mpdl)}
unterstützt. Deren Portfolio umfasst vorrangig die zentrale Lizenzierung
von relevanten elektronischen Informationsressourcen, die Bereitstellung
von Softwarelösungen, das Betreiben eines Publikationsrepositoriums und
das Vorantreiben von Open-Access. Zudem stellt sie den Bibliotheken der einzelnen Max-Planck-Institute
\textit{COUNTER}-Statistiken zur Verfügung, die von den Verlagen geliefert werden.

Außer diesen bereitgestellten Daten erhebt die Bibliothek Daten unter anderem über
die Frequentation des Lesesaals, die Nutzung des nehmenden Fernleihservices, des
Dokumentenlieferdienstes \textit{subito} und des Bestandswachstums vor Ort.
Nach den unterschiedlichen Verantwortlichkeiten aufgeteilt, werden diese Daten an verschiedenen virtuellen Orten erhoben.
Eine systematische Auswertung der Daten findet nur unzureichend statt.
Daher regt sich der Wunsch seitens der Bibliotheksleitung und der Mitarbeiter:innen nach einem gemeinsamen Tool,
mit dem übersichtlich und klar alle notwendigen nutzungs- und sammlungsbezogenen Statistiken einer
Spezialbibliothek erfasst und dargestellt werden können.\footnote{Zwar führt \textit{HeBis} eine Bestandsstatistik, diese ist aber insbesondere für die
Evaluation und Optimierung von Geschäftsprozessen einer Spezialbibliothek
insuffizient. \url{https://www.hebis.de/de/1ueber_uns/statistik/cbs_statistik.php} Auch an der deutschen Bibliotheksstatistik nimmt die Bibliothek nicht teil. Beide bieten zudem nur Zahlenkolonnen und keine weiteren Visualisierungen an.}

Es gibt eine Vielzahl kommerzieller Lösungen für den Bibliotheksbereich, die auf Business-Intelligence-Software basieren.
Zu nennen wären \textit{AlmaAnalytics} für das
Next-Generation-Library-System \textit{Alma} von \textit{ExLibris}\footnote{\url{https://www.exlibrisgroup.com/products/alma-library-services-platform/alma-analytics}
Stand: 26.05.2020}, \textit{BibControl} von \textit{OCLC}\footnote{\url{https://www.oclc.org/de/bibcontrol.html} Stand: 26.05.2020},
\textit{CollectionHq} von \textit{Baker \& Taylor}\footnote{\url{https://www.collectionhq.com/} Stand: 26.05.2020} oder \textit{Libinsight} von \textit{SpringShare}\footnote{\url{https://springshare.com/libinsight/} Stand: 26.05.2020}.
Darüber hinaus gibt es Business-Intelligence-Applikationen, die von
Bibliotheken für Reporting, Datenanalyse und Datenvisualisierung adaptiert werden,
wie zum Beispiel \textit{Tableau} von der Firma \textit{Tableau Software} oder
\textit{Crystal Reports} von \textit{SAP}.
Diese Applikationen sind entweder
an bestimmte Bibliothekssysteme zurückgebunden, limitiert in ihren
Funktionen\cite{RN47} oder zu generisch.
Überdies wird sowohl von \textit{HeBis} bzw. von der
Lokal-Bibliothekssystembetreuung als auch von der \textit{mpdl} keine Applikation
in dieser Richtung angeboten.
Ebenso ist ungewiss, wann die Ablösung des schon betagten \textit{CBS/LBS} hin zu
einem neuen Next-Generation-Library-System im \textit{HeBis-Verbund} stattfinden wird und ob
es ein Modul zur statistischen Datenerhebung liefern wird.

\section{inhaltlicher Abriss der Masterarbeit}
Im Folgenden wird ein skizzenhafter Überblick gegeben, welche Punkte die
theoretische Konzeption und praktische Umsetzung beinhalten könnten.\\

\textbf{Theoretische Konzeption}\\
\begin{itemize}
    \item Identifizierung der sammlungs- und bestandsbezogenen Geschäftsgänge und Servicedienstleistungen und der in diesem Zusammenhang stehenden Daten.
    \item Sind die Daten ausreichend, um auf ihnen sinnvolle Analysen vollführen zu
          können? In welchem Format sollten die Daten für die statistischen
          Analysen vorliegen? Wie müssen diese aufbereitet bzw. bereinigt werden? Gibt es ggf. auch Aspekte des Datenschutzes, die zu berücksichtigen sind.
    \item Wie sollen die Daten in der Applikation vorgehalten werden?
    \item Wie sieht die Anbindung an LBS-Datenbank für die Abfrage von Live-Daten
          aus?
    \item Identifizierung der statistischen Methoden. Welche Fragen werden an die Daten
          gestellt, welche Antworten sollen diese liefern?
    \item Wahl des Frameworks für die Applikation/Dashboard. Begründung warum?
    \item Identifizierung geeigneter grafischer Visualisierungen. Welche
          Interaktionen soll die Applikation bieten? Kleine Bedarfsanalyse mit den Kolleg:innen.
    \item Identifizierung und Beschreibung der funktionalen und nicht funktionalen
          Anforderungen der Applikation - Lastenheft. Kleine Bedarfsanalyse mit den Kolleg:innen.
    \item Praktische Überlegungen hinsichtlich der modularen Aufteilung in der
          Programmierung.\\
\end{itemize}


\textbf{Praktische Umsetzung}\\
\begin{itemize}
    \item Prototypische Umsetzung ausgewählter Anforderungen des Lastenheftes nach Priorisierung der Anforderungen.
    \item Programmierung nach modularem Architekturansatz, d.h. die Integration neuer Datenquellen, die Umsetzung neuer Auswertungen und neuer Visualisierungen
					sollen mit geringem Aufwand erfolgen.
    \item Programmierung für die automatische Weiterverarbeitung von ausgewählten Daten.
    \item Programmierung der Algorithmen von ausgewählten statistischen Analysen.
    \item Programmierung von ausgewählten Datenvisualisierungstechnologien.
\end{itemize}

\bibliographystyle{plain}
\bibliography{literature/lit}


\end{document}
