\documentclass[10pt,a4paper,twocolumn,conference]{IEEEtran}

\usepackage[german]{babel}    % Deutsche Sprache in automatisch generiertem
\usepackage{latexsym}         % Fuer recht seltene Zeichen
\usepackage[utf8]{inputenc}   % =E4 =F6 =FC =DF; danach  geht auch das ß richtig
\usepackage{caption}          % Figure-Captions formatieren
\usepackage{sectsty}          % Section headings formatieren
\usepackage[fixlanguage]{babelbib}
\usepackage[a4paper,lmargin={2.5cm},rmargin={2.5cm},tmargin={3cm},bmargin={2.5cm}]{geometry}
\usepackage{blindtext}

\pdfinfo{
	/Title		(Titel)
	/Subject	(Expose zur Bachelor-Studienjahres-/Master-/Diplomarbeit)
	/Author		(Max Mustermann)
}

\selectbiblanguage{german}
\allsectionsfont{\sffamily}
\captionsetup{margin=1cm,font=small,labelfont=bf}
%\setlength\parindent{24pt}

\newcommand\notice[1]{}
\newcommand\seppar{ \vspace{2ex} \noindent }
\usepackage{hyperref}

%\renewcommand*{\thesection}{\Roman{section}.}
\begin{document}
\nocite{*}
\title{{\bf Visualisierung von bibliografischen Metadaten als zusätzlichen Einstieg in die
        Recherche nach Medien } \\ 
    \begin{large}
        Exposé zur Masterarbeit                                                                             
    \end{large}}
\author{
	Peter Breternitz \\
	TH Wildau, Wildau Institute of Technology\\ Bibliotheksinformatik \\
	peter.breternitz@th-wildau.de
}


\date{\today}

\maketitle

\section{Einführung}
Gedächtnisinstitutionen wie Bibliotheken verfügen über einen Schatz 
hochqualitativer bibliografischer Metadaten. Diese werden in klassischen 
Online-Katalogen oder in Discovery-Systemen zur Anzeige gebracht. 
Discovery-Sytemen basieren auf der Suchmaschinentechnologie und zeichnen sich 
unter anderem durch einen großen Suchraum, eine intuitive Bedienbarkeit und
einem Ranking der Treffer nach Relevanz aus. Neben einer Facettierung, 
mit der die gefundenen Ergebnisse einer Suchanfrage fein justiert werden 
können, scheinen Discovery-Systeme hinsichtlich der Visualisierung von Suchergebnissen 
nicht mehr zu bieten als die klassischen Kataloge. Vor der Folie
des Semantic Webs und dessen Technologien wie zum Beispiel RDF wäre die Frage
hier, ob Bibliotheken alle Potentiale ihres Schatzes für die Darstellung von
Suchergebenissen ausschöpfen? Oder anders: gibt es Alternativen, die dass 
explorative Suchen fördern, die Ergebnisse zu neuen Wissensräume zusammenschließen 
und Erkenntnisse zu Tage bringen, die in der klassischen Suchanzeige verloren
gehen beziehungsweise nicht angezeigt werden können. Mit diesen Topoi
beschäftigt sich die zu enstehende Arbeit in einem ersten theoretischen Schritt.
Nachfolgend soll in einem praktischen Teil ein proof-of-concept anhand 
bibliografischer Metadaten einer Gedächntnisinstitution entwickelt werden, dass
als Ergebnis einen interaktiven, explorativen und virtuellen Zugang zu den
Beständen liefert. Dabei geht es nur um ein zusätzliches Angebot an
die WissenschaftlerInnen. Darüberhinaus könnte das tool als Werkzeug für den
Bestandsaufbau hinsichtlich  eines ausgeglichenen Wachstums zum
Beispiel der Bestandsgruppen fungieren. 

\section{Problemstellung}
Hier sollen theoretische Annahmen und Konzepte kurz eingeführt und erklärt
werden. Aus bibliothekarischer Perspektive werden in einem Teil wichtige
Konzepte wie das Semantic Web und dessen Technologien erklärt. Stichpunkte und
Fragen für diesen \textit{Theorie-I-Teil} sind:\\
Semantic Web, FRBR, LRM, RDF, LOD, Warum sind Metadaten von
Bibliotheksbeständen besonders für Visualisierungen geeignet? Catalogue
Enrichment.\cite{RN8}\\ Für den zweiten Theorieteil werden Schlaglichter auf Data
Science, Digital Humanities geworfen. Dabei spielen auch die verwendeten
Technologien eine Rolle. Stichpunkte sind für diesen \textit{Theorie-II-Teil:}\\
Visualisierungen, Grenzen und Möglichkeiten, Interaktive Grafiken,
Data Science Life Cycle

\section{Literaturübersicht}
Die Literaturdiskussion führt die beiden Theoriestränge aus dem vorhergehenden
Kapitel in den \textit{Theorie-III-Teil} zusammen und diskutiert anhand 
von Literatur aus der bibliothekarischen Praxis bereits vorhandene Beispiele. 
Dabei wird hierbei auch die Frage der technischen Umsetzung eine Rolle spielen,
aber auch auf insbesondere eventuelle Leerstellen hingewiesen.

\section{Eigener Ansatz und berücksichtigte Methoden}
Entwicklung Proof-of-Conceptes

\section{Abriß}
Ein grober Abriß der Masterarbeit bezüglich des Inhaltes und des Umfangs ist im
Folgendem angegeben:\\
\textit{Einleitung (5 S.)}:\\ Tool mit dem man Suchergebnisse visualisieren kann\\
Tool für die kontrollierte Bestandsentwicklung Übersicht über die Kapitel - 1- 4 Sätze, was in den nächsten Kapiteln dargelegt wird\\
Darlegung warum es sinnvoll ist\\
\textit{Theoretische Teil I (10 S.)}:\\
Semantic Web Technoliogien\\
FRBR, LRM, RDF, Linked Open Data, Catalogue Enrichment\\
Ausloten der Möglichkeiten, Warum eignen sich Metadaten aus Bibliotheken und anderen Kultureinrichtungen besonders\\
\textit{Theoretische Teil II (15 S.)}:\\
Grenzen und Möglichkeiten von Datenvisualisierungen vor dem
Hintergrund Data Science und Digital Humanities\\
Interaktive Visualisierungen\\ 
\textit{Theoretische Teil III (15 S.)}:\\
Literaturdiskussion Bibliotheken - Metadaten - Visualisierungen\\
Beispiele - Technologien\\
\textit{Praktischer Teil (15 S.)}:\\
Entwicklung eines proof-of-Conceptes anhand von Metadaten einer
Bibliothek oder einer Kulturinstitution\\
Aufzeigen des Workflows\\
Theoretisches Modell der Data Science anwenden auf das Projekt\\
praktische Methoden - Wie und mit was programmieren\\
Auswertung welcher Metadatenfelder\\
Catalogue Enrichment

            
            
\bibliographystyle{alpha}
\bibliography{literature/lit} 


\end{document}
