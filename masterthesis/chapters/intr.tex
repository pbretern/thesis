\chapter{Einführung}
Ausgehend von ökonomischen, informationstechnologischen und marktpolitischen Einschnitten in den
vergangenen Jahrzehnten\footnote{Als Gründe zu nennen wären hier: die Explosion der Zeitschriftenpreise im Bereich der
Science, Technology \& Medicine (STM), das Aufkommen von E-Publishing und die Konzentration auf wenige
Verlage},sind Bibliotheken dazu veranlasst, ihr Budget hinsichtlich der Informationsbedarfe
ihrer Nutzer:innen behutsamer zu planen und sich in zunehmenden Maße gegenüber ihren Unterhaltsträgern zu rechtfertigen.

Die Relevanz von bibliothekarischen Kennzahlen ist in diesem Zusammenhang größer geworden.
Deswegen ist es wichtig, Daten aus bibliothekarischen Servicedienstleistungen und Geschäfts-prozessen zu aggregieren, zu erheben und statistisch
auszuwerten, um auf Basis der daraus erzielten Erkenntnisse handeln zu können.

\section{Problemstellung}

\section{Ziel der Arbeit}
Ziel der zu entstehenden Arbeit ist die Entwicklung einer
interaktiven Business-Intelligence-Applikation als proof-of-concept,
mit der systematisch die relevanten Daten einer hybriden Spezialbibliothek aggregiert, statistisch
analysiert und mit geeigneten und modernen Datenvisualisierungstechniken\footnote{Visualisierungen können komplexe Sachverhalte herunterbrechen und
so große Datenmengen - im Gegensatz zu großen Tabellen - leicht verständlich
darstellen. Im Kontext dieser Arbeit konzentriere ich mich auf Ansätze, die Visualisierungen mittels Visualisierungstechniken algorithmisch aus
Daten erzeugen (Informationsvisualisierung, Datenvisualisierung und visuelle Analyse).\cite{RN100}}
ausgegeben werden sollen.
Vor allem soll sich hier auf automatisierte Prozesse zur Gewinnung der Ergebnisse konzentriert werden.

Mit diesen automatisch angefertigten statistischen Datenanalysen sollen zukünftige
Entscheidungen im Bibliotheksmanagement wie Erwerbungspolitik, Budgetplanung und
Mittelallokation hinsichtlich der weiteren Entwicklung der
Servicedienstleistungen evidenzbasiert und datengetrieben unterstützt werden.

Darüber hinaus soll die Applikation  eine Funktion beinhalten, ausgewählte
Resultate automatisiert als \textit{factsheet} zu exportieren, um diese
als Rechenschaftsbericht gegenüber Stakeholdern der Bibliothek präsentieren zu können.

\section{Verwandte Arbeiten}
Es gibt eine Vielzahl kommerzieller Lösungen für den Bibliotheksbereich, die auf Business-Intelligence-Software basieren.
Zu nennen wären \textit{AlmaAnalytics} für das
Next-Generation-Library-System \textit{Alma} von \textit{ExLibris}\footnote{\url{https://www.exlibrisgroup.com/products/alma-library-services-platform/alma-analytics}
Stand: 26.05.2020}, \textit{BibControl} von \textit{OCLC}\footnote{\url{https://www.oclc.org/de/bibcontrol.html} Stand: 26.05.2020},
\textit{CollectionHq} von \textit{Baker \& Taylor}\footnote{\url{https://www.collectionhq.com/} Stand: 26.05.2020} oder \textit{Libinsight} von \textit{SpringShare}\footnote{\url{https://springshare.com/libinsight/} Stand: 26.05.2020}.
Darüber hinaus gibt es Business-Intelligence-Applikationen, die von
Bibliotheken für Reporting, Datenanalyse und Datenvisualisierung adaptiert werden,
wie zum Beispiel \textit{Tableau} von der Firma \textit{Tableau Software} oder
\textit{Crystal Reports} von \textit{SAP}.
Diese Applikationen sind entweder
an bestimmte Bibliothekssysteme zurückgebunden, limitiert in ihren
Funktionen\cite{RN47} oder zu generisch.
Überdies wird sowohl von \textit{HeBis} bzw. von der
Lokal-Bibliothekssystembetreuung als auch von der \textit{mpdl} keine Applikation
in dieser Richtung angeboten.
Ebenso ist ungewiss, wann die Ablösung des schon betagten \textit{CBS/LBS} hin zu
einem neuen Next-Generation-Library-System im \textit{HeBis-Verbund} stattfinden wird und ob
es ein Modul zur statistischen Datenerhebung liefern wird.
Ein gutes Beispiel für ein datengetriebenes Unterstützungssystem findet sich in
der Literatur bei Spielberg, der sich mit dem Thema der Bestandspflege an der
\textit{Universitätsbibliothek Essen} befasst und eine Applikation (weiter-)entwickelt hat, die
die Fachreferent:innen bei der Aussonderung und Erwerbung von Medien
unterstützt.\cite{RN48}
Ebenso finden sich in der Fachliteratur Ansätze, die vorrangig anhand einzelner
Fragestellungen hinsichtlich der Bestandsentwicklung\cite{RN28} oder anderer
bibliothekarischer Servicedienstleistungen\cite{RN43,RN41,RN45} verschiedene statistische Analysen
vollzogen und diese visualisiert haben.
Eine Ausnahme bildet die Entwicklung eines Dashboards an der \textit{New York
University Health Sciences Libraries}, das versucht, möglichst viele Metriken
aus bibliothekarischen Dienstleistungen aufzunehmen.\cite{RN34}
Fast alle Projekte sind an größeren
Universitätsbibliotheken mit ganz unterschiedlichen softwaretechnischen
Herangehensweisen\cite{RN31,RN42} und Zielen\cite{RN1} entstanden.

Dennoch fehlen in der gesichteten Literatur Teile, die sich mit der Budgetierung
befassen und Auskunft über Mittelallokation geben.

Zudem fehlt ein Beispiel in der Literatur, das holistisch alle relevanten Daten, die in den
Geschäftsgängen und Servicedienstleistungen insbesondere einer Spezialbibliothek entstehen,
aggregiert, auf diesen Daten automatisch statistische Analysen ausführt und diese mit modernen Visualisierungstechniken
interaktiv darstellt.

\section{Gliederung der Arbeit}

