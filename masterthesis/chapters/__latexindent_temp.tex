\chapter{Ausgangssituation}

Im folgenden Kapitel wird die wissenschaftliche Spezialbibliothek des \textit{Max-Planck-Institutes für empirische Ästhetik} 
porträtiert. Anschließend werden die bibliothekarischen Informationsdienstleistungen der Bibliothek 
skizziert und der Frage nachgegangen, welche statistischen Daten aggregiert und ausgewertet wurden. 

\section{Bibliothek}
\subsection{Allgemeines}
%Wann wurde die Bibliothek gegründet?\\
Die Spezialbibliothek wurde im Zuge der Gründung des \textit{Max-Planck-Institutes für empirische Ästhetik} 
in Frankfurt im Jahr 2013 gegründet. Die Aufgabe des Institutes ist die interdisziplinäre Erforschung 
empirischer Fragestellungen der Ästhetik. Das Institut besteht aus drei Abteilungen \textit{Sprache und Literatur}, 
\textit{Musik} und \textit{Neurowissenschaften} sowie einigen Forschungsgruppen. %Ungefähr 150 Mitarbeiter:innen 
%arbeiten in dem Institut. 



% Warum wurde die Bibliothek gegründet?
%Welche Aufgaben hat die Bibliothek?
%Welche Servicedienstleistungen bietet die Bibliothek an?
Die Bibliothek ist eine Serviceeinrichtung des Institutes und dient mit ihren Informationsdienstleistungen 
der Forschung.
Zentral ist dabei die Informationsversorgung der Forschenden. Die benötigten Informationen sind Bücher, 
Zeitschriften, Zeitschriftenartikel sowohl in gedruckter als auch in elektronischer Form.
Seit der Institutsgründung wird neben des nutzungsorientierten Bestandaufbaus ebenfalls eine planmäßige 
Bestandsentwicklung betrieben. Das Erwerbungsprofil der Bibliothek leitet sich aus dem Forschungsauftrag des Institutes 
ab und umfasst dementsprechend die Erwerbung von Informationsressourcen, die sich den theoretischen und 
empirischen Fragestellungen der Ästhetik widmen.

%Wie groß ist der Bibliotheksbestand? 
Der Bibliotheksbestand ist hybrid. Er besteht sowohl aus gedruckten als auch Online-Medien sowie 
audiovisuellen Materialien. An Bestand umfasst die Bibliothek zirka 11.000 Bücher, 30 laufende Zeitschriften, 
knapp über 200 audiovisuelle Medien sowie die Lizenzierung von Online-Datenbanken
und Online-Zeitschriften.

\subsection{Weiterer Organisationsrahmen}
%Wie sieht der größere Organisationsrahmen der Bibliothek aus?\\
Um alle Informationsbedarfe der Forscher:innen zu befriedigen, wird die Bibliothek in ihren Aufgaben von der
\acrfull{mpdl} unterstützt. Deren Portfolio umfasst vorrangig die zentrale 
Lizenzierung von relevanten elektronischen Informationsressourcen, die Bereitstellung von Softwarelösungen, 
das Betreiben des Publikationsrepositoriums \acrshort{PuRe.MPG} der \acrfull{MPG} und
das Vorantreiben von Open-Access. 

Darüber hinaus ist die Spezialbibliothek Teil des \textit{hessischen Bibliotheksverbundes} \acrshort{hebis}. 
Die Geschäftsprozesse der Katalogisierung und der Erwerbung finden im \acrfull{CBS} und 
im Lokalsystem \acrfull{LBS} von \acrfull{OCLC} statt. \acrshort{LBS} wird gehostet und betreut vom Lokalsystem-Team Frankfurt. 
Als Service-Leistung werden der Bibliothek besondere Funktionalitäten für das \acrlong{CBS} bereitgestellt.


\subsection{Informationsdienstleistungen}
Das Bibliotheks-Team ist verantwortlich für den Ablauf und Organisation der bibliothekarischen Informationsdienstleistungen, 
die der Informationsversorgung dienen. Eine Übersicht der Informationsdienstleistungen nach den Basisfunktionen\cite[S. 204 f.]{RN200} einer Bibliothek 
zeigt \autoref{tab:Informationsdienstleistungen}. Die zentralen Informationsdienstleistungen der Spezialbibliothek bestehen 
aus dem Benutzungsservice und der Sammeltätigkeit. 
\begingroup
\setlength{\tabcolsep}{12pt} % Default value: 6pt
\renewcommand{\arraystretch}{1.5} 
\begin{table}[h]
    \centering
    \begin{adjustbox}{max width=\textwidth}
    \begin{tabular}{p{0.4\textwidth}p{0.9\textwidth}}
       \toprule
       \textbf{Basisfunktion}          & \textbf{Beschreibung}\\
       \midrule
        Benutzung                               &Ausleihe, Lesesaalnutzung, Organisation der Lieferdienste (Fern und Ortsleihe, Dokumentenlieferdienste)\\
        DV Management                           &PuRe, Medien-Datenbank\\
        Ordnen                                  &Aufstellungssystematik\\
        %Sammeln bes. Materialien                &Stimulus wie Musikstücke, Bilder, Textlizenzen, Zeitschriftenartikel\\
        Sammeln und Erschließen                 &geplanter Bestandsaufbau, Integrierter Geschäftsgang Medienerwerbung und Medienerschließung, besondere Materialien\\
        Vermitteln                              &Literaturrecherche, Nutzung elektronischer Ressourcen, Urheberrecht und Publikationsberatung\\
   
       \bottomrule
    \end{tabular}
    \end{adjustbox}
    \caption
    \end{table}
\endgroup

Die Dienstleistungsbereiche der Benutzung sind zuständig für die Organisation der Fern- und Ortsleihe von Informationsressourcen 
aus anderen Bibliotheken, die nicht in das Erwerbungsprofil der Spezialbibliothek fallen. Ferner sind 
diese Informationsdienstleistungen für die Informationsbeschaffung sowohl über Dokumentenlieferdienste als auch für die 
Akquise von einzelnen Zeitschriftenaufsätzen zuständig. Weitere Informationsdienstleistungen sind
die Betreuung des Publikationsrepositoriums \textit{PuRe.MPG} des Institutes, spezielle Beratungsdienstleistungen 
zum Urheberrecht und zum Publishing sowie klassische Auskunfts- und Informationsdienste. Seit 2016 geschieht die Ausleihe der
Medien über ein Selbstverbuchungssystem.


\subsection{Datengrundlage}

Zu fast jeder Informationsdienstleistung der Spezialbibliothek werden quantitative Daten generiert. 
Auf \autoref{tab:Informationsdienstleistungen} aufbauend zeigt \autoref{tab:Statistische_Daten} aggregierte und
ausgewertete Daten der Informationsdienstleistungen. Die Tabelle stellt zudem dar in welchem Format die Daten vorliegen, 
wie regelmäßig die Daten erfasst werden, die Quelle aus der die Daten stammen und ob die Daten bereits ausgewertet und visualisiert wurden.


\begingroup
\setlength{\tabcolsep}{4pt} % Default value: 6pt
\renewcommand{\arraystretch}{1.5}
%\resizebox{\textwidth}{!}{
\begin{table}[h]
    \centering
    \begin{adjustbox}{max width=\textwidth}
    \LARGE
    \begin{tabular}{lllllcc}
      %\begin{tabular}{p{3cm}p{5cm}p{1cm}p{1.5cm}p{2cm}p{4cm}}
       \toprule
       \textbf{Basisfunktion}               &\textbf{Daten}                             &\textbf{Turnus}    &\textbf{Quelle}  &\textbf{Format}          &\textbf{Syst. Auswertung} & \textbf{Visualisierungen}\\
       \midrule     
            Benutzung                       & Ausleihe aus dem Bestand                  & unregelmäßig      & LBS          & Mail, xlsx                & nein  & -\\
            Benutzung                       & Ausleihe Lieferdienste                    & monatlich         & intern       & xlsx                      & ja    & teilweise, Liniendiagramm\\ 
            Benutzung                       & Besonders nachgefragte Medien (OPAC)      & monatlich         & LBS          & Mail, txt                 & nein  & -\\ 
            Benutzung                       & Lesesaalnutzung                           & monatlich         & intern       & xlsx                      & nein  & -\\ 
            %Sammeln bes. Materialien        & Anzahl der Materialien                     & unregelmäßig      & intern       & xlsx                      & nein  & -\\
            Sammeln                         & Ausgaben nach Kostenstellen               & monatlich         & LBS          & Mail, txt                 & ja    & -\\ 
            Sammeln                         & Ausgaben nach Lieferanten                 & monatlich         & LBS          & Mail, txt                 & ja    & Balken und Kreisddiagramm\\ 
            Sammeln                         & Größe und Art des Bestandes               & unregelmäßig      & LBS, intern  & csv                       & nein  & -\\ 
            Sammeln                         & Neuerwerbungslisten                       & unregelmäßig      & LBS, intern  & tsv                       & nein  & -\\ 
            Vermitteln                      & COUNTER 5-Statistiken elektr. Ressourcen  & -                 & mpdl         & csv, tsv, txt             & nein  & -\\ 

        \bottomrule
    \end{tabular}
    \end{adjustbox}
    \caption
    \label{tab:Statistische_Daten}
    }
     \end{table}
\endgroup


Die statistischen Daten der Ausleihe aus dem Bestand werden der Bibliothek bei Bedarf durch das LBS abgezogen. 
Monatliche Ausleihstatistiken müssen mit dem LBS-Team noch abgestimmt werden. Die Ausleihzahlen liegen für 
den Zeitraum von 2016 - 2020 kumaltiv, nach der internen  Identifikationsnummer des Titeldatensatzes im 
\acrshort{CBS} und Jahr und als Rohdaten vor.

Monatlich bekommt die Bibliothek Statistiken über die Ausgaben nach Kostenstellen und Lieferanten zugeschickt. 
Die Kostenstellen bilden die einzelnen Abteilungen und Forschungsgruppen des Institutes ab. 
Erhoben werden nur die Daten der Ausgaben nach Lieferanten, um die Ausgabenverteilung zu steuern. 

Intern erfasst die Bibliothek monatlich die Daten der Ausleihe über die Lieferdienste. Unterschieden 
wird in der Erfassung nach Materialart der Medien, nach Ausleihort und Ausleihart. Zudem werden manuell Nutzungsstatistiken
des Lesesaals geführt.

Die von der \acrshort{mpdl} zur Verfügung gestellten COUNTER 5-Statistiken können über ein internes 
Portal abgerufen werden. Die Statistiken erstrecken sich vom Zugriff auf eJournals bis zum Zugriff auf ebooks.
Die Statistiken werden zu unregelmäßigen Zeitpunkten von den Verlagen bereitgestellt. 
Die Statistiken wurden bisher von der Bibliothek nicht beachtet.





\clearpage
Welche bibliothekarischen GG gibt es?\\
Welche statistischen Daten sind schon vorhanden?\\
In welchem Format liegen die statistischen Daten vor?\\
Welche statistischen Auswertungen gibt es?\\
Welche statistischen Auswertungen soll es noch geben?\\
Welche grafischen Auswertungen gibt es?\\
Welche grafischen Auswertungen soll es noch geben?



Die Bibliothek des \textit{Max-Planck-Institutes für empirische Ästhetik}
ist Teil des \textit{hessischen Bibliotheksverbundes (hebis)}. Die Geschäftsprozesse
der Katalogisierung und der Erwerbung finden im Zentralsystem \textit{CBS} und im im Lokalsystem \textit{LBS4} von
\textit{OCLC} statt. \textit{LBS4} wird gehostet und betreut vom Lokalsystem-Team Frankfurt. Als Service-Leistung werden der Bibliothek besondere Funktionalitäten
für das \textit{CBS} bereitgestellt. Außerdem erhält die Bibliothek unter anderem Ausleih-, Budget- und
Umsatzübersichten als Text per E-Mail zugesandt.


Neben der Verankerung in der deutschen Bibliotheksverbundlandschaft
wird die Bibliothek in ihren Aufgaben von der
\textit{Max Planck Digital Library (mpdl)}
unterstützt. Deren Portfolio umfasst vorrangig die zentrale Lizenzierung
von relevanten elektronischen Informationsressourcen, die Bereitstellung
von Softwarelösungen, das Betreiben eines Publikationsrepositoriums und
das Vorantreiben von Open-Access. Zudem stellt sie den Bibliotheken der einzelnen Max-Planck-Institute
\textit{COUNTER}-Statistiken zur Verfügung, die von den Verlagen geliefert werden.



Eine systematische Auswertung der Daten findet nur unzureichend statt.
Daher regt sich der Wunsch seitens der Bibliotheksleitung und der Mitarbeiter:innen nach einem gemeinsamen Tool,
mit dem übersichtlich und klar alle notwendigen nutzungs- und sammlungsbezogenen Statistiken einer
Spezialbibliothek erfasst und dargestellt werden können.\footnote{Zwar führt \textit{hebis} eine Bestandsstatistik, diese ist aber insbesondere für die
Evaluation und Optimierung von Geschäftsprozessen einer Spezialbibliothek
insuffizient. \url{https://www.hebis.de/de/1ueber_uns/statistik/cbs_statistik.php} 
Auch an der deutschen Bibliotheksstatistik nimmt die Bibliothek nicht teil. Beide bieten zudem nur Zahlenkolonnen und keine weiteren Visualisierungen an.}

