\chapter{Einführung}
%Das große Problem\\
Als Ende des Jahres 2019 der Ausbruch der Covid-19-Pandemie begann, entwickelte die 
Johns Hopkins University ein Dashboard als Antwort auf die anhaltende Unsicherheit im Bereich der öffentlichen Gesundheit. 
Dieses visualisiert seit je die gemeldeten Fälle weltweit. Es wurde entwickelt, um Forschern, Gesundheitsbehörden und der breiten Öffentlichkeit 
ein benutzerfreundliches Instrument an die Hand zu geben, mit dem sich der Ausbruch verfolgen lässt. 
Zu dessen Datenquellen gehören unter anderem die Informationen der Weltgesundheitsorganisation, staatliche und nationale
Gesundheitsämter. Die Daten wurden aggregiert und verdichtet. So visualisiert das Dashboard mit einer Landkarte und Punkten den Ausbruch. 
Dazu gibt es Zahlen der bestätigten COVID-19-Fälle, der Todesfälle und  der Genesungen für alle betroffenen Länder\cite[Vgl.][533]{dong_interactive_2020}.

Das faszinierende an dem Dashboard ist, das es gelungen ist, alle relevanten Zahlen auf einer Seite darzustellen
und somit der interessierten Öffentlichkeit schnell einen Überblick zu verschaffen.

\section{Problemstellung}
%Das kleine Problem\\
Ausgehend von ökonomischen, informationstechnologischen und marktpolitischen Einschnitten in den
vergangenen Jahrzehnten, sind Bibliotheken dazu veranlasst, ihr Budget hinsichtlich der Informationsbedarfe
ihrer Nutzer:innen behutsamer zu planen und sich in zunehmenden Maße gegenüber ihren Unterhaltsträgern zu rechtfertigen.
Die Relevanz von bibliothekarischen Statistiken ist in diesem Zusammenhang größer geworden.
Deswegen ist es wichtig, Daten aus den bibliothekarischen Bereichen zu aggregieren, zu erheben und statistisch
auszuwerten, um auf Basis der daraus erzielten Erkenntnisse handeln zu können. 
Die Transparenz von statistischen Daten sorgt für eine bessere Grundlage in den Verhandlungen mit den Stakeholdern
einer Bibliothek. Zudem wird durch sie der Einsatz des Bibliotheksbudgets zielgerichteter auf die Bedürfnisse der Nutzer:innen zugeschnitten.
Dazu ist es zweckmäßig, alle anfallenden Daten für die Budgetplanung und Mittelallokation einer Bibliothek zentral zu sammeln und mit geeigneten 
statistischen Methoden und Verfahren langfristig auszuwerten. Um den Wert dieser aus den Daten gewonnenen Information elegant den Stakeholdern zu kommunizieren und zu präsentieren,
können geeignete Verfahren der Datenvisualisierung zum Einsatz kommen. Die technische Realisierung kann durch gewöhnliche Tabellenkalkulationsprogramme umgesetzt werden.
Um den mitunter hohen Zeitaufwand einerseits zu minimieren und den Automatisierungsgrad hinsichtlich der Aggregation und Auswertung der bibliothekarischen Daten 
andererseits zu erhöhen, können aber auch andere technische Umsetzungen eingesetzt werden. In Bereichen der Wirtschaft kommen sogenannte Business-Intelligence-Systeme zum Einsatz,
die die Entscheidungsfindung auf Grundlage von Unternehmensdaten IT-basiert unterstützen.
%kann geschehen mit herkömmlichen Anwendungen wie Tabellenkalkulationsprogrammen oder mit anderen BI-Systemen...

%Was ist der Markt?\\
%--------------------
Es gibt bereits eine Vielzahl kommerzieller Lösungen für den Bibliotheksbereich, die auf Business-Intelligence-Systemen basieren.
Zu nennen wären \textit{AlmaAnalytics} für das Next-Generation-Library-System \textit{Alma} von \textit{ExLibris}\footnote{\url{https://www.exlibrisgroup.com/products/alma-library-services-platform/alma-analytics}
Stand: 26.05.2020}, \textit{BibControl} von \textit{OCLC}\footnote{\url{https://www.oclc.org/de/bibcontrol.html} Stand: 26.05.2020},
\textit{CollectionHq} von \textit{Baker \& Taylor}\footnote{\url{https://www.collectionhq.com/} Stand: 26.05.2020} oder \textit{Libinsight} von \textit{SpringShare}\footnote{\url{https://springshare.com/libinsight/} Stand: 26.05.2020}.
Darüber hinaus gibt es Business-Intelligence-Applikationen, die von Bibliotheken für Reporting, Datenanalyse und Datenvisualisierung adaptiert werden,
wie zum Beispiel \textit{Tableau} von der Firma \textit{Tableau Software},
\textit{Crystal Reports} von \textit{SAP} oder Microsoft BI.
Diese Applikationen sind entweder an bestimmte Bibliothekssysteme zurückgebunden, limitiert in ihren
Funktionen\cite{golas_statistische_2018} oder zu generisch.

%Wem hilft es?\\
%--------------
Für die Spezialbibliothek des Max-Planck-Institutes für empirische Ästhetik begründet sich die Notwendigkeit für eine solche Applikation durch das
Fehlen eines zentralen Nachweisortes für bibliothekarische
Statistiken in der Bibliothek. Überdies wird sowohl vom \textit{HeBis}-Verbund als auch von der \textit{mpdl} keine Tools in dieser Richtung angeboten.
Ebenso ist ungewiss, wann die Ablösung des schon betagten Bibliothekssystems hin zu
einem neuen Next-Generation-Library-System im \textit{HeBis-Verbund} stattfinden wird und ob
es ein Modul zur statistischen Datenerhebung liefern wird.
Da die Spezialbibliothek zudem verschiedene Recherche-Systeme den
Wissenschaftler:innen anbietet, wäre eine Engführung der statistischen
Datenerhebung auf eine Plattform begrüßenswert.
Des Weiteren ist das Erfordernis, bibliothekarische Geschäftsprozesse zu evaluieren und die
Servicedienstleistungen bezüglich der Ziele der Institution noch weiter zu
optimieren, von großer Relevanz. 
Die zu entstehende Applikation könnte hierbei helfen, systematisches Controlling einzuführen und das
Bibliotheksmanagement weiter zu professionalisieren.
%predictive analysis\\


%Warum jetzt?\\
%--------------

Mit dem Ende der Konsolidierungsphase der
Bibliothek, die im Zuge des \textit{Max-Planck-Institutes für empirische
Ästhetik} 2014 gegründet wurde, tritt sie ein in eine Phase, in der ab dem Jahr
2021 Budgetplanungen eine größere Rolle spielen werden.

%Ist das Problem lösbarerer geworden?\\
-------------------\\
Ein System jenseits von Excel zu entwickeln ist heute einfacher geworden, da es einerseits einen Markt für
solche Anwendungen, der im Schatten von DataScience wächst. So muss man kein ausgwachsener FullStack-Entwickler,
um eine Anwendung zu programmieren,
%Wodurch?\\
sondern man kann zurück greifen auf ausgereiftere und mächtige Frameworks
wenig aufwendig Daten statistisch auszuwerten, eher Problem, dass es so viele heterogene Daten aus verschiedenen Datenquellen
gibt die aufbereitet werden müssen. Dashboards gibt es zwar schon lange, sind aber jetzt auch einfacher geworden zu gestalten.
%Was ist der Trend?\\
Ad-hoc Realtime Data - Weg bewegen von schwierig aufsetzen DWH hinzu Data Lakes, die Daten auswerten, wenn sie benötigt werden


\section{Ziel der Arbeit}
%Ihr Ziel der Arbeit in zwei Sätzen.\\
Das Ziel der Arbeit ist die Schaffung eines Dashboards für Budgetplanung in Bibliotheken. 
In Anlehnung an \acrfull{BI}-Systeme soll ein System als proof-of-concept entstehen,
mit dem systematisch die relevanten Daten einer hybriden Spezialbibliothek aggregiert, statistisch
analysiert und mit geeigneten und modernen Datenvisualisierungen ausgegeben werden sollen.
Um künftigen Anforderungen gewachsen zu sein, soll sie
modulbasiert programmiert werden und dadurch leicht erweiterbar und eventuell von
anderen Bibliotheken nachnutzbar sein.

Warum genau dieses Problem?\\
Ist Ihr Beitrag völlig neu, oder nur ein Baustein?\\
Ist Ihr Problem schwer zu lösen oder „straight forward“?\\
eher schwieriger zu lösen, da generischer Ansatz gewählt werden soll -> abstrahieren ein bisschen von konkreter Implementierung
Eher Forschung oder eher Anwendung?\\
eher anwendung
Wenn Sie ein System bauen...\\
Welche Anfragen / Aufgaben wollen Sie beantworten / lösen können?\\
Entwicklung eines Workflows für die Aktualisierung der Daten...
Framework für template design für Daten, die eingespeist werden sollen
Welche Kernfunktionalität soll Ihr System haben?\\
Automatisierte Prozesse bei der Auswertung mit statistischen Verfahren, Import der Daten soll fast vollständig automatisiert sein
Interaktiviät -> Multidemensionalität, Eingrenzung der Zeiträume, Auswählen welche Auswertungen nach Medienart ... Domainwissen
Auswahl aus mehreren Visualisierungen
Bereitstellung der wichtigsten KPI's in einem PDF

Was ist ein typischer (Bedienungs-) Prozess für Ihr System?\\
1) montaliche Budget und Umsatzzahlen kommen als email, werden automatisch in csv umgewandelt -> werden an große csv / datenbank geschrieben
mit der veränderten Datenlage entstehen Veränderungen in den Zahlen -> in Visualisierungen
2) Abfrage des Systems nach Neuerwerbungen mit Systemstellen in der RVK, vierteljahrlich, halbjahrlich, jährlich -> Kontrolle wie und in welchen Systematikgruppen
der Bestand wächst -> RVK-Systematik-Tabelle
Wer nutzt Ihr System, Ihren Algorithmus?\\
Nutzen sollen das System Bibliotheksmitarbeiter:innen und Bibliotheksleitung, auch zum Einspeisen der daten -> Fallback, Fehler abfangen beim Import
Wodurch ist dieses Nutzungsverhalten gekennzeichnet?\\
wenig bis keine Kenntnisse -> soll leicht nutzbar sein ohne große Vorkenntnisse durch automatisierte Prozesse zur Gewinnung der Ergebnisse konzentriert werden.
nur Datei in Ordner schieben
Import/ Export  soll automatisch geschehen.

Anleihen von BI-Systemen in der Architektur, Anhaltspunkte 

Ausleihzahlen nach Jahr, Quartal, Monat
Bestand -> Bestandssegmente(Klass.Gruppen) -> Bestandssegment (einezelne Klassen) -> Einzeltitel => graphisch darstellen
=> Welche Bestandsgruppe am besten geht / schlechtesten

Budgewt



Mit diesen automatisch angefertigten statistischen Datenanalysen sollen zukünftige
Entscheidungen im Bibliotheksmanagement wie Erwerbungspolitik, Budgetplanung und
Mittelallokation hinsichtlich der weiteren Entwicklung der
Servicedienstleistungen evidenzbasiert und datengetrieben unterstützt werden.

Darüber hinaus soll die Applikation  eine Funktion beinhalten, ausgewählte
Resultate automatisiert als \textit{factsheet} zu exportieren, um diese
als Rechenschaftsbericht gegenüber Stakeholdern der Bibliothek präsentieren zu können.

\section{Verwandte Arbeiten}

% Welche Vorarbeiten gibt es schon?\\
% Wo und Wann sind die Vorarbeiten entstanden?\\
% Welche Ziele haben die Vorarbeiten verfolgt?\\
% Auf welche Schwierigkeiten sind sie gestossen?

Ein gutes Beispiel für ein datengetriebenes Unterstützungssystem findet sich in
der Literatur bei Spielberg, der sich mit dem Thema der Bestandspflege an der
\textit{Universitätsbibliothek Essen} befasst und eine Applikation (weiter-)entwickelt hat, die
die Fachreferent:innen bei der Aussonderung und Erwerbung von Medien
unterstützt.\cite{spielberg_eike_t_fachref-assistent_nodate}
Ebenso finden sich in der Fachliteratur Ansätze, die vorrangig anhand einzelner
Fragestellungen hinsichtlich der Bestandsentwicklung\cite{hughes_long-term_2016} oder anderer
bibliothekarischer Servicedienstleistungen\cite{kutlay_shiny_2020, knievel_use_2006,meyer_using_2018} verschiedene statistische Analysen
vollzogen und diese visualisiert haben.
Eine Ausnahme bildet die Entwicklung eines Dashboards an der \textit{New York
University Health Sciences Libraries}, das versucht, möglichst viele Metriken
aus bibliothekarischen Dienstleistungen aufzunehmen.\cite{morton-owens_trends_2012}
Fast alle Projekte sind an größeren
Universitätsbibliotheken mit ganz unterschiedlichen softwaretechnischen
Herangehensweisen\cite{finch_using_2016, wiegand_visualizing_2013} und Zielen\cite{phetteplace_effectively_2012} entstanden.

Dennoch fehlen in der gesichteten Literatur Teile, die sich mit der Budgetierung
befassen und Auskunft über Mittelallokation geben.

Zudem fehlt ein Beispiel in der Literatur, das holistisch alle relevanten Daten, die in den
Geschäftsgängen und Servicedienstleistungen insbesondere einer Spezialbibliothek entstehen,
aggregiert, auf diesen Daten automatisch statistische Analysen ausführt und diese mit modernen Visualisierungstechniken
interaktiv darstellt.

Anwendung kommen deskriptive Statistik, es geht vielmehr darum Daten zusammen zu tragen, als diese zu explorieren 
Abgrenzung deskriptive Statistik / explorative Datenanalyse

\section{Gliederung der Arbeit}
%Wann lesen wir was und warum?\\
Im \autoref{chap:two} werden die theoretischen Grundlagen für die folgenden Kapitel gelegt. Das Kapitel befasst
sich mit den Themen Bibliothek und Statistik, Datenvisualisierung und Business-Intelligence-Systemen. Dabei wird herausgestellt, wie wichtig Statistik
im bibliothekarischen Bereich sind, was Datenvisualisierungen sind und warum sie eingesetzt werden sollen und welche Anleihen Business-Intelligence-Systeme 
für das zu entstehende System liefern können. \autoref{chap:three}
wird die Bibliothek vorgestellt und darauf eingegangen welche bibliothekarischen Statistiken bereits erhoben wurden.
Nachdem die Ausgangssituation bestimmt wurde, wird mit der Anforderungsanalyse im \autoref{chap:four} die Konzeption einer Lösung vorgestellt.
Im \autoref{chap:five} wird die Umsetzung diskutiert. Bevor das System bewertet wird, wird das Design und die Implementierung vorgestellt.
Das Fazit wird im \autoref{chap:six} mit dem Stand der Umsetzung, den lessons learned und einem Ausblick auf Themen, die noch bearbeitet werden könnten, gezogen.
