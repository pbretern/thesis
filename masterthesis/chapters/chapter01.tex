\chapter{Einführung}
%Das große Problem\\
Als Antwort auf die anhaltende Unsicherheit im Bereich der öffentlichen Gesundheit, die die Ausbreitung der COVID-19-Pandemie hervorgerufen hat, 
entwickelte zu Beginn des Jahres 2020 ein Team um die Professorin Lauren Gardner der \textit{Johns Hopkins University} ein Dashboard. 
Dieses visualisiert die gemeldeten Fälle der COVID-19-Pandemie weltweit. Das Dashboard wurde entwickelt, um Forschenden, Gesundheitsbehörden und der breiten Öffentlichkeit 
ein benutzerfreundliches Instrument an die Hand zu geben, mit dem sich der Ausbruch leicht verfolgen lässt.
Visualisiert durch eine Weltkarte und unterschiedlich großen Punktmarkierungen zeigt das Dashboard die gegenwärtige Ausbreitung.
%Zu dessen Datenquellen gehören unter anderem die Informationen der Weltgesundheitsorganisation, staatliche und nationale
%Gesundheitsämter. Die Daten wurden aggregiert und verdichtet. 
Zusätzlich werden Zahlen der bestätigten COVID-19-Fälle, der Todesfälle und der Genesungen für alle betroffenen Länder angezeigt\cite[Vgl.][533]{dong_interactive_2020}.
In der Anfangszeit des Dashboards wurden die Daten noch manuell gesammelt und bearbeitet. Mittlerweile wurde ein halb-automatischer Prozess implementiert.
%Auf dem Github-Repositorium, das zu dem Dashboard eingerichtet wurde, liegen die gesammelten Daten in csv-Format vor.
Durch die übersichtliche Darstellung auf dem Dashboard können rasch Informationen über die Gefahrenlage von einer breiten Öffentlichkeit
rezipiert werden. Daraus kann ein Verständnis der derzeitigen Maßnahmen bei der Öffentlichkeit entstehen sowie die Informationen in eigene Handlungsoptionen
integriert werden.


\section{Problemstellung}
%Das kleine Problem\\
Ausgehend von ökonomischen, informationstechnologischen und marktpolitischen Veränderungen in den
vergangenen Jahrzehnten, sind Bibliotheken dazu veranlasst, ihr Budget hinsichtlich der Informationsbedarfe
ihrer Nutzer:innen behutsamer zu planen und sich in zunehmenden Maße gegenüber ihren Unterhaltsträgern zu rechtfertigen.
Die Relevanz von bibliothekarischen Statistiken ist in diesem Zusammenhang größer geworden.
Deswegen ist es wichtig, Daten aus den bibliothekarischen Bereichen zu aggregieren, statistisch zu erheben und 
auszuwerten, um auf Basis der daraus erzielten Erkenntnisse handeln zu können. 
Die Transparenz von statistischen Daten sorgt für eine bessere Grundlage in den Verhandlungen mit den Stakeholdern
einer Bibliothek. Zudem wird durch sie der Einsatz des Bibliotheksbudgets zielgerichteter auf die Bedürfnisse der Nutzer:innen zugeschnitten.
Dazu ist es zweckmäßig, alle anfallenden Daten für die Budgetplanung und Mittelallokation einer Bibliothek zentral zu sammeln und mit geeigneten 
statistischen Methoden und Verfahren langfristig auszuwerten. Um den Wert dieser aus den Daten gewonnenen Information elegant den Stakeholdern zu kommunizieren und zu präsentieren,
können geeignete Verfahren der Datenvisualisierung zum Einsatz kommen. Die technische Realisierung kann durch gewöhnliche Tabellenkalkulationsprogramme erfolgen.
Um den mitunter hohen Zeitaufwand und die Fehleranfälligkeit manueller Prozesse einerseits zu minimieren und den Automatisierungsgrad hinsichtlich der Aggregation und Auswertung der bibliothekarischen Daten 
andererseits zu erhöhen, können aber auch andere technische Umsetzungen eingesetzt werden. In Bereichen der Wirtschaft kommen \acrfull{BI}-Systeme zum Einsatz, 
die IT-basiert Entscheidungsfindungen unterstützen (siehe dazu \autoref{chap:two_three}).
%die die Entscheidungsfindung auf Grundlage von Unternehmensdaten IT-basiert unterstützen.
%kann geschehen mit herkömmlichen Anwendungen wie Tabellenkalkulationsprogrammen oder mit anderen BI-Systemen...

%Was ist der Markt?\\
%--------------------
Es gibt bereits eine Vielzahl kommerzieller Lösungen für den Bibliotheksbereich, die auf \textit{\acrshort{BI}-Systemen} basieren.
Zu nennen wären \textit{AlmaAnalytics} für das Next-Generation-Library-System \textit{Alma} von \textit{ExLibris} \cite{ex_libris_alma_2020},
\textit{BibControl} vom \acrfull{OCLC} \cite{oclc_bibcontrol_2020},
\textit{CollectionHq} von \textit{Baker \& Taylor}\cite{baker__taylor_select_2020} oder \textit{Libinsight} von \textit{SpringShare} \cite{springShare_libinsight_2020}.
Darüber hinaus gibt es \textit{\acrlong{BI}-Systeme}, die von Bibliotheken für Reporting, Datenanalyse und Datenvisualisierung adaptiert werden,
wie zum Beispiel \textit{Tableau} von der Firma \textit{Tableau Software} \cite{tableau_software_software_2020},
\textit{Crystal Reports} von \textit{SAP} \cite{sap_pixel-perfect_2020} oder \textit{Microsoft BI} von \textit{Microsoft} \cite{microsoft_datenvisualisierung_2020}.
Diese Applikationen sind entweder an bestimmte Bibliothekssysteme zurückgebunden, limitiert in ihren
Funktionen oder zu generisch.

%Wem hilft es?\\
%--------------
Für die Spezialbibliothek des \acrfull{MPI EA} begründet sich die Notwendigkeit für eine an \textit{\acrshort{BI}-Systemen} angelehnte Applikation durch das
Fehlen eines zentralen Nachweissystems für bibliothekarische
Statistiken in der Bibliothek. Überdies wird sowohl vom \acrfull{HeBis}-Verbund, deren Mitglied die Bibliothek ist, als auch 
von der \acrfull{mpdl} keine Systeme in dieser Richtung angeboten. Ebenso ist ungewiss, wann die Ablösung des schon betagten Bibliothekssystems hin zu 
einem neuen Next-Generation-Library-System in \textit{\acrshort{HeBis}} stattfinden wird und ob
es ein Modul zur statistischen Datenerhebung mitbringen wird. 
%Da die Spezialbibliothek zudem verschiedene Recherche-Systeme den Wissenschaftler:innen anbietet, wäre eine Engführung der statistischen Datenerhebung auf eine Plattform begrüßenswert.
Des Weiteren ist das Erfordernis, bibliothekarische Geschäftsprozesse zu evaluieren und die
Servicedienstleistungen bezüglich der Ziele der Institution noch weiter zu
optimieren, von großer Relevanz. 
Die zu entstehende Applikation könnte hierbei helfen, systematisches Controlling einzuführen und das
Bibliotheksmanagement weiter zu professionalisieren.
%predictive analysis\\
%Warum jetzt?\\
%-------------
Mit dem Ende der Konsolidierungsphase der
Bibliothek, die im Zuge des \textit{\acrshort{MPI EA}} 2013 gegründet wurde, tritt sie ein in eine Phase, in der ab dem Jahr
2021 Budgetplanungen eine größere Rolle spielen werden.


\section{Ziel der Arbeit}
%Ihr Ziel der Arbeit in zwei Sätzen.\\
Das Ziel der Arbeit ist die Schaffung eines Dashboards für die Budgetplanung in der Spezialbibliothek des \textit{\acrshort{MPI EA}}.
In Anlehnung an \textit{\acrshort{BI}-Systeme} soll ein ganzheitliches System als Proof-of-Concept entstehen,
mit dem systematisch die relevanten Daten der hybriden Spezialbibliothek aggregiert, statistisch
mit geeigneten und modernen Datenvisualisierungen analysiert und ausgewertet werden sollen.
%Mit diesen automatisch angefertigten statistischen Datenanalysen sollen zukünftige
%Entscheidungen im Bibliotheksmanagement wie Erwerbungspolitik, Budgetplanung und
%Mittelallokation hinsichtlich der weiteren Entwicklung der
%Servicedienstleistungen evidenzbasiert und datengetrieben unterstützt werden.
Darüber hinaus soll es möglich sein, aus dem System ausgewählte
Resultate automatisiert als Standardbericht zu exportieren, um diese
als \textit{factsheet} gegenüber Stakeholdern der Bibliothek präsentieren zu können.
Um künftigen Anforderungen gewachsen zu sein, soll das Dashboard
modulbasiert programmiert werden und dadurch leicht erweiterbar sowie eventuell von
anderen Bibliotheken nachnutzbar sein.


Es gibt einen wachsenden Markt für Systeme, die solche Anwendungen möglich machen. Dieser wächst im Schatten von Data-Science und Big Data. 
Der Markt verfügt über ausgereifte und mächtige Frameworks, die statistische Auswertungen mit wenig Programmieraufwand erlauben. 
Der höhere zeitliche Aufwand bei solchen statistischen Auswertungen liegt demgegenüber einerseits in der Aufbereitung der Daten, 
insbesondere dann, wenn die Daten aus heterogenen Datenquellen kommen. 
Das setzt dementsprechend eine genaue Analyse der Daten voraus. Andererseits kann die Zusammenführung der einzelnen statistischen Auswertungen in eine Applikation 
eine nicht leicht zu nehmende Hürde darstellen. Die Entwicklung von interaktiven Dashboards oder ähnlichen Web-Anwendungen ist ebenfalls leichter geworden, da
Kenntnisse in einer Programmiersprache durchaus ausreichend sind, um eine solche Applikation zu programmieren.
%Was ist der Trend?\\
%Warum genau dieses Problem?\\

%Ist Ihr Beitrag völlig neu, oder nur ein Baustein?\\
%eher ein Baustein -> Versuch Lösung, die generisch passt. Vorgaben wie die Daten ausszusehen haben
%Ist Ihr Problem schwer zu lösen oder „straight forward“?\\
%eher schwieriger zu lösen, da generischer Ansatz gewählt werden soll -> abstrahieren ein bisschen von konkreter Implementierung
%Anhand der zu analysierenden Daten -> Vorgaben wie Daten auszusehen haben -> damit es reibungslos klappt
%flaches System sein -> keine Problemee beim Aufsetzen
%Eher Forschung oder eher Anwendung?\\
%eher anwendung
%Wenn Sie ein System bauen...\\

%Welche Anfragen / Aufgaben wollen Sie beantworten / lösen können?\\
Folgende Aufgaben und Zwecke soll das System lösen können:
Die Daten sollen aus den heterogenen Datenquellen mit einem \acrfull{ETL}-Prozess bearbeitet und in
geeigneter Form gespeichert und bereitgestellt werden. Die Auswertungen erfolgen mittels statistischer Verfahren, 
die aus den Formulierungen der Fragen an die Daten abgeleitet werden. Dabei sollen Verfahren der deskriptiven und explorativen Statistik zum Einsatz kommen.
Die Auswertungen werden dann über das interaktive Dashboard visuell dargestellt.
Die visuelle Darstellung der Daten soll nach einstellbaren Parametern schrittweise verfeinert werden können.

Ein wesentlicher Zweck des Dashboards ist das Monitoring der laufenden Budgetkosten für die Medien. So sollen die Kosten für das laufende Jahr auch im Vergleich zu den
Vorjahren dargestellt werden können. Ein anderer wichtiger Baustein sind die Ausleihzahlen der Medien. Diese können
anhand der Dimensionen der Zeit oder der Aufstellungssystematik analysiert werden. Diese Analysen können sich 
nach Zeitraum beziehungsweise nach Bestandssegmenten bis hinunter auf Titelebene als kleinste Ebene ausdifferenzieren. 
Ebenso vorgesehen ist eine Analyse der Neuerwerbungen nach Zeit anhand der Dimension der Aufstellungssystematik.

Der Standardbericht greift auf vorberechnete Auswertungen und Darstellungen zurück und generiert anfrageorientiert einen Bericht, der die wichtigsten 
\acrfull{KPI} der Bibliothek enthält. Er wird im Format PDF verteilt und kann ohne bibliothekarisches Domänwissen gelesen werden.
%-> gleichmäßiges Wachstum der Bestandsgruppen, Löcher in den Bestandsgruppen

Das Dashboard ist für das Monitoring und die Analyse der Daten durch die Bibliothek gedacht. 
So kann die Bibliotheksleitung die Ausgaben für Medien nach Lieferanten überwachen und gegebenfalls steuernd eingreifen.
Anhand der Analyse der Daten kann die Bibliotheksleitung das Jahresbudget besser steuern.
Am Wachstum des Bestandes beziehungsweise der Bestandssegmente können die Bibliotheksmitarbeiter:innen ablesen, welche Bestandsgruppe
wie schnell wächst.
Der Standardbericht als aggregierte Form wichtigster \textit{\acrshort{KPI}} ist für die Kommunikation und Präsentation gegenüber der Institutsleitung gedacht.

%Das System sollen die Bibliotheksleitung zum Monitoring Nutzen hinsichtlich der Erwerbungsstrategien
%Nutzung der Ausleihstatistiken, Nutzungsstatistiken elektronischer Medien -> Evaluation der 
%Umsetzung strategischer Ziele -> 
%Monitoring einsatz für die Erwerbung beim Aufteilen des Geldes an Lieferanten




%Entwicklung eines Workflows für die Aktualisierung der Daten...
%Framework für template design für Daten, die eingespeist werden sollen

%ETL-Prozess automatisch startet
%Welche Kernfunktionalität soll Ihr System haben?\\
%Automatisierte Prozesse bei der Auswertung mit statistischen Verfahren, ETL-Prozess der Daten soll fast vollständig automatisiert sein
%Interaktiviät -> Multidemensionalität, Eingrenzung der Zeiträume, Auswählen welche Auswertungen nach Medienart ... Domainwissen
%Auswahl aus mehreren Visualisierungen
%Bereitstellung der wichtigsten KPI's in einem PDF

%Was ist ein typischer (Bedienungs-) Prozess für Ihr System?\\
%1) montaliche Budget und Umsatzzahlen kommen als email, werden automatisch in csv umgewandelt -> werden an große csv / datenbank geschrieben
%mit der veränderten Datenlage entstehen Veränderungen in den Zahlen -> in Visualisierungen
%2) Abfrage des Systems nach Neuerwerbungen mit Systemstellen in der RVK, vierteljahrlich, halbjahrlich, jährlich -> Kontrolle wie und in welchen Systematikgruppen
%der Bestand wächst -> RVK-Systematik-Tabelle
%Wer nutzt Ihr System, Ihren Algorithmus?\\
%Nutzen sollen das System Bibliotheksmitarbeiter:innen und Bibliotheksleitung, auch zum Einspeisen der daten -> Fallback, Fehler abfangen beim Import
%Wodurch ist dieses Nutzungsverhalten gekennzeichnet?\\
%wenig bis keine Kenntnisse -> soll leicht nutzbar sein ohne große Vorkenntnisse durch automatisierte Prozesse zur Gewinnung der Ergebnisse konzentriert werden.
%nur Datei in Ordner schieben
%Import/ Export  soll automatisch geschehen.

%Anleihen von BI-Systemen in der Architektur, Anhaltspunkte 

%Ausleihzahlen nach Jahr, Quartal, Monat
%Bestand -> Bestandssegmente(Klass.Gruppen) -> Bestandssegment (einezelne Klassen) -> Einzeltitel => graphisch darstellen
%=> Welche Bestandsgruppe am besten geht / schlechtesten

%Budgewt


%Danksagung
%wichtige Personen, aber dann, mit einer langen Kunstpause: die allerwichigste, teuerste, schönste und klügste (nicht zu vergessen: witzigste). Ca. 12 Zeilen. 


\section{Verwandte Arbeiten}
Es gab in den letzten Jahren Versuche an Universitätsbibliotheken Dashboards zu entwickeln, um daten-getrieben Entscheidungen zu unterstützen.
So wurde an den \textit{New York University Health Sciences Libraries} ein Dashboard entwickelt, das versucht, möglichst viele Metriken
aus bibliothekarischen Dienstleistungen aufzunehmen. Die Architektur des Dashboards besteht aus drei Hauptteilen. Die Daten werden mit 
Import-Skripten aus den verschiedenen Datenquellen bezogen, mit einem \textit{\acrshort{ETL}-Prozess} bearbeitet und in ein Data Warehouse geladen. Das Data Warehouse 
stellt eine einfache MySQL-Datenbank dar. Die Daten werden aus dieser mit einem Mix von PHP/Javascript-Skripten in einem Dashboard mit unterschiedlichen Diagrammen
dargestellt \cite{morton-owens_trends_2012}. Ebenfalls wurde an der \textit{James C. Kirkpatrick Library der University of Central Missouri} ein Dashboard 
zur Unterstützung Evidenz-basierter Entscheidungen der Bibliotheksleitung entwickelt. 
Vom Entwicklungs-Team wurde sich gegen kommerzielle Lösungen entschieden und ein Dash-board mit dem Ruby-on-Rails-Framework entwickelt. Es zeigt verschiedene Daten wie Study-Room- oder Computer-Usage-Data \cite{james_c_kirkpatrick_library_jckl_2020}.
\citeauthor{horne-popp_if_2018} geben detaillierten Einblick in die Designentscheidungen und Problematiken während der Entwicklung dieses Tools \cite[vgl.][194 ff.]{horne-popp_if_2018}. 

An der \textit{Universitätsbibliothek der Technischen Universität Berlin} wird \textit{Alma Analytics} benutzt, was eine Business-Intelligence-Lösung des Bibliothekssystemanbieters \textit{ExLibris} darstellt. 
\Citeauthor{golas_statistische_2018} beschreibt die Neuimplementierung der statistischen Abfragen nach Einführung des cloudbasierten Bibliotheksmanagementsystems \textit{Alma} an der Universitätsbibliothek.
%Dabei wurden die Möglichkeiten von Alma Analytics benutzt, was eine Business-Intelligence-Lösung des Bibliothekssystemanbieters darstellt. 
Neu implementiert wurden unter anderem Trends nach der \acrfull{RVK}, die RVK-Schlagwörter-Zuordnung, Statistiken über Ausleihen und Vormerkungen \cite{golas_statistische_2018}.

In eine ähnliche Richtung wie das zu entwickelnde Tool, geht die an der \textit{Technischen Hochschule Wildau} entwickelte Software \textit{BibloVis}.
\textit{BibloVis} basiert auf einem modular aufgebauten Client-Server-Modell und ermöglicht die Einbindung und Visualisierung von Daten wie
der Katalognutzung, Raumauslastungen und  anderen bibliothekarischen Service-Angeboten. Die Grundlage bildet die Auswertungen von \textit{\acrshort{CSV}}-
und \textit{\acrshort{XML}}-Dateien \cite{block_biblovis_2015}.

%bietet eine komfortable Alternative zu verstreut abgelegten Excel-Statistiken und zugangsgesteuerten Online-Portalen, mit welcher kurzfristig ein Gesamtüberblick erzeugt, 
%längerfristig Analysen betrieben werden können, und somit die Entscheidungsfindung unterstützt wird
%Das auf den Technologien HTML5 und Javascript basierende modular aufgebaute Server-Client Modell ermöglicht 
%z. B. das Einbinden und Visualisieren von Daten aus der aktuellen Katalog-Nutzung, Downloads von online-Angeboten, Raumauslastung, Homepage-Aufrufen, Medienrückstellquote, Besucherzählung, 
%Verbuchungsprozessen, Fernleihprozessen etc. Mit BibloVis können unterschiedliche Datenquellen aufbereitet werden, z.B. gibt es auch ein Importmodul 
%für Daten aus der Deutschen Bibliotheksstatistik (DBS), Nutzungsstatistiken unterschiedlicher Verlagsangebote (z.B. Elsevier, Wiso- und Springerbooks). 
%Grundlage sind Auswertungen von cvs- oder xml-Dateien. 


Der Einsatz von \textit{Tableau} an den \textit{Ohio State University Libraries} wird von \Citeauthor{murphy_data_2013} beschrieben. 
Anhand von zwei Projekten werden die Einsatzmöglichkeiten insbesondere der Datenvisualisierungsoptionen 
dieser Software dargestellt. Unter anderem wurden die Transaktionsprotokolldateien auf Bibliotheksleitfäden untersucht, 
die von der Bibliothek zu den Universitätskursen herausgegeben wurden.  Diese wurden über einen Zeitraum von 2009-2012 ausgewertet und 
mit verschiedenen Datenvisualisierungen dargestellt \cite[vgl.][469 f.]{murphy_data_2013}.

Ein gutes Beispiel für ein datengetriebenes Unterstützungssystem findet sich bei \Citeauthor{spielberg_eike_t_fachref-assistent_nodate}, der sich mit dem Thema der Bestandspflege an der
\textit{Universitätsbibliothek Essen} befasst und eine Applikation (weiter-)entwickelt hat, die
die Fachreferent:innen bei der Aussonderung und Erwerbung von Medien
unterstützt \cite{spielberg_eike_t_fachref-assistent_nodate}.


Ebenso finden sich in der Fachliteratur Ansätze, die vorrangig anhand einzelner
Fragestellungen hinsichtlich der Bestandsentwicklung\cite{hughes_long-term_2016} oder anderer
bibliothekarischer Servicedienstleistungen\cite{kutlay_shiny_2020, knievel_use_2006,meyer_using_2018} verschiedene statistische Analysen
vollzogen und diese visualisiert haben.


Fast alle Projekte sind an größeren
Bibliotheken mit ganz unterschiedlichen softwaretechnischen
Herangehensweisen\cite{finch_using_2016, wiegand_visualizing_2013} und Zielen\cite{phetteplace_effectively_2012} entstanden.
%Wegen fehlender Erfahrung mit Programmiersprachen, experimentieren \Citeauthor{finch_using_2016}

Zudem fehlt ein Beispiel in der Literatur, das holistisch alle relevanten Daten, die in den
Geschäftsgängen und Servicedienstleistungen insbesondere einer Spezialbibliothek entstehen,
aggregiert, auf diesen Daten automatisch statistische Analysen ausführt und diese mit modernen Visualisierungstechniken
interaktiv darstellt.

%Dennoch fehlen in der gesichteten Literatur Teile, die sich mit der Budgetierung
%befassen und Auskunft über Mittelallokation geben.

%Zudem fehlt ein Beispiel in der Literatur, das holistisch alle relevanten Daten, die in den
%Geschäftsgängen und Servicedienstleistungen insbesondere einer Spezialbibliothek entstehen,
%aggregiert, auf diesen Daten automatisch statistische Analysen ausführt und diese mit modernen Visualisierungstechniken
%interaktiv darstellt.

%Anwendung kommen deskriptive Statistik, es geht vielmehr darum Daten zusammen zu tragen, als diese zu explorieren 
%Abgrenzung deskriptive Statistik / explorative Datenanalyse

\section{Gliederung der Arbeit}
%Wann lesen wir was und warum?\\
Im \autoref{chap:two} werden die theoretischen Grundlagen für die folgenden Kapitel gelegt. 
%Das Kapitel befasst sich mit den Themen Bibliothek und Statistik, Datenvisualisierung und Business-Intelligence-Systemen. 
Das Kapitel beschreibt den Rahmen, in dem die Entwicklung eines Dashboards eingebettet ist. Dabei wird herausgestellt, wie wichtig Statistik
im bibliothekarischen Bereich ist, was Datenvisualisierungen sind und warum sie eingesetzt werden sollen und welche Anleihen \textit{\acrshort{BI}-Systeme} 
für das zu entstehende System liefern können. Im \autoref{chap:three}
wird die Bibliothek vorgestellt und darauf eingegangen, welche bibliothekarischen Statistiken bereits erhoben wurden.
Nachdem die Ausgangssituation bestimmt wurde, wird mit der Anforderungsanalyse  im \autoref{chap:four} das generierte Wissen von \autoref{chap:two}
aufgegriffen und die Konzeption einer Lösung vorgestellt.
Im \autoref{chap:five} wird die Umsetzung diskutiert. Bevor das System bewertet wird, wird das Design und die Implementierung vorgestellt.
Das Fazit wird im \autoref{chap:six} mit dem Stand der Umsetzung, den \textit{lessons learned} und einem Ausblick auf Themen, die noch bearbeitet werden könnten, gezogen.
