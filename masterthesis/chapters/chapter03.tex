\chapter{Ausgangssituation}

Im folgenden Kapitel wird die wissenschaftliche Spezialbibliothek des \textit{Max-Planck-Institutes für empirische Ästhetik} porträtiert. 
Anschließend werden die bibliothekarischen Geschäftsgänge der
Bibliothek skizziert und der Frage nachgegangen, welche statistischen Daten aggregiert und ausgewertet wurden. 

\section{Bibliothek, Aufgabe, Kennzahlen}

%Wann wurde die Bibliothek gegründet?\\

Die Spezialbibliothek wurde im Zuge der Gründung des \textit{Max-Planck-Institutes für empirische Ästhetik} in Frankfurt im Jahr 2013 gegründet.
Die Aufgabe des Institutes ist die interdisziplinäre Erforschung empirischer Fragestellungen der Ästhetik. Das Institut besteht aus drei Abteilungen 
\textit{Sprache und Literatur}, \textit{Musik} und \textit{Neurowissenschaften} sowie einigen Forschungsgruppen. Ungefähr 150 Mitarbeiter:innen arbeiten 
in dem Institut. 
% Warum wurde die Bibliothek gegründet?\\
%Welche Aufgaben hat die Bibliothek?\\

Die Bibliothek ist eine Serviceinrichtung des Institutes und dient mit ihren Informationsdienstleistungen der Forschung.
%Welche Servicedienstleistungen bietet die Bibliothek an?\\
Zentral ist dabei die Informationsversorgung der Forschenden.
Die benötigten Informationen sind Bücher, Zeitschriften, Zeitschriftenartikel in gedruckter als auch elektronischer Form
Seit der Institutsgründung wird neben des nutzerorientierten Bestandsaufbaus ebenfalls eine bestandsorientierte Erwerbung betrieben.
Das Erwerbungsprofil der Bibliothek leitet sich vom Forschungsauftrag des Institutes ab und umfasst dementsprechend die Erwerbung von Informationsresourcen, die sich den theoretischen und empirischen Fragestellungen der Ästhetik widmen.
%Wie groß ist der Bibliotheksbestand? 

Der Bibliotheksbestand ist hybrid. Er besteht sowohl aus gedruckten als auch Online-Medien sowie audiovuisuellen Materialien.
An Bestand umfasst die Bibliothek zirka 11.000 Bücher, 30 laufende Zeitschriften, knapp über 200 audiovisuelle Medien sowie die Lizenzierung von Online-Datenbanken
und Online-Zeitschriften.
%Wie sieht der größere Organisationsrahmen der Bibliothek aus?\\
Um alle Informatiosbedarfe der Forscher:innen zu befriedigen, wird die Bibliothek in ihren Aufgaben von der
\textit{Max Planck Digital Library ({mpdl})} unterstützt. Deren Portfolio umfasst vorrangig die zentrale Lizenzierung
von relevanten elektronischen Informationsressourcen, die Bereitstellung
von Softwarelösungen, das Betreiben eines Publikationsrepositoriums und
das Vorantreiben von Open-Access. 

Darüber hinaus ist die Spezialbibliothek Teil des \textit{hessischen Bibliotheksverbundes (HeBis)}. Die Geschäftsprozesse
der Katalogisierung und der Erwerbung finden im Zentralsystem \acrlong{CBS} und im im Lokalsystem \textit{LBS4} von
\textit{OCLC} statt. \textit{LBS4} wird gehostet und betreut vom Lokalsystem-Team Frankfurt. Als Service-Leistung werden der Bibliothek besondere Funktionalitäten
für das \textit{CBS} bereitgestellt.



\section{Datengrundlage aus den bibliothekarischen Geschäftsgängen}
Das Bibliotheks-Team ist verantwortlich für den Ablauf und Organisation der bibliothekarischen Informationsdienstleistungen, die der Informationsversorgung dienen.
Eine Übersicht der Informationsdienstleistungen zeigt \autoref{tab:Informationsdienstleistungen}.
Die zentralen Informationsdienstleistungen der Spezialbibliothek bestehen aus dem sogenannten Buchservice und der Medienerwebung und -erschließung. Der Buchservice ist zuständig für die Organisation der Fern- und Ausleihe von Informationsressourcen aus anderen Bibliotheken, die nicht in das Erwerbungsprofil der Spezialbibliothek fallen. Ferner ist der Buchservice für die Informationsbeschaffung sowohl über Dokumentenlieferdienste als auch für die Acquise von einzelnen Zeitschriftenaufsätzen zuständig. 
Die Bibliothek betreut das Publikationsrepositorium \textit{Pure} des Institutes und bietet darüber hinaus Beratungsdienstleistungen zum Urheberrecht und zum Publishing an. 

\begin{table}[h]
    \centering
    \begin{adjustbox}{max width=\textwidth}
    \begin{tabular}{ll}
       \toprule
       \textbf{Bezeichnung}& \textbf{Beschreibung}\\
       \midrule
        Benutzung   &Ausleihe, Lesesaalnutzung, Aufstellungssystematik\\
        Beratung    &Allgemeine Beratungstätigkeit zur Literaturrecherche und Nutzung elektronischer Ressourcen, spezielle Beratungstätigkeit zu   Urheberrecht und Publishing\\
        Buchservice &Organisation der Fern- und Ausleihe von Informationsressourcen aus anderen Bibliotheken, Dokumentenlieferdienste\\
        DV Management   &Pure, MedienDatenbank\\
        Erwerbung                           &Integrierte Geschäftsgang Medienerwerung und Medienerschließung (Buch, Zeitschriften, Datenbanken, audiovisuelle Medien, Noten)\\
        Erwerbung besonderer Materialien    &Erwerbung von Stimuli wie Musikstücke, Bilder, Lizenzen zu Texten, einzelne Zeitschriftenartikel\\
         
       \bottomrule
    \end{tabular}
    \end{adjustbox}
    \caption
    \end{table}

Kategorie           Name                                Beschreibung
Erwerbung           Bestandsaufbau- und management
Erschließung        Formalerschließung
Benutzung           Bestandsvermittlung


Bestandsvermittlung Medienbereitstellung uns -ausleihe vor Ort über den Selbstverbucher
Buchservice -> Organisation der Ausleihe und Fernleihe von Informationsressourcen aus anderen Bibliotheken, die nicht in das Sammelprofil der Bibliothek passen. 
    Beschaffung von Dokumenten über Lieferservice wie subito
    oder auch Kauf von Zeitschriftenartikeln
Bereitstellung von audiovisuellen Stimuli über eine eigene Datenbank (Musikstücke)
Auskunfts- und Informationsdienst



Welche bibliothekarischen GG gibt es?\\
Welche statistischen Daten sind schon vorhanden?\\
In welchem Format liegen die statistischen Daten vor?\\
Welche statistischen Auswertungen gibt es?\\
Welche statistischen Auswertungen soll es noch geben?\\
Welche grafischen Auswertungen gibt es?\\
Welche grafischen Auswertungen soll es noch geben?

\clearpage

\begin{table}[h]
    \centering
    \begin{adjustbox}{max width=\textwidth}
    \begin{tabular}{llll}
       \toprule
       \textbf{Geschäftsgang}& \textbf{erhobene Daten} & \textbf{Format} & \textbf{Quelle}\\
       \midrule     
            Buchservice              & Fernleihe                            & Excel  & eigenständig \\ 
            Buchservice              & Scan                                 & Excel  & eigenständig \\ 
            Buchservice              & Subito                               & Excel  & eigenständig \\ 
            Buchservice              & Sonstiges                            & Excel  & eigenständig \\ 
            Buchservice              & elektronisch                         & Excel  & eigenständig \\ 
            Buchservice              & Ausleihen pro Abteilung              & Excel  & eigenständig \\ 
            Bibliotheksbenutzung     & Benutzer:Innenanzahl                 & Excel  & eigenständig \\ 
            Bibliotheksbenutzung     & nachgefragte Medien                  & email  & LBS          \\ 
            Ausleihe                 & Ausleihstatistik                     & Excel  & LBS          \\ 
            Erwerbung                & monatliche Ausgaben nach Lieferanten & email  & LBS          \\ 
            Erwerbung                & Budgetübersicht nach Kostenstellen   & email  & LBS          \\ 
            Bucheingang              & monatliche Neuerwerbungslisten       & Word   & LBS          \\ 
            Elektronische Ressourcen & Counter-Statistiken                  & tsv    & mpdl         \\ 
            Bestand                  & eigene                               & csv    & LBS          \\
        \bottomrule
    \end{tabular}
    \end{adjustbox}
    \caption
    \end{table}



Die Bibliothek des \textit{Max-Planck-Institutes für empirische Ästhetik}
ist Teil des \textit{hessischen Bibliotheksverbundes (HeBis)}. Die Geschäftsprozesse
der Katalogisierung und der Erwerbung finden im Zentralsystem \textit{CBS} und im im Lokalsystem \textit{LBS4} von
\textit{OCLC} statt. \textit{LBS4} wird gehostet und betreut vom Lokalsystem-Team Frankfurt. Als Service-Leistung werden der Bibliothek besondere Funktionalitäten
für das \textit{CBS} bereitgestellt. Außerdem erhält die Bibliothek unter anderem Ausleih-, Budget- und
Umsatzübersichten als Text per E-Mail zugesandt.


Neben der Verankerung in der deutschen Bibliotheksverbundlandschaft
wird die Bibliothek in ihren Aufgaben von der
\textit{Max Planck Digital Library (mpdl)}
unterstützt. Deren Portfolio umfasst vorrangig die zentrale Lizenzierung
von relevanten elektronischen Informationsressourcen, die Bereitstellung
von Softwarelösungen, das Betreiben eines Publikationsrepositoriums und
das Vorantreiben von Open-Access. Zudem stellt sie den Bibliotheken der einzelnen Max-Planck-Institute
\textit{COUNTER}-Statistiken zur Verfügung, die von den Verlagen geliefert werden.


Außer diesen bereitgestellten Daten erhebt die Bibliothek Daten unter anderem über
die Frequentation des Lesesaals, die Nutzung des nehmenden Fernleihservices, des
Dokumentenlieferdienstes \textit{subito} und des Bestandswachstums vor Ort.
Nach den unterschiedlichen Verantwortlichkeiten aufgeteilt, werden diese Daten an verschiedenen virtuellen Orten erhoben.
Eine systematische Auswertung der Daten findet nur unzureichend statt.
Daher regt sich der Wunsch seitens der Bibliotheksleitung und der Mitarbeiter:innen nach einem gemeinsamen Tool,
mit dem übersichtlich und klar alle notwendigen nutzungs- und sammlungsbezogenen Statistiken einer
Spezialbibliothek erfasst und dargestellt werden können.\footnote{Zwar führt \textit{HeBis} eine Bestandsstatistik, diese ist aber insbesondere für die
Evaluation und Optimierung von Geschäftsprozessen einer Spezialbibliothek
insuffizient. \url{https://www.hebis.de/de/1ueber_uns/statistik/cbs_statistik.php} 
Auch an der deutschen Bibliotheksstatistik nimmt die Bibliothek nicht teil. Beide bieten zudem nur Zahlenkolonnen und keine weiteren Visualisierungen an.}
