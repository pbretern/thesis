\chapter{Ausgangssituation}

Im folgenden Kapitel wird die wissenschaftliche Spezioalbibliothek porträtiert. Anschließend werden die serviceorientierten Geschäftsgänge der
Bibliothek skizziert und der Frage nachgegangen, welche statistischen Daten aggregiert und ausgewertet wurden. 

\section{Bibliothek, Aufgabe, Kennzahlen}

%Wann wurde die Bibliothek gegründet?\\

Die Spezialbibliothek wurde im Zuge der Gründung des \textit{Max-Planck-Institutes für empirische Ästhetik} in Frankfurt im Jahr 2013 gegründet.
Die Aufgabe des Institutes ist die interdisziplinäre Erforschung empirischer Fragestellungen der Ästhetik. Das Institut besteht aus drei Abteilungen 
\textit{Sprache und Literatur}, \textit{Musik} und \textit{Neurowissenschaften} sowie einigen Forschungsgruppen. Ungefähr 150 Mitarbeiter:innen arbeiten 
in dem Institut. 
% Warum wurde die Bibliothek gegründet?\\
%Welche Aufgaben hat die Bibliothek?\\

Die Bibliothek ist eine Serviceinrichtung des Institutes und dient mit ihren Dienstleistungen der Forschung.
%Welche Servicedienstleistungen bietet die Bibliothek an?\\
Die Dienstleistungen sind zentriert auf die Informationsbeschaffung für die Forschenden. 
Die benötigten Informationen bestehen aus Literatur wie Büchen, Zeitschriften, Zeitschriftenartikel und Stimuli.
Daneben wird seit der Gründung sukzessive ein Bestandsaufbau zu theoretischen und empirischen Fragestellungen der Ästhetik betrieben.
%Wie groß ist der Bibliotheksbestand? 

Der Bestand ist hybrid. Er besteht sowohl aus gedruckten als auch Online-Medien sowie audiovuisuellen Materialien.
An Bestand umfasst die Bibliothek zirka 11.000 Bücher, 30 laufende Zeitschriften, knapp über 200 audiovisuelle Medien sowie die Lizenzierung von Online-Datenbanken
und Online-Zeitschriften.
%Wie sieht der größere Organisationsrahmen der Bibliothek aus?\\
Um alle Informatiosbedarfe der Forscher:innen zu befriedigen, wird die Bibliothek in ihren Aufgaben von der
\textit{Max Planck Digital Library ({mpdl})} unterstützt. Deren Portfolio umfasst vorrangig die zentrale Lizenzierung
von relevanten elektronischen Informationsressourcen, die Bereitstellung
von Softwarelösungen, das Betreiben eines Publikationsrepositoriums und
das Vorantreiben von Open-Access. 

Darüberhinaus ist die Spezialbibliothek Teil des \textit{hessischen Bibliotheksverbundes (HeBis)}. Die Geschäftsprozesse
der Katalogisierung und der Erwerbung finden im Zentralsystem \acrlong{CBS} und im im Lokalsystem \textit{LBS4} von
\textit{OCLC} statt. \textit{LBS4} wird gehostet und betreut vom Lokalsystem-Team Frankfurt. Als Service-Leistung werden der Bibliothek besondere Funktionalitäten
für das \textit{CBS} bereitgestellt.

Wie groß das Team, wie strukturiert?

\clearpage


Die Bibliothek des \textit{Max-Planck-Institutes für empirische Ästhetik}
ist Teil des \textit{hessischen Bibliotheksverbundes (HeBis)}. Die Geschäftsprozesse
der Katalogisierung und der Erwerbung finden im Zentralsystem \textit{CBS} und im im Lokalsystem \textit{LBS4} von
\textit{OCLC} statt. \textit{LBS4} wird gehostet und betreut vom Lokalsystem-Team Frankfurt. Als Service-Leistung werden der Bibliothek besondere Funktionalitäten
für das \textit{CBS} bereitgestellt. Außerdem erhält die Bibliothek unter anderem Ausleih-, Budget- und
Umsatzübersichten als Text per E-Mail zugesandt.


Neben der Verankerung in der deutschen Bibliotheksverbundlandschaft
wird die Bibliothek in ihren Aufgaben von der
\textit{Max Planck Digital Library (mpdl)}
unterstützt. Deren Portfolio umfasst vorrangig die zentrale Lizenzierung
von relevanten elektronischen Informationsressourcen, die Bereitstellung
von Softwarelösungen, das Betreiben eines Publikationsrepositoriums und
das Vorantreiben von Open-Access. Zudem stellt sie den Bibliotheken der einzelnen Max-Planck-Institute
\textit{COUNTER}-Statistiken zur Verfügung, die von den Verlagen geliefert werden.


Außer diesen bereitgestellten Daten erhebt die Bibliothek Daten unter anderem über
die Frequentation des Lesesaals, die Nutzung des nehmenden Fernleihservices, des
Dokumentenlieferdienstes \textit{subito} und des Bestandswachstums vor Ort.
Nach den unterschiedlichen Verantwortlichkeiten aufgeteilt, werden diese Daten an verschiedenen virtuellen Orten erhoben.
Eine systematische Auswertung der Daten findet nur unzureichend statt.
Daher regt sich der Wunsch seitens der Bibliotheksleitung und der Mitarbeiter:innen nach einem gemeinsamen Tool,
mit dem übersichtlich und klar alle notwendigen nutzungs- und sammlungsbezogenen Statistiken einer
Spezialbibliothek erfasst und dargestellt werden können.\footnote{Zwar führt \textit{HeBis} eine Bestandsstatistik, diese ist aber insbesondere für die
Evaluation und Optimierung von Geschäftsprozessen einer Spezialbibliothek
insuffizient. \url{https://www.hebis.de/de/1ueber_uns/statistik/cbs_statistik.php} 
Auch an der deutschen Bibliotheksstatistik nimmt die Bibliothek nicht teil. Beide bieten zudem nur Zahlenkolonnen und keine weiteren Visualisierungen an.}


\section{Datengrundlage aus den bibliothekarischen Geschäftsgängen}


Welche bibliothekarischen GG gibt es?\\
Welche statistischen Daten sind schon vorhanden?\\
In welchem Format liegen die statistischen Daten vor?\\
Welche statistischen Auswertungen gibt es?\\
Welche statistischen Auswertungen soll es noch geben?\\
Welche grafischen Auswertungen gibt es?\\
Welche grafischen Auswertungen soll es noch geben?

\clearpage

\begin{table}[h]
    \centering
    \begin{adjustbox}{max width=\textwidth}
    \begin{tabular}{llll}
       \toprule
       \textbf{Geschäftsgang}& \textbf{erhobene Daten} & \textbf{Format} & \textbf{Quelle}\\
       \midrule     
            Buchservice              & Fernleihe                            & Excel  & eigenständig \\ 
            Buchservice              & Scan                                 & Excel  & eigenständig \\ 
            Buchservice              & Subito                               & Excel  & eigenständig \\ 
            Buchservice              & Sonstiges                            & Excel  & eigenständig \\ 
            Buchservice              & elektronisch                         & Excel  & eigenständig \\ 
            Buchservicw              & Ausleihen pro Abteilung              & Excel  & eigenständig \\ 
            Bibliotheksbenutzung     & Benutzer:Innenanzahl                 & Excel  & eigenständig \\ 
            Bibliotheksbenutzung     & nachgefragte Medien                  & email  & LBS          \\ 
            Ausleihe                 & Ausleihstatistik                     & Excel  & LBS          \\ 
            Erwerbung                & monatliche Ausgaben nach Lieferanten & email  & LBS          \\ 
            Erwerbung                & Budgetübersicht nach Kostenstellen   & email  & LBS          \\ 
            Bucheingang              & monatliche Neuerwerbungslisten       & Word   & LBS          \\ 
            Elektronische Ressourcen & Counter-Statistiken                  & tsv    & mpdl         \\ 
            Bestand                  & eigene                               & csv    & LBS          \\
        \bottomrule
    \end{tabular}
    \end{adjustbox}
    \caption
    \end{table}


