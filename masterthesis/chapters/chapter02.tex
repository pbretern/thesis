\chapter{Theoretische Grundlagen}
\label{chap:two}
In diesem Kapitel wird der theoretische Rahmen für die weiteren Kapitel gelegt. Im
ersten Abschnitt werden die Grundlagen der Budgetplanung und Mittelallokation im Zusammenhang mit bibliothekarischen Statistiken erläutert. 
Der darauf folgende Abschnitt handelt von Datenvisualisierungen und deren Einsatz
für Datenrepräsentationen und Datenpräsentation. Abschließend wird das Modell der Business-Intelligence-Software als Schmelzpunkt der 
beiden vorangegangen Kapitel eingeführt.

\section{Bibliothek und Statistik}
\label{chap:two_one}
% Bibliotheksrahmen - Etatplanung - Etatbedarfe, Zielsetzung der Bibliothek
Die Etatplanungen von Bibliotheken richten sich nach deren Informations- und Versorgungsauftrag. 
Seit Beginn der 1990er Jahre müssen sich Bibliotheken mit den Auswirkungen einer veränderten Medienlandschaft auseinandersetzen.
Sie kämpfen mit dem größer werdenden Informationsangebot, den steigenden Preisen auf dem Publikationsmarkt, 
den zunehmenden Kommerzialisierungstendenzen in der Verlagslandschaft und den neuen Medientypen. 
Zu nennen wären hier konkret: die Explosion der Zeitschriftenpreise im Bereich der \acrfull{STM}, die Konzentration auf wenige Verlage, 
und dem Aufkommen von Ebooks. Demgegenüber steigen Bibliotheksetats nur mäßig. 
Somit geht ein Kaufkraftverlust einher \cite[Vgl.][164 ff.]{moravetz-kuhlmann_monika_erwerbungspolitik_2015}.
Diese Entwicklung betrifft nicht nur Universitätsbibliotheken, sondern auch Spezialbibliotheken von Forschungseinrichtungen.
Bibliotheken haben Instrumente entwickelt, um den Informationsauftrag trotz dieser Widrigkeiten zu erfüllen.
So entstehen set Mittde der 1990 er Jahre von Bund und Ländern geförderte Konsortien, um den Kostendruck auf Bibliotheken im Bereich der elektronischen
Fachinformationen zu mildern. Neue Geschäftsmodelle werden zur Abfederung der Kosten entwickelt, um Preisnachlässe bei den Verlagen zu erzielen
\cite[Vgl.][169 ff.]{moravetz-kuhlmann_monika_erwerbungspolitik_2015}. Das Projekt \textit{Deal} -- ein Projekt der Hochschulrektorenkonferenz (HRK) in Zusammenarbeit mit den
wissenschaftlichen Einrichtungen in Deutschland -- konnte so in den vergangenen Jahren Verträge mit den Verlagen \textit{Springer} und \textit{Wiley} erfolgreich abschließen \cite[Vgl.][]{projekt_deal_projekt_2020}.

Um den Veränderungen des Publikationsmarktes lokal in der Bibliothek zu begegnen, wird es immer wichtiger, das Bibliotheksbudget und die Mittelallokation kosteneffizient zu planen. 
Dies geschieht bisher in größeren Bibliotheken durch Etatbedarfs- und Etatverteilungsmodelle \cite[Vgl.][172 ff.]{moravetz-kuhlmann_monika_erwerbungspolitik_2015}.
%Ziel dieser Modelle ist die effiziente Mittelallokation 
% transparente und gerechte Verteilung knapper Ressourcen% innerhalb der Bibliothek. 
% Mittelallokation bezeichnet die Verteilung knapper Ressourcen. 
Diese Modelle basieren auf der statistischen Erhebung von bibliothekarischen Kennzahlen.

%Was ist Statistik\\
%hat schon immer große Rolle in Bibliotheken gespielt\\
%BIX, Deutsche Bibliotheksstatistik (seit wann)\\
Bibliotheksstatistik reflektiert das Gestern, Heute und Morgen, indem 
sie die bibliothekarischen Servicedienstleistungen evaluiert und den zukünftigen Zielen und Aufgaben anpasst \cites[Vgl.][2 f.]{jilovsky_cathie_library_2004}[Vgl.][462]{laitinen_markku_library_2013}.
Im deutschen Bibliothekswesen gibt es die umfangreiche \acrfull{DBS}. 
Träger der \textit{\acrshort{DBS}} sind das \acrfull{hbz NRW},  das \acrfull{KBN}, die \acrfull{KMK} sowie den teilnehmenden Bibliotheken.
Aufgabe der \textit{\acrshort{DBS}} ist die jährliche statistische Datenerhebung von Bibliothekskennzahlen. 
Seit 1999 werden die Daten nur noch online erfasst, ausgewertet und präsentiert \cite[Vgl.][2]{schmidt_deutsche_2008}.
Neben anderen Servicedienstleistungen bietet die \textit{\acrshort{DBS}} Gesamtauswertungen an.
%Dennoch ist die \textit{\acrshort{DBS}} vielmehr eine Datengrundlage für die Auswertung der Daten als eine Auswertung solcher.
Daneben gab es den \acrfull{BIX}, der ursprünglich für die Leistungsmessung in Öffentlichen Bibliotheken konzipiert wurde. 
2002 wurde er erweitert auf das Wissenschaftliche Bibliothekssystem. Der \textit{\acrshort{BIX}} wurde 2015 aufgrund von Finanzierungsproblemen eingestellt. 

%Erhebung von qualitativen und quantitativen Daten Bsp.:\\
Bibliothekarische Kennzahlen werden durch quantitative und qualitative Evaluationsverfahren erhoben. Diese Verfahren
sind auf den Bestand der Bibliothek zentriert. 
Bestand ist nach Johannsen und Mittermaier
\textquote{... die Gesamtheit aller Medien, die eine Bibliothek ihren Nutzern anbietet, sei es, dass sie diese 
„physisch“ besitzt, sei es, dass sie entsprechende Nutzungsrechte erworben hat.} \cite[252]{johannsen_jochen_bestands-_2015}.
Als Typen der Bestandsevaluation sind sammlungs-, nutzungsbezogene und nutzer:innenbezogene Evaluationen zu nennen.\cite[Vgl.][302]{johnson_peggy_fundamentals_2014}
Basieren die sammlungs- und nutzungsbezogene Evaluation auf quantitativen Daten, greift die nutzer:innenbezogene Evaluation zumeist auf qualitative Daten zurück. 
\cite[Vgl.][461 ff.]{blake_data_2004}.

Die sammlungsbezogene Evaluation betrifft die Größe des Bestandes und das Wachstum über die Jahre. Die Bestimmung der Bestandsstärke- und tiefe, 
der Ausgewogenheit in den Bestandssegmenten sind Ziele der sammlungsbezogenen Evaluation. 
Ebenfalls lässt sich die Frage nach der aktuellsten Literatur im Bestand oder in einem Segment durch die sammlungsbezogene Evaluation klären.

Nutzungsbezogene Evaluation umfasst die Lesesaalnutzung, die Ausleihe vor-Ort, die Nutzung des Fernleihservices oder Dokumentenlieferdienste und die Online-Nutzung von elektronischen Ressourcen \cite[Vgl.][254 ff.]{johannsen_jochen_bestands-_2015}.
Die Frage nach den Zugriffsstatistiken auf elektronischen Ressourcen beansprucht in der nutzungsbezogenen Evaluation einen größer werdenden Raum.
Die internationale Organisation \textit{\acrfull{COUNTER}} gibt dazu die COUNTER-Statistiken heraus. Mitglieder der Organisation sind Verlage, Bibliotheken
und Zwischenhändler. Die COUNTER-Statistiken sind mittlerweile der Quasi-Standard für die Zugriffsstatistiken 
auf elektronischer Ressourcen geworden. Diese werden getrennt nach Art der Informationsressourcen in verschiedenen Reports herausgegeben. \cite[Vgl.][260 ff.]{johannsen_jochen_bestands-_2015}. 
Mittlerweile ist die fünfte Iteration der COUNTER-Statistiken \textit{\acrshort{COP 5}} erschienen \cite[Vgl.][]{counter_abstract_2020}.
%Im Jahr 2019 ersetzte sie die vorhergehende Version.
Die Bibliotheken sind bei dem Bezug von diesen Statitiken auf die Unterstützung der Verlage angewiesen. Diese stellen unregelmäßig die \textit{\acrshort{COP 5}}-Statistiken zur
Verfügung. Ziele der nutzungsbezogenen Evaluation sind die Identifizierung von ausleihträchtigen Medienbeständen (Vormerkungs- und Rennerlisten) und
die Deakquisition schlecht oder gar nicht genutzter Titel. Ebenso kann die Evaluation von Fernleih- und Dokumentenlieferungen Hinweise auf Bestandslücken liefern
\cite[Vgl.][255 ff.]{johannsen_jochen_bestands-_2015}. Als Konsequenz aus den COUNTER-Statistiken kann die Abbestellung von elektronischen Ressourcen resultieren.

Die nutzer:innenbezogene Evaluation ist auf den Nutzer:innenkreis der Bibliothek und dessen Informationsbedürfnisse zentriert. 
%Fundamental ist der Unterschied zwischen den einzelnen Evaulationsverfahren in der Erhebung der Daten. 
Die sammlungs- und nutzungsorientierten Evaluationsverfahren basieren auf der Erhebung von quantitativen Daten wie der Bestandsgröße oder der Anzahl von Ausleihen. 
Nutzer:innenbezogene Evaluation benutzt qualitative Daten, die sie aus Befragungen erhebt.
%Warum ist Messbarkeit von bibliothekarischen Daten wichtig?\\
%Welchen Impact für Budgetplanung können statistische Daten haben?\\

Die einzelnen Evaluationen vermitteln ein realistisches Gesamtbild der Bibliothek und deren Service-Dienstleistungen. 
Die datengetriebenen Evalutionsauswertungen bieten Hinweise auf Optimierungen der bibliothekarischen Service-Dienstleistungen. 
Die Auswertungen können durch die Bibliotheksleitung aufgenommen werden und in strategische (zukünftige) Entscheidungen einfließen. 
So kann ein detailliertes Erwerbungsprofil und somit eine gezieltere Erwerbungspolitik entstehen. 
Dadurch wird das Management der Ressourcen effektiver und effizienter \cite[Vgl.][297]{johnson_peggy_fundamentals_2014}.
Gegenüber Stakeholdern kann auf der Grundlage der Evaluationen gezielt um Budget verhandelt werden.
\begin{displayquote}
    The purpose of the statistics is to give the management of the library or another decision-maker 
    a satisfactory and correct picture about the situation of the library as a support to them - the statistics are the mirror of the library!
    \cite[463]{laitinen_markku_library_2013}
\end{displayquote}

Um ein zufriedenstellendes und korrektes Bild der Situation der Bibliothek zu präsentieren, helfen sorgsam ausgewählte Datenvisualisierungen.



\clearpage
\section{Datenvisualisierung}
\label{chap:two_two}
Im Gegensatz zur Inferenzstatistik ist die Aufgabe der Deskriptiven
Statistik die Gewinnung von Information aus Daten. Dazu werden verschiedene statistische Verfahren beziehungsweise Methoden angewendet. 
Datenvisualisierungen sind statistische Verfahren der Deskriptiven und der Explorativen Statistik.
\cites[Vgl.][3]{cleff_deskriptive_2011}[Vgl.][7 ff.]{coolidge_statistics_2021}.


Der Begriff der Datenvisualisierung umschreibt die visuelle Repräsentation und Präsentation von Daten \cite[Vgl.][15 ff.]{kirk_data_2019}.
% Ähnlich der Definition von \Citeauthor{kirk_data_2019} ist Datenvisualisierung nach \citeauthor{Cairo} 
%Oberbegriff für Informationsvisualisierung / Scientific Visualization\\
Er wird in Teilen der Literatur als Oberbegriff für \textquote{\textit{Information visualization}} und 
\textquote{\textit{Scientific Visualization}} verstanden \cite[Vgl.][11]{few_now_2009}.
% Abgrenzung zu Infographics
Datenvisualisierung grenzt sich in Form und Inhalt von dem Begriff der Infographik ab. 

Infographiken haben die Aufgabe Nachrichten zu kommunizieren.
Sie bestehen aus einer Mischung von Diagrammen, Karten, Illustrationen und Text. Klarheit und Tiefe der Darstellungen sind dabei wichtig
\cite[Vgl.][31]{cairo_truthful_2016}. Sie werden auch als \textquote{\textit{Explanation Graphics}}
bezeichnet und bestimmen sich dadurch, dass sie Geschehen und Ereignisse graphisch darstellen. 
Historisch sind Infographiken mit dem Medium der Printzeitungen und Printzeitschriften verbunden \cite[Vgl.][27]{kirk_data_2019}.

% Was ist unter Datenvisualisierung zu verstehen?\\
Datenvisualisierungen sollen die Analyse, Exploration und Entdeckung der Daten ermöglichen. Sie sollen das Verständnis der dargestellten Daten erleichtern
und sind nicht primär dafür geschaffen, Geschichten über die Informationen zu erzählen \cite[Vgl.][20 ff.]{kirk_data_2019}. 
Sie dienen als Werkzeug, um aus den visualisierten Daten Schlussfolgerungen ziehen zu können \cite[Vgl.][31]{cairo_truthful_2016}. % -> Sind Daten dargestellt > Exploration der Daten
% Mit der visuellen Darstellung kann die Exploration dieser beginnen und bietet Raum für neue Zusammenhänge...

In der Literatur finden sich verschiedene Eigenschaften von Datenvisualisierungen.
Datenvisualisierungen sollen auf gründlicher und ernsthafter Forschung basieren. Sie sind funktional, dass heißt
sie bemühen sich die Daten genau darzustellen. Darüber hinaus sollen Datenvisualisierungen attraktiv, 
aufschlussreich und erhellend sein \cite[Vgl.][45]{cairo_truthful_2016}. 
Ferner sollen sie computerunterstützt und interaktiv sein \cite[Vgl.][12]{few_now_2009}.


Datenvisualisierungen beruhen auf visuellen Elementen wie Charts, Diagrammen, Tabellen und Karten.
Mit dem Einsatz dieser visuellen Elemente sollen Muster, Trends und Ausnahmen in Daten leichter sichtbar gemacht werden.
Datenvisualisierungen setzen einen visuellen Reiz, der schneller vom menschlichen Auge verarbeitet werden kann \cite[Vgl.][32]{few_now_2009}.

%Deswegen sind mit dem Einsatz der visuellen Elemente Muster, Trends und Ausnahmen in den Daten leichter erkennbar.
% Zweck der Datendarstllung Bindung an Daten
%Ein diskretes Merkmal kann auf Basis der natürlichen Zahlen abzählbar viele Merkmalsausprägungen (values) annehmen. Die Bestandsgröße einer
%Bibliothek ist ein diskretes Merkmal. Im Gegensatz dazu können die Merkmalsausprägungen eines stetigen Merkmals jeden beliebigen Wert annehmen. 
%So ist die Raumtemperatur ein stetiges Merkmal.
% Welchen Zweck? Für wen?
Welche Datenvisualisierungen zum Einsatz kommen, wird von verschiedenen Faktoren beeinflusst. Es ist zunächst zu überlegen, welche Zwecke die Auswertung der Daten verfolgt und
für wen die Daten mit grafischen Elementen präsentiert werden. Davon hängt ab, mit welchen Schwerpunkten Merkmale und Eigenschaften präsentiert werden sollen.
\cite[Vgl.][17]{kirk_data_2019}.

% Datentyp
Datentypen sind ebenso von großer Relevanz für die zum Einsatz kommenden Datenvisualisierungen. Datentypen 
\textquote{...define the nature of the values held under each variable and about each item 
in your dataset.} \cite[99]{kirk_data_2019}. 
Eine Variable (Merkmal) kann quantitativ oder qualitativ sein.
Die \autoref{fig:data types} zeigt die statistischen Datentypen nach der \acrfull{NOIR} 
\cite[Vgl.][12 ff.]{bortz_statistik_2010} mit den möglichen Aussagengehalten und Beispielen.

 
 \begin{figure}[h]
    \centering
        \includegraphics[width=12cm]{dt}
        \caption{Datentypen mit Aussagegehalt und Beispielen}
        \label{fig:data types}
\end{figure}

In der wissenschaftlichen Literatur findet sich auch nur die Unterscheidung zwischen nominalen, ordinalen und metrischen
Merkmalen \cite[Vgl.][20]{cleff_deskriptive_2011}. Unter Berücksichtigung der großen Bandbreite (variety) an Daten, schlägt \Citeauthor{kirk_data_2019} ferner eine Erweiterung der \acrshort{NOIR}-Systematik um einen textuellen Datentyp
vor. \cite[Vgl.][100]{kirk_data_2019}

Nominale Datentypen können auch Ziffern beinhalten. 
Unterschieden werden die Merkmale ferner nach diskret und stetig.
Ein diskretes Merkmal kann auf der Basis der natürlichen Zahlen abzählbar viele Merkmalsausprägungen annehmen.
%Die Bestandsgröße eineBibliothek ist ein diskretes Merkmal. 
Im Gegensatz dazu können die Merkmalsausprägungen eines stetigen Merkmals jeden beliebigen Wert annehmen. 
%So ist die Raumtemperatur ein stetiges Merkmal. 

Der Einsatz von visuellen Elementen wird ebenfalls bestimmt von der Größe der Datenmenge. Bei kleineren Datenmengen können die Daten noch mit einer Tabelle übersichtlich dargestellt werden.
Bei größer werdenden Datenmengen reichen Tabellen aber nicht mehr aus. Deswegen können Diagramme wie Linien- oder Balkendiagramme zum Einsatz kommen. 

Schließlich ist zu bestimmen, wie die Daten der Datenmenge zueinander in Verbindung stehen. Repräsentieren die Daten eine Zeitreihe oder eine Häufigkeitsverteilung.


\section{Business-Intelligence-Systeme}

Was sind Business-Intelligence-Löungen?\\
%Wo kommen Buisiness-Intelligence-Lösungen zum Einsatz zum Einsatz?
Es gibt eine Vielzahl kommerzieller Lösungen für den Bibliotheksbereich, die auf Business-Intelligence-Software basieren.
Zu nennen wären \textit{AlmaAnalytics} für das Next-Generation-Library-System \textit{Alma} von \textit{ExLibris}\footnote{\url{https://www.exlibrisgroup.com/products/alma-library-services-platform/alma-analytics}

Stand: 26.05.2020}, \textit{BibControl} von \textit{OCLC}\footnote{\url{https://www.oclc.org/de/bibcontrol.html} Stand: 26.05.2020},
\textit{CollectionHq} von \textit{Baker \& Taylor}\footnote{\url{https://www.collectionhq.com/} Stand: 26.05.2020} oder \textit{Libinsight} von \textit{SpringShare}\footnote{\url{https://springshare.com/libinsight/} Stand: 26.05.2020}.
Darüber hinaus gibt es Business-Intelligence-Applikationen, die von
Bibliotheken für Reporting, Datenanalyse und Datenvisualisierung adaptiert werden,
wie zum Beispiel \textit{Tableau} von der Firma \textit{Tableau Software} oder
\textit{Crystal Reports} von \textit{SAP}.
Diese Applikationen sind entweder
an bestimmte Bibliothekssysteme zurückgebunden, limitiert in ihren
Funktionen\cite{golas_statistische_2018} oder zu generisch.
Überdies wird sowohl von \textit{HeBis} bzw. von der
Lokal-Bibliothekssystembetreuung als auch von der \textit{mpdl} keine Applikation
in dieser Richtung angeboten.
Ebenso ist ungewiss, wann die Ablösung des schon betagten \textit{CBS/LBS} hin zu
einem neuen Next-Generation-Library-System im \textit{HeBis-Verbund} stattfinden wird und ob
es ein Modul zur statistischen Datenerhebung liefern wird.
