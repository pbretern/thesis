\chapter{Konzeption einer Lösung}
\label{chap:four}

\section{Anforderungsanalyse}
Ausgehend von der Analyse der Ausgangssituation werden im Folgenden 
Anforderungen herausgearbeitet. Die Vision und die Ziele des Systems als Teil der Anforderungen werden kurz formuliert. Im Anschluss daran,
werden die Rahmenbedingungen formuliert. Danach werden sowohl die funktionalen Anforderungen als
auch die Qualitätsanforderungen festgelegt. Alle Anforderungen dienen als Grundlage für die Entwicklung des Prototypen. 
Anhand der Anforderungen erfolgt auch die Bewertung des Prototypen im \autoref{chap:five}. 
Die Anforderungen werden nach dem \textit{MoSCoW-Prinzip} priorisiert. 
Dabei wird unterschieden in Muss-Anforderungen (M - Must), in Soll-Anforderungen(S - Should), in Kann-Anforderungen (C - Could) und in Anforderungen,
die im Zuge der Implementierung des Prototypes (noch) nicht umgesetzt werden (W - Would / Won't).
Soweit möglich wurden für die Rahmenbedingungen und die Qualitätsanforderungen messbare Bedingungen formuliert.
\subsection{Vision und Ziele}
Die Bibliothek des \acrshort{MPI EA} soll durch das System in die Lage versetzt werden ihr Budget effizient und bedarfsgerecht zu planen.
Die Bibliotheksleitung und die Mitarbeiter:innen der Bibliothek können sich relevante \textit{\acrshort{KPI}} durch das System mit
Datenvisualisierungen anzeigen lassen.
Ausgewählte \textit{\acrshort{KPI}} werden der Institutsleitung als Standardreport präsentiert.

\subsection{Rahmenbedingungen}
Lorem ipsum dolor sit amet, consectetuer adipiscing elit. Ut purus elit, vestibulum ut. Lorem ipsum dolor sit amet, consectetuer adipiscing elit. Ut purus elit, vestibulum ut.
Lorem ipsum dolor sit amet, consectetuer adipiscing elit. Ut purus elit, vestibulum ut.
Lorem ipsum dolor sit amet, consectetuer adipiscing elit. Ut purus elit, vestibulum ut.
\begingroup
\setlength{\tabcolsep}{10pt} % Default value: 6pt
\renewcommand{\arraystretch}{1.25} 
\begin{table}[h]
    \centering
    \begin{adjustbox}{max width=\textwidth}
    \begin{tabular}{lp{6.5cm}p{6.5cm}l}
       \toprule
       \textbf{ID}          & \textbf{Beschreibung} & \textbf{Messbarkeit} & \textbf{Priorisierung}\\
       \midrule
        R10                               &Das System ist eine administrative Anwendung & -  & must\\
        R20                               &Zielgruppen für das System sind die Bibliotheksleitung und die Bibliotheksmitarbeiter:innen. & -  & must\\
        R30                               &Die Institutsleitung ist die Zielgruppe für den Standardbericht. & -  & must\\
        R40                               &Das System wird als Desktop-Anwendung in einer Büroumgebung eingesetzt. & Zugriff auf das System funktioniert nur im institutseigenen IP-Adressbereich & must\\
        R50                               &Das System wird bedarfsorientiert gestartet und beendet. & -  & must\\
        R60                               &Der Betrieb des Systems läuft unbeaufsichtigt. & -  & must\\
        R70                               &Die Entwicklungsumgebung kann identisch mit der Produktivumgebung sein. & Vergleich der Entwicklungsumgebung mit der Produktivumgebung  & could\\
        R80                               &Das System läuft auf mindestens zwei Browsern (Chrome,Safari) und auf den Betriebssystemen macOs und Linux. & Testen des Systems mit mindestens zwei Browsern und beiden Betriebssystemen  & must\\
        R90                               &Die Software-Requirements sind für das System hinterlegt. & Überprüfen, ob entsprechendes Dokument beigefügt ist  & must\\
        R100                              &Das System läuft in einem geschützten Bereich im internen Netzwerk des Institutes, der nur für das Bibliothekspersonal einsehbar ist. & Zugriff auf das System von außerhalb des IP-Adressbereiches des Institutes  & won't\\
        R110                              &Das System wird auch von anderen Bibliotheken mit einer ähnlichen Datenlage eingesetzt. & Testen des Systems in einer anderen Bibliothek & won't\\
       \bottomrule
    \end{tabular}
    \end{adjustbox}
    \caption
    \end{table}
\endgroup

% \textbf{R10} Das System ist eine administrative Anwendung.\\ 
% \textbf{R20} Zielgruppe für das System sind die Bibliotheksleitung und die Bibliotheksmitarbeiter:innen. (und für den StandardReport die Institutsleitung
% (Verwaltungsleitung / Geschäftsführung)).\\
% \textbf{R30} Das System wird als Desktop-Anwendung in einer Büroumgebung eingesetzt.\\
% \textbf{R40} Das System wird bedarfsorientiert gestartet und beendet.\\
% \textbf{R50} Die tägliche Betriebszeit des Systems wird bedarfsorientiert gesteuert.\\
% \textbf{R60} Der Betrieb des Systems soll unbeabsichtigt laufen.\\
% \textbf{R70} Die Entwicklungsumgebung kann identisch mit der Zielumgebung sein.\\
% \textbf{R80} Das System soll mindestens drei Browser (Chrome, Firefox) unterstützen und auf den Betriebssystemen MacOS und Linux laufen.\\
% \textbf{R85} Die Software-Requirements für das System sollen hinterlegt sein.\\
% \textbf{R90} Das System soll in einem geschützten Bereich im internen Netzwerk des Institutes, der nur für das Bibliothekspersonal einsehbar ist, laufen.\\
% \textbf{R100} Das System soll auch von anderen Bibliotheken mit einer ähnlichen Datenlage eingesetzt werden können.






%\subsection{Kontext und Überblick}


\subsection{Funktionale Anforderungen}
Im Folgenden werden die funktionalen Anforderungen an das System formuliert. Das System lässt sich in drei Bereiche strukturieren.\\
Die \autoref{tab:funktionale Anforderungen I} beschreibt die Anforderungen an das System bezüglich des \textit{ETL-Bereiches} und der Speicherung der Daten. 

\begingroup
\setlength{\tabcolsep}{10pt} % Default value: 6pt
\renewcommand{\arraystretch}{1.25}
\begin{table}[h]
    \centering
    \begin{adjustbox}{max width=\textwidth}
    \begin{tabular}{lp{13cm}c}
       \toprule
       \textbf{ID}          & \textbf{Beschreibung} &\textbf{Priorisierung}\\
       \midrule
        F10                               &Das System importiert die Daten automatisch. & must\\
        F20                               &Das System speichert die Daten automatisch.  & must\\
        F30                               &Das System speichert die Daten für ein Backup redundant.  & should\\
        F40                               &Das System verschiebt die importierten Daten für das Backup.  & could\\
        F50                               &Das System löscht unnötige Daten am Ende des Import-Prozesses automatisch.  & could\\
        F60                               &Das System verarbeitet die Daten mit Daten aus anderen Quellen.  & must\\
    \bottomrule
    \end{tabular}
    \end{adjustbox}
    \caption
    \end{table}
\endgroup

Der zweite Bereich beschreibt die Anforderungen an das System für die Analyse der Daten und ist abgebildet in der \autoref{tab:funktionale Anforderungen II}.

\begingroup
\setlength{\tabcolsep}{10pt} % Default value: 6pt
\renewcommand{\arraystretch}{1.25} 
\begin{table}[h]
    \centering
    \begin{adjustbox}{max width=\textwidth}
    \begin{tabular}{lp{13cm}c}
       \toprule
       \textbf{ID}          & \textbf{Beschreibung} &\textbf{Priorisierung}\\
       \midrule
        F70                               &Das System analysiert die Daten automatisch nach deskriptiven und explorativen Methoden der Statistik.  & must\\
        F80                               &Das System bietet eine grafische Benutzungsoberfläche an (GUI).  & must\\
        F90                               &Das System filtert die Daten nach bestimmten Kriterien. Filterkriterien sind in Tabelle aufgeführt. & must\\
        F100                              &Das System analysiert die gefilterten Daten automatisch nach deskriptiven und explorativen Methoden der Statistik. & must\\
        \bottomrule
    \end{tabular}
    \end{adjustbox}
    \caption
    \end{table}
\endgroup

Die Anforderungen für die Präsentation und die Erstellung des Standardberichtes beschreibt die \autoref{tab:funktionale Anforderungen III}.

\begingroup
\setlength{\tabcolsep}{10pt} % Default value: 6pt
\renewcommand{\arraystretch}{1.25} 
\begin{table}[h]
    \centering
    \begin{adjustbox}{max width=\textwidth}
    \begin{tabular}{lp{13cm}c}
       \toprule
       \textbf{ID}          & \textbf{Beschreibung} &\textbf{Priorisierung}\\
       \midrule
        F110                              &Das System bietet die Filterung der Daten über die GUI an.  & must\\
        F120                              &Das System bietet Datenvisualisierungen für die betreffenden Daten an. & must\\
        F130                              &Das System bietet eine Auswahl an Datenvisualisierungen für die jeweiligen Daten an. & should\\
        F140                              &Das System erstellt bedarfsorientiert eine PDF mit den relevanten KPIs. & must\\
        F150                              &Das System speichert das PDF-Dokument. & must\\
        F160                              &Das System speichert Diagramme der relevanten KPIs als Bilder. & must\\
        \bottomrule
    \end{tabular}
    \end{adjustbox}
    \caption
    \end{table}
\endgroup


% Speicherort\\
% Wie sollen die Daten importiert werden?\\
% Von wo sollen die Daten importiert werden?\\
% Wie sollen die Daten gespeichert werden?\\
% Wo sollen die Daten gespeichert werden?\\
% Sollen Backups der importierten Daten gemacht werden?\\
% Soll es eine log-Datei geben?\\
% Antwort: zentraler Platz\\

% Auswertung der Daten\\
% Welche Daten sollen ausgewertet werden?\\



% Visualisierung der Daten\\
% Welche Visualisierungen sind für die Daten sinnvoll?\\
% Welche Visualisierungen sollen zum Einsatz kommen?\\
% Welche Annotationen sollen zur Anwendung kommen?\\
% Welche Farben sollen zur Anwendung kommen?\\



% Interaktivität\\
% Soll aus es die Möglichkeit geben aus den Visualisierungen auszuwählen?\\
% Soll es die Filterung der Daten zur Darstellung als Möglichkeit der Interaktivität geben?\\
% Welche Möglichkeiten der Interaktivität soll es geben (Filterung, Highlighting, Animation)\\
\subsection{Qualitätsanforderungen}
Lorem ipsum dolor sit amet, consectetuer adipiscing elit. Ut purus elit, vestibulum ut, 
placerat ac, adipiscing vitae, felis. Curabitur dictum gravida mauris. Nam arcu libero, nonummy eget, consectetuer id, vulputate a, magna.
\begingroup
\setlength{\tabcolsep}{10pt} % Default value: 6pt
\renewcommand{\arraystretch}{1.25} 
\begin{table}[h]
    \centering
    \begin{adjustbox}{max width=\textwidth}
    \begin{tabular}{lp{6.5cm}p{6.5cm}c}
       \toprule
       \textbf{ID}          & \textbf{Beschreibung} & \textbf{Messbarkeit} & \textbf{Priorisierung}\\
       \midrule
        Q10                               &Das System ist portierbar auf eine andere Plattform. & Testen auf Linux und macOs & must\\
        Q20                               &Das System ist leicht erlernbar.& Schulungsdauer <= 1h  & must\\
        Q30                               &Der Zugriff auf das System erfolgt passwortgeschützt. & -  & won't\\
        Q40                               &Der Aufbau des Systems ist modular. Die Module sind testbar. & Tests der Module unabhängig voneinander & should\\
        Q50                               &Das System verfügt zur Analyse der Daten stets über die aktuellsten Daten. & Überprüfen auf aktuellste Daten & must\\
        Q60                               &Zur Programmierung des Systems wird freie Software (Programmiersprachen) genutzt. & Überprüfen, ob Software freie Lizenz enthält  & must\\
        Q70                               &Die eingesetzte Software ist weitverbreitet. & Überprüfen der marketshare der Software & must\\
        Q80                               &Das System ist einfach zu warten.. & - & should\\
        Q90                               &Das System ist unter einer Open-Source-Lizenz zu entwickeln. & Vergabe einer freien Lizenz für das zu entwickelnde System  & must\\
        Q100                              &Das System ist zu 99,9 Prozent zuverlässig.  &- & should\\
        Q110                              &Die Reaktionszeit des Systems auf Benutzungsanfragen beträgt <= 2 Sekunden. & Tests mit dem System& should\\
        Q120                              &Auf das System wird in vollem Umfang von mehreren Endgeräten gleichzeitig zugegriffen. (Mehrbenutzerzugriff) & Tests mit mehreren Systemen, die gleichzeitig auf System zugreifen& should\\
        Q130                              &Das Layout des Dashboards ist strukturiert, übersichtlich und selbsterklärend. & Erfassen relevanter Informationen <= 5 Sekunden& must\\
        Q140                              &Das Layout des Dashboards ist am Corporate Design des Institutes auszurichten. & -& won't\\
        %Q150                              &Die verschiedenen Diagrammtypen werden zielgerichtet eingesetzt. & -& must
        %Q160                              &Optische Gestaltungsmerkmale und Farben der Datenvisualisierungen werden zur Verdeutlichung der Information eingesetzt. & -& must\\
        %Q170                              &Optische Effekte oder Animationen der Datenvisualisierungen sind sparsam einzusetzen. & -& must\\
        %Q180                              &Die Darstellung der Informationen ist auf die Zielgruppen zugeschnitten. & -& must\\
        %Q190                              &Die Datenvisualisierung zielt auf die Beantwortung der spezifischen Fragestellung. & -& must\\
        %Q200                              &Zahlen und Daten werden im passenden Kontext präsentiert. & -& must\\
       \bottomrule
    \end{tabular}
    \end{adjustbox}
    \caption
    \end{table}
\endgroup
% Was sind nicht-funktionale Anforderungen?\\
% \textbf{Q10} Das System soll portierbar auf eine andere Plattform sein.\\
% \textbf{Q20} Das System ist leicht erlernbar.\\
% \textbf{Q30} Der Zugriff auf das System soll passwortgeschützt erfolgen.\\
% \textbf{Q40} Zum Testen der einzelnen Bestandteile soll das System modular aufgebaut sein.\\
% \textbf{Q50} Das System verfügt zur Analyse der Daten stets über die aktuellsten Daten\\
% \textbf{Q60} Zur Programmierung des Systems muss freie Software (Programmiersprachen) genutzt werden\\
% \textbf{Q70} Die eingesetzte Software ist weitverbreitet.\\
% \textbf{Q80} Das System ist einfach zu warten.\\
% \textbf{Q10} Das System ist unter einer Open-Source-Lizenz zu entwickeln.\\
% \textbf{Q11} Das System ist zu 99,9 Prozent zuverlässig.\\
% \textbf{Q12} Die Reaktionszeit des Systems auf Benutzungsanfragen beträgt <= 5 Sekunden
% \textbf{Q13} Auf das System kann in vollem Umfang von mehreren Endgeräten gleichzeitig zugegriffen werden. (Mehrbenutzerzugriff)



%\subsection{Abnahmekriterien}
%\subsection{Anwendungsfälle}
%Was sind Anwendungsfälle (welche Daten aus den bibliothekarischen GG)? 
%\footnote{misto}