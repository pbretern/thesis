\chapter{Konzeption einer Lösung}
\label{chap:four}

\section{Anforderungsanalyse}
Ausgehend von der Analyse der Ausgangssituation werden im Folgenden funktionale und nichtfunktionale
Anforderungen herausgearbeitet. Diese dienen als Grundlage des Prototypen. Anhand dieser Anforderungen erfolgt 
auch die Bewertung des Prototypen. Priorisiert werden die Anforderungen nach dem MoSCoW-Prinzip. 
Dabei wird unterschieden in Muss-Anforderungen (M - Must), in Soll-Anforderungen(S - Should), in Kann-Anforderungen (C - Could) und in Anforderungen,
die (noch) nicht umgesetzt werden (W - Would / Won't).

\subsection{Ziel}
Das Ziel der Anforderungsanalyse ist die Bestimmung der  Anforderungen für die Umsetzung des Prototypen.
Die Anforderungen werden in einer Tabelle mit einer Identifikationsnummer 


\subsection{Funktionale Anforderungen}
Was sind funktionalen Anforderungen?\\
Speicherort\\
Wie sollen die Daten importiert werden?\\
Von wo sollen die Daten importiert werden?\\
Wie sollen die Daten gespeichert werden?\\
Wo sollen die Daten gespeichert werden?\\
Sollen Backups der importierten Daten gemacht werden?\\
Soll es eine log-Datei geben?\\
Antwort: zentraler Platz\\

Auswertung der Daten\\
Welche Daten sollen ausgewertet werden?\\



Visualisierung der Daten\\
Welche Visualisierungen sind für die Daten sinnvoll?\\
Welche Visualisierungen sollen zum Einsatz kommen?\\
Welche Annotationen sollen zur Anwendung kommen?\\
Welche Farben sollen zur Anwendung kommen?\\



Interaktivität\\
Soll aus es die Möglichkeit geben aus den Visualisierungen auszuwählen?\\
Soll es die Filterung der Daten zur Darstellung als Möglichkeit der Interaktivität geben?\\
Welche Mögkichkeiten der Interaktivität soll es geben (Filterung, Highliting, Animation)\\
\subsection{Nicht funktionale Anforderungen}
Was sind nicht-funktionale Anforderungen?
\subsection{Anwendungsfälle}
Was sind Anwendungsfälle (welche Daten aus den bibliothekarischen GG)? 
\footnote{misto}