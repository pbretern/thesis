\chapter{Konzeption einer Lösung}
\label{chap:four}

\section{Anforderungsanalyse}
Ausgehend von der Analyse der Ausgangssituation werden im Folgenden 
Anforderungen herausgearbeitet. Die Vision und die Ziele des Systems als Teil der Anforderungen werden kurz formuliert. Im Anschluss daran,
werden die Rahmenbedingungen formuliert. Danach werden sowohl die funktionalen Anforderungen als
auch die Qualitätsanforderungen festgelegt. Alle Anforderungen dienen als Grundlage für die Entwicklung des Prototypen. 
Anhand der Anforderungen erfolgt auch die Bewertung des Prototypen im \autoref{chap:five}. 
Die Anforderungen werden nach dem \textit{MoSCoW-Prinzip} priorisiert. 
Dabei wird unterschieden in Muss-Anforderungen (M - Must), in Soll-Anforderungen(S - Should), in Kann-Anforderungen (C - Could) und in Anforderungen,
die im Zuge der Implementierung des Prototypes (noch) nicht umgesetzt werden (W - Would / Won't).
Soweit möglich wurden für die Rahmenbedingungen und die Qualitätsanforderungen messbare Bedingungen formuliert. Im \autoref{chap:four_one_five} werden
vier Use-Cases für das System beschrieben, die es zu erfüllen hat.

\subsection{Vision und Ziele}
Die Bibliothek des \acrshort{MPI EA} soll durch das System in die Lage versetzt werden ihr Budget effizient und bedarfsgerecht zu planen.
Die Bibliotheksleitung und die Mitarbeiter:innen der Bibliothek können sich relevante \textit{\acrshort{KPI}} durch das System mit
Datenvisualisierungen anzeigen lassen.
Ausgewählte \textit{\acrshort{KPI}} werden der Institutsleitung als Standardreport präsentiert.

\subsection{Rahmenbedingungen}
Die Rahmenbedingungen legen die organisatorischen Anforderungen für das zu entwickelnde System fest. 
Darunter zählen die die Anwendungsbereiche, die Zielgruppe und technische Anforderungen. 
\begingroup
\setlength{\tabcolsep}{10pt} % Default value: 6pt
\renewcommand{\arraystretch}{1.25} 
\begin{table}[h]
    \centering
    \begin{adjustbox}{max width=\textwidth}
    \begin{tabular}{lp{6.5cm}p{6.5cm}l}
       \toprule
       \textbf{ID}          & \textbf{Beschreibung} & \textbf{Messbarkeit} & \textbf{Priorisierung}\\
       \midrule
        R1                                &Das System ist eine administrative Anwendung & -  & must\\
        R2                                &Zielgruppen für das System sind die Bibliotheksleitung und die Bibliotheksmitarbeiter:innen. & -  & must\\
        R3                                &Die Institutsleitung ist die Zielgruppe für den Standardbericht. & -  & must\\
        R4                                &Das System wird als Desktop-Anwendung in einer Büroumgebung eingesetzt. & Zugriff auf das System funktioniert nur im institutseigenen IP-Adressbereich & must\\
        R5                                &Das System wird bedarfsorientiert gestartet und beendet. & -  & must\\
        R6                                &Der Betrieb des Systems läuft unbeaufsichtigt. & -  & must\\
        R7                                &Die Entwicklungsumgebung kann identisch mit der Produktivumgebung sein. & Vergleich der Entwicklungsumgebung mit der Produktivumgebung  & could\\
        %R8                               &Das System läuft auf mindestens zwei Browsern (Chrome,Safari) und auf den Betriebssystemen macOs und Linux. & Testen des Systems mit mindestens zwei Browsern und beiden Betriebssystemen  & must\\
        R8                                &Die Software-Requirements sind für das System hinterlegt. & Überprüfen, ob entsprechendes Dokument beigefügt ist  & must\\
        R9                                &Das System läuft in einem geschützten Bereich im internen Netzwerk des Institutes, der nur für das Bibliothekspersonal einsehbar ist. & Zugriff auf das System von außerhalb des IP-Adressbereiches des Institutes  & won't\\
        R10                              &Das System wird auch von anderen Bibliotheken mit einer ähnlichen Datenlage eingesetzt. & Testen des Systems in einer anderen Bibliothek & won't\\
       \bottomrule
    \end{tabular}
    \end{adjustbox}
    \caption
    \end{table}
\endgroup

% \textbf{R10} Das System ist eine administrative Anwendung.\\ 
% \textbf{R20} Zielgruppe für das System sind die Bibliotheksleitung und die Bibliotheksmitarbeiter:innen. (und für den StandardReport die Institutsleitung
% (Verwaltungsleitung / Geschäftsführung)).\\
% \textbf{R30} Das System wird als Desktop-Anwendung in einer Büroumgebung eingesetzt.\\
% \textbf{R40} Das System wird bedarfsorientiert gestartet und beendet.\\
% \textbf{R50} Die tägliche Betriebszeit des Systems wird bedarfsorientiert gesteuert.\\
% \textbf{R60} Der Betrieb des Systems soll unbeabsichtigt laufen.\\
% \textbf{R70} Die Entwicklungsumgebung kann identisch mit der Zielumgebung sein.\\
% \textbf{R80} Das System soll mindestens drei Browser (Chrome, Firefox) unterstützen und auf den Betriebssystemen MacOS und Linux laufen.\\
% \textbf{R85} Die Software-Requirements für das System sollen hinterlegt sein.\\
% \textbf{R90} Das System soll in einem geschützten Bereich im internen Netzwerk des Institutes, der nur für das Bibliothekspersonal einsehbar ist, laufen.\\
% \textbf{R100} Das System soll auch von anderen Bibliotheken mit einer ähnlichen Datenlage eingesetzt werden können.






%\subsection{Kontext und Überblick}


\subsection{Funktionale Anforderungen}
Im Folgenden werden die funktionalen Anforderungen an das System formuliert. Das System lässt sich in drei Bereiche strukturieren.\\
Die \autoref{tab:funktionale Anforderungen I} beschreibt die Anforderungen an das System bezüglich des \textit{ETL-Bereiches} und der Speicherung der Daten. 

\begingroup
\setlength{\tabcolsep}{10pt} % Default value: 6pt
\renewcommand{\arraystretch}{1.25}
\begin{table}[h]
    \centering
    \begin{adjustbox}{max width=\textwidth}
    \begin{tabular}{lp{13cm}c}
       \toprule
       \textbf{ID}          & \textbf{Beschreibung} &\textbf{Priorisierung}\\
       \midrule
        F1                               &Das System importiert die Daten automatisch. & must\\
        F2                               &Das System speichert die Daten automatisch.  & must\\
        F3                               &Das System speichert automatisch die Daten für ein Backup redundant.  & should\\
        %F40                               &Das System verschiebt die importierten Daten für das Backup.  & could\\
        F4                               &Das System löscht unnötige Daten am Ende des Import-Prozesses automatisch.  & could\\
        F5                               &Das System verarbeitet die Daten mit Daten aus anderen Quellen.  & must\\
    \bottomrule
    \end{tabular}
    \end{adjustbox}
    \caption
    \end{table}
\endgroup

Der zweite Bereich beschreibt die Anforderungen an das System für die Analyse der Daten und ist abgebildet in der \autoref{tab:funktionale Anforderungen II}.

\begingroup
\setlength{\tabcolsep}{10pt} % Default value: 6pt
\renewcommand{\arraystretch}{1.25} 
\begin{table}[h]
    \centering
    \begin{adjustbox}{max width=\textwidth}
    \begin{tabular}{lp{13cm}c}
       \toprule
       \textbf{ID}          & \textbf{Beschreibung} &\textbf{Priorisierung}\\
       \midrule
        F6                               &Das System bietet eine grafische Benutzungsoberfläche an (GUI).  & must\\
        F7                               &Das System analysiert die Daten automatisch nach deskriptiven und explorativen Methoden der Statistik.  & must\\
        F8                               &Das System filtert die Daten nach bestimmten Kriterien. Filterkriterien sind in Tabelle aufgeführt. & must\\
        F9                              &Das System analysiert die gefilterten Daten automatisch nach deskriptiven und explorativen Methoden der Statistik. & must\\
        \bottomrule
    \end{tabular}
    \end{adjustbox}
    \caption
    \end{table}
\endgroup

Die Anforderungen für die Präsentation und die Erstellung des Standardberichtes beschreibt die \autoref{tab:funktionale Anforderungen III}.

\begingroup
\setlength{\tabcolsep}{10pt} % Default value: 6pt
\renewcommand{\arraystretch}{1.25} 
\begin{table}[H]
    \centering
    \begin{adjustbox}{max width=\textwidth}
    \begin{tabular}{lp{13cm}c}
       \toprule
       \textbf{ID}          & \textbf{Beschreibung} &\textbf{Priorisierung}\\
       \midrule
        F10                              &Das System bietet die Filterung der Daten über die GUI an.  & must\\
        F11                              &Das System bietet Datenvisualisierungen für die betreffenden Daten an. & must\\
        F12                              &Das System bietet eine Auswahl an Datenvisualisierungen für die jeweiligen Daten an. & should\\
        F13                              &Das System erstellt bedarfsorientiert eine PDF mit den relevanten KPIs. & must\\
        F14                              &Das System speichert das PDF-Dokument. & must\\
        F15                              &Das System speichert Diagramme der relevanten KPIs als Bilder. & must\\
        \bottomrule
    \end{tabular}
    \end{adjustbox}
    \caption
    \end{table}
\endgroup


% Speicherort\\
% Wie sollen die Daten importiert werden?\\
% Von wo sollen die Daten importiert werden?\\
% Wie sollen die Daten gespeichert werden?\\
% Wo sollen die Daten gespeichert werden?\\
% Sollen Backups der importierten Daten gemacht werden?\\
% Soll es eine log-Datei geben?\\
% Antwort: zentraler Platz\\

% Auswertung der Daten\\
% Welche Daten sollen ausgewertet werden?\\



% Visualisierung der Daten\\
% Welche Visualisierungen sind für die Daten sinnvoll?\\
% Welche Visualisierungen sollen zum Einsatz kommen?\\
% Welche Annotationen sollen zur Anwendung kommen?\\
% Welche Farben sollen zur Anwendung kommen?\\



% Interaktivität\\
% Soll aus es die Möglichkeit geben aus den Visualisierungen auszuwählen?\\
% Soll es die Filterung der Daten zur Darstellung als Möglichkeit der Interaktivität geben?\\
% Welche Möglichkeiten der Interaktivität soll es geben (Filterung, Highlighting, Animation)\\
\subsection{Nicht funktionale Anforderungen}
In der \autoref{tab: nfAnforderungen} werden die nicht funktionalen Anforderungen, die Qualitätskrierien an das System
beschreiben, dargelegt.
\begingroup
\setlength{\tabcolsep}{9pt} % Default value: 6pt
\renewcommand{\arraystretch}{1.0} 
\begin{table}[H]
    \centering
    \begin{adjustbox}{max width=\textwidth}
    \begin{tabular}{lp{7.0cm}p{7.0cm}c}
       \toprule
       \textbf{ID}          & \textbf{Beschreibung} & \textbf{Messbarkeit} & \textbf{Priorisierung}\\
       \midrule
        NF1                               &Das System ist portierbar auf eine andere Plattform. & Testen auf unterschiedlichen Systemen (OS, Browser). & must\\
        NF2                               &Das System ist leicht erlernbar.& Schulungsdauer <= 1 Stunde  & must\\
        NF3                               &Der Zugriff auf das System erfolgt passwortgeschützt. & -  & won't\\
        NF4                               &Der Aufbau des Systems ist modular. Die Module sind testbar. & Tests der Module unabhängig voneinander. & should\\
        NF5                               &Das System verfügt zur Analyse der Daten stets über die aktuellsten Daten. & Überprüfen auf aktuellste Daten. & must\\
        NF6                               &Zur Programmierung des Systems wird freie Software (Programmiersprachen) genutzt. & Überprüfen, ob Software freie Lizenz enthält.  & must\\
        NF7                               &Die eingesetzte Software ist weitverbreitet. & Überprüfen der Verbreitung der Software. & must\\
        NF8                               &Das System ist einfach zu warten. & - & should\\
        NF9                               &Das System ist unter einer Open-Source-Lizenz zu entwickeln. & Vergabe einer freien Lizenz für das zu entwickelnde System.  & must\\
        %Q100                              &Das System ist zu 99,9 Prozent zuverlässig.  &- & should\\
        NF10                              &Die Reaktionszeit des Systems auf Benutzungsanfragen beträgt <= 2 Sekunden. & Tests mit dem System& should\\
        NF11                              &Auf das System wird in vollem Umfang von mehreren Endgeräten gleichzeitig zugegriffen. (Mehrbenutzerzugriff) & Tests mit mehreren Systemen, die gleichzeitig auf System zugreifen& should\\
        NF12                              &Das Layout des Dashboards ist strukturiert, übersichtlich und selbsterklärend. & Erfassen relevanter Informationen <= 5 Sekunden.& must\\
        NF13                              &Das Layout des Dashboards ist am Corporate Design des Institutes auszurichten. & -& won't\\
        NF14                              &Die verschiedenen Diagrammtypen werden zielgerichtet eingesetzt. & -& must\\
        NF15                              &Optische Gestaltungsmerkmale und Farben der Datenvisualisierungen werden zur Verdeutlichung der Information eingesetzt. &- & must\\
        NF16                              &Optische Effekte oder Animationen der Datenvisualisierungen sind sparsam einzusetzen. & -& must\\
        NF17                              &Die Darstellung der Informationen ist auf die Zielgruppen zugeschnitten. & -& must\\
        NF18                              &Die Datenvisualisierung zielt auf die Beantwortung der spezifischen Fragestellung. & -& must\\
        NF19                              &Zahlen und Datenvisualisierungen werden im passenden Kontext präsentiert. & -& must\\
       \bottomrule
    \end{tabular}
    \end{adjustbox}
    \caption
    \end{table}
\endgroup
% Was sind nicht-funktionale Anforderungen?\\
% \textbf{Q10} Das System soll portierbar auf eine andere Plattform sein.\\
% \textbf{Q20} Das System ist leicht erlernbar.\\
% \textbf{Q30} Der Zugriff auf das System soll passwortgeschützt erfolgen.\\
% \textbf{Q40} Zum Testen der einzelnen Bestandteile soll das System modular aufgebaut sein.\\
% \textbf{Q50} Das System verfügt zur Analyse der Daten stets über die aktuellsten Daten\\
% \textbf{Q60} Zur Programmierung des Systems muss freie Software (Programmiersprachen) genutzt werden\\
% \textbf{Q70} Die eingesetzte Software ist weitverbreitet.\\
% \textbf{Q80} Das System ist einfach zu warten.\\
% \textbf{Q10} Das System ist unter einer Open-Source-Lizenz zu entwickeln.\\
% \textbf{Q11} Das System ist zu 99,9 Prozent zuverlässig.\\
% \textbf{Q12} Die Reaktionszeit des Systems auf Benutzungsanfragen beträgt <= 5 Sekunden
% \textbf{Q13} Auf das System kann in vollem Umfang von mehreren Endgeräten gleichzeitig zugegriffen werden. (Mehrbenutzerzugriff)



%\subsection{Abnahmekriterien}
\subsection{Anwendungsfälle}
\label{chap:four_one_five}
Im Folgenden Abschnitt werden die Anwendungsfälle dargestellt, die das System beantworten soll. Dabei wird auf die in \autoref{tab:Statistische_Daten} aufgeführten
Evaluationstypen referiert und eine Auswahl für die Anwendungsfälle getroffen.
\subsubsection{Anwendungsfall 1}

\begingroup
\setlength{\tabcolsep}{10pt} % Default value: 6pt
\renewcommand{\arraystretch}{1.25} 
\begin{table}[H]
    \centering
    \begin{adjustbox}{max width=\textwidth}
    \begin{tabular}{lp{7.0cm}p{6.5cm}}
       \toprule
       \textbf{}          & \textbf{Beschreibung} &\textbf{Systemverhalten}\\
       \midrule
        Titel                            &Ausleihzahlen Bibliotheksbestand  & -\\
        Akteur:in                        &Bibliotheksleitung, Bibliotheksmitarbeiter:innen & -\\
        Ziel                             &Anzeige der Ausleihzahlen des Bestandes nach Jahr und Monat. & Das System generiert eine Anzeige mit den jeweiligen Parametern.\\
        Vorbedingungen                   &Die Bibliotheksleitung und die Bibliotheksmitarbeiter:innen haben Zugriff auf das System. & Das System ist im Betrieb.\\
        Inhalt                           &Die Bibliotheksleitung oder die Bibliotheksmitarbeiter:innen wählen den gewünschten Zeitraum aus und lassen sich die Ausleihzahlen anzeigen. & Das System filtert die betreffenden Datensätze nach Gesamtzeitraum und Jahr. Das System zeigt diese Datensätze mit Datenvisualisierungen an.\\
                                         %& &Das System zeigt diese Datensätze mit Datenvisualisierungen an.\\
                                         & &Das System zeigt die ausleihstärksten Titel absteigend.\\
                                         & &Das System zeigt die Verteilung der ausgeliehenen Titel in der RVK-Aufstellungssystematik pro Jahr an.\\
                                         & &Das System zeigt den Trend bezüglich der Anzahl der ausgeliehenen Titel im Vergleich zum Gesamtbestand für den gesamten Zeitraum an.\\
                                         & &Das System zeigt die Ausleihen pro Jahr und für den Gesamtzeitraum im Vergleich zum Bestand und zum Bestandswachstum.\\
        Ausnahmeverhalten               &- &- \\
        Anforderungen                   &?& -\\ 
        \bottomrule
    \end{tabular}
    \end{adjustbox}
    \caption
    \end{table}
\endgroup

\subsubsection{Anwendungsfall 2}

\begingroup
\setlength{\tabcolsep}{10pt} % Default value: 6pt
\renewcommand{\arraystretch}{1.25} 
\begin{table}[H]
    \centering
    \begin{adjustbox}{max width=\textwidth}
    \begin{tabular}{lp{7.0cm}p{7.0cm}}
       \toprule
       \textbf{}          & \textbf{Beschreibung} &\textbf{Systemverhalten}\\
       \midrule
        Titel                            &Ausleihzahlen bibliotheksinterne Lieferdienste& -\\
        Akteur:innen                     &Bibliotheksleitung, Bibliotheksmitarbeiter:innen& -\\
        Ziel                             &Anzeige der Ausleihzahlen der bibliotheksinternen Lieferdienste.& Das System generiert eine Anzeige mit den jeweiligen Parametern\\
        Vorbedingungen                   &Die Bibliotheksleitung und die Bibliotheksmitarbeiter:innen haben Zugriff auf das System.& Das System ist im Betrieb.\\
        Interaktionsfolge                &Die Bibliotheksleitung oder die Bibliotheksmitarbeiter:innen wählen den gewünschten Zeitraum aus und lassen sich die Ausleihzahlen bibliotheksinterner Lieferdienste anzeigen. & Das System filtert die betreffenden Datensätze nach Gesamtzeitraum und Jahr. Das System zeigt diese Datensätze Datenvisualisierungen an.\\
                                        & &Das System zeigt die Nutzung der verschiedenen internen Lieferservices nach Gesamtzeitraum, Jahr, und Monat an.\\
                                        & &Das System zeigt die verschiedenen internen Lieferservices nach Nutzung der Institutsabteilungen an.\\
                                        & &Das System zeigt Trends in der Nutzung der verschiedenen Lieferservices  durch die Abteilungen nach Jahr und im Gesamtzeitraum an.\\
        Ausnahmeverhalten               &- & -\\
        Nachbedingungen                 &?& -\\

        Anforderungen                   &?& -\\
        \bottomrule
    \end{tabular}
    \end{adjustbox}
    \caption
    \end{table}
\endgroup

\subsubsection{Anwendungsfall 3}

\begingroup
\setlength{\tabcolsep}{10pt} % Default value: 6pt
\renewcommand{\arraystretch}{1.25} 
\begin{table}[H]
    \centering
    \begin{adjustbox}{max width=\textwidth}
    \begin{tabular}{lp{7.0cm}p{7.0cm}}
       \toprule
       \textbf{}          & \textbf{Beschreibung} &\textbf{Systemverhalten}\\
       \midrule
        Titel                            &Lesesaalnutzung& -\\
        Akteur:innen                     &Bibliotheksleitung, Bibliotheksmitarbeiter:innen& -\\
        Ziel                             &Anzeige der Nutzung des Lesesaals während der Öffnungszeiten.& Das System generiert eine Anzeige mit den jeweiligen Parametern\\
        Vorbedingungen                   &Die Bibliotheksleitung und die Bibliotheksmitarbeiter:innen haben Zugriff auf das System.& Das System ist im Betrieb.\\
        Interaktionsfolge                &Die Bibliotheksleitung oder die Bibliotheksmitarbeiter:innen lassen sich die Nutzung des Lesesaals in einem gewünschten Zeitraum anzeigen. & Das System filtert die betreffenden Datensätze nach Gesamtzeitraum und Jahr. Das System zeigt diese Datensätze mit Datenvisualisierungen an.\\
                                         %& &Das System zeigt diese Datensätze mit Datenvisualisierungen.\\
                                         & &Das System zeigt die Nutzung des Lesesaals nach Monat, Jahr, gefiltert in drei Zeitschichten.\\
        Ausnahmeverhalten               &- & -\\
        Nachbedingungen                 &?& -\\

        Anforderungen                   &?& -\\
        \bottomrule
    \end{tabular}
    \end{adjustbox}
    \caption
    \end{table}
\endgroup



\subsubsection{Anwendungsfall 4}

\begingroup
\setlength{\tabcolsep}{10pt} % Default value: 6pt
\renewcommand{\arraystretch}{1.25} 
\begin{table}[H]
    \centering
    \begin{adjustbox}{max width=\textwidth}
    \begin{tabular}{lp{7.0cm}p{7.0cm}}
       \toprule
       \textbf{}          & \textbf{Beschreibung} &\textbf{Systemverhalten}\\
       \midrule
        Titel                            &Neuerwerbungen& -\\
        Akteur:innen                     &Bibliotheksleitung, Bibliotheksmitarbeiter:innen& -\\
        Ziel                             &Anzeige der Anzahl der Neuerwerbungen nach RVK-Systematikstelle pro Monat.& Das System generiert eine Anzeige mit den jeweiligen Parametern\\
        Vorbedingungen                   &Die Bibliotheksleitung und die Bibliotheksmitarbeiter:innen haben Zugriff auf das System.& Das System ist im Betrieb.\\
        Interaktionsfolge                &Die Bibliotheksleitung und die Bibliotheksmitarbeiter:innen wählen einen Monat über ein Dropdown-Menu aus.& Das System filtert die betreffenden Datensätze nach Gesamtzeitraum und Jahr. Das System zeigt diese Datensätze mit Datenvisualisierungen an.\\
                                         & &Das System zeigt die Gesamtzahl der Neuerwerbungen nach Medientyp pro Monat.\\
                                         & &Das System zeigt die Anzahl der Neuerwerbungen nach Medientyp pro Monat nach RVK-Systematikgruppen an.\\
        Ausnahmeverhalten               &- & -\\
        Nachbedingungen                 &?& -\\

        Anforderungen                   &?& -\\
        \bottomrule
    \end{tabular}
    \end{adjustbox}
    \caption
    \end{table}
\endgroup


\subsubsection{Anwendungsfall 5}

\begingroup
\setlength{\tabcolsep}{10pt} % Default value: 6pt
\renewcommand{\arraystretch}{1.25} 
\begin{table}[H]
    \centering
    \begin{adjustbox}{max width=\textwidth}
    \begin{tabular}{lp{7.0cm}p{7.0cm}}
       \toprule
       \textbf{}          & \textbf{Beschreibung} &\textbf{Systemverhalten}\\
       \midrule
        Titel                            &Bestandswachstum& -\\
        Akteur:innen                     &Bibliotheksleitung, Bibliotheksmitarbeiter:innen& -\\
        Ziel                             &Anzeige des Wachstums des Bibliotheksbestandes nach einzelnen RVK-Systematikgruppen.& Das System generiert eine Anzeige mit den jeweiligen Parametern\\
        Vorbedingungen                   &Die Bibliotheksleitung und die Bibliotheksmitarbeiter:innen haben Zugriff auf das System.& Das System ist im Betrieb.\\
        Interaktionsfolge                &Die Bibliotheksleitung und die Bibliotheksmitarbeiter:innen wählen über ein DropDown-Menu RVK-Bestandsgruppen aus.& Das System filtert die betreffenden Datensätze für die RVK-Bestandsgruppen. Das System zeigt diese Datensätze mit Datenvisualisierungen an.\\
                                         & &Das System zeigt die Gesamtzahl der Titel nach Jahren an.\\
                                         & &Das System zeigt die Anzahl der Titel pro RVK-Systematikstelle an.\\
                                         & &Das System zeigt die Gesamtzahl der Titel und die Zahl der RVK-Systematikstellen an.\\
                                         & &Das System zeigt die Anzahl der Titel nach Medienart für jede RVK-Systematikstelle pro Jahr an.\\
                                         & &Das System zeigt Trends in der Verteilung der Titel in den RVK-Systematikgruppen über den Gesamtzeitraum an.\\
        Ausnahmeverhalten               &- & -\\
        Nachbedingungen                 &?& -\\

        Anforderungen                   &?& -\\
        \bottomrule
    \end{tabular}
    \end{adjustbox}
    \caption
    \end{table}
\endgroup


\subsubsection{Anwendungsfall 6}

\begingroup
\setlength{\tabcolsep}{10pt} % Default value: 6pt
\renewcommand{\arraystretch}{1.25} 
\begin{table}[H]
    \centering
    \begin{adjustbox}{max width=\textwidth}
    \begin{tabular}{lp{7.0cm}p{7.0cm}}
       \toprule
       \textbf{}          & \textbf{Beschreibung} &\textbf{Systemverhalten}\\
       \midrule
        Titel                            &Budget- und Umsatzübersicht& -\\
        Akteur:innen                     &Bibliotheksleitung, Bibliotheksmitarbeiter:innen& -\\
        Ziel                             &Anzeige der Budget- und Umsatzübersicht für den gesamten Zeitraum und das laufende Jahr.& Das System generiert eine Anzeige mit den jeweiligen Parametern\\
        Vorbedingungen                   &Die Bibliotheksleitung und die Bibliotheksmitarbeiter:innen haben Zugriff auf das System.& Das System ist im Betrieb.\\
        Interaktionsfolge                &Die Bibliotheklsleitung und die Bibliotheksmitarbeiter:innen können sich den Lieferanten auswählen und den Gesamtumsatz und den Umsatz pro Jahr ansehen.& Das System filtert die betreffenden Datensätze für den Lieferanten. Das System zeigt diese Datensätze mit Datenvisualisierungen an.\\
                                         &Die Bibliotheksleitung und die Bibliotheksmitarbeiter:innen können sich die Budgetübersicht nach Kostenstelle auswählen und Gesamtbudget per Kostenstelle und pro Jahr ansehen. &Das System filtert die betreffenden Datensätze für die Kostenstellen. Das System zeigt diese Datensätze mit Datenvisualisierungen an.\\
                                         &Die Bibliotheksleitung und die Bibliotheksmiarbeiter:innen können sich den Verlauf des Gesamtbudgets und der Gesmtumsätze über den Gesamtzeitraum ansehen. &Das System zeigt das Gesamtbudget und den Gesamtumsatz über den Gesamtzeitraum an.\\
                                         & &Das System zeigt Trends für das Gesamtbudget und den Gesamtumsatz über den Zeitraum an.\\
        Ausnahmeverhalten               &- & -\\
        Nachbedingungen                 &?& -\\

        Anforderungen                   &?& -\\
        \bottomrule
    \end{tabular}
    \end{adjustbox}
    \caption
    \end{table}
\endgroup


\subsubsection{Anwendungsfall 7}

\begingroup
\setlength{\tabcolsep}{10pt} % Default value: 6pt
\renewcommand{\arraystretch}{1.25} 
\begin{table}[H]
    \centering
    \begin{adjustbox}{max width=\textwidth}
    \begin{tabular}{lp{7.0cm}p{7.0cm}}
       \toprule
       \textbf{}          & \textbf{Beschreibung} &\textbf{Systemverhalten}\\
       \midrule
        Titel                            &Standardbericht& -\\
        Akteur:innen                     &Bibliotheksleitung, Bibliotheksmitarbeiter:innen& -\\
        Ziel                             &Genierung eines Standardberichts mit den relevanten \textit{\acrshort{KPI}} der Bibliothek zur Vorlage bei der Institutsleitung. & Das System generiert eine Anzeige mit den jeweiligen Parametern\\
        Vorbedingungen                   &Die Bibliotheksleitung und die Bibliotheksmitarbeiter:innen haben Zugriff auf das System.& \\
        Interaktionsfolge                &Die Bibliotheksmitarbeiter:innen lösen ein Skript zur Generierung des Standardberichtes aus.& Das System speichert Diagrammbilder und Texte.\\
        &                                &Das System speichert die Diagrammbilder und Texte an den jeweilig dafür vorgesehenden Ort.\\
        &                                &Das System generiert mit den Diagrammbildern und Texten das PDF.\\
        &                                &Das System speichert die generierte PDF.\\
       
        Ausnahmeverhalten               &- & -\\
        Nachbedingungen                 &?& -\\

        Anforderungen                   &?& -\\
        \bottomrule
    \end{tabular}
    \end{adjustbox}
    \caption
    \end{table}
\endgroup

%Was sind Anwendungsfälle (welche Daten aus den bibliothekarischen GG)? 
%\footnote{misto}
