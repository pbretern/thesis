\chapter{Konzeption einer Lösung}
\label{chap:four}
Ausgehend von der Analyse der Ausgangssituation werden im Folgenden 
Anforderungen für das zu entwickelnde System herausgearbeitet.
Alle Anforderungen dienen als Grundlage für die Entwicklung des Proof-of-Concepts. 
Schließlich werden 7 Anwendungsfälle für das System im \autoref{chap:four_one_five} beschrieben, die es zu erfüllen hat.
Anhand der Anforderungen und der Anwendungsfälle erfolgt im \autoref{chap:five} die Bewertung des Proof-of-Concepts.


\section{Anforderungsanalyse}
Die Vision und die Ziele des Systems werden zunächst kurz formuliert. Im Anschluss daran,
werden die Rahmenbedingungen erarbeitet. Danach werden sowohl die funktionalen Anforderungen als
auch die nicht-funktionalen Anforderungen festgelegt. 


Die Anforderungen werden nach dem \textit{MoSCoW-Prinzip} priorisiert. 
Hierbei wird unterschieden in Muss-Anforderungen (M - must), in Soll-Anforderungen (S - should), in Kann-Anforderungen (C - could) und in Anforderungen,
die im Zuge der Implementierung des Proof-of-Concepts (noch) nicht umgesetzt werden (W - would / won't).
Soweit nötig wurden für die Rahmenbedingungen und für die nicht-funktionalen Anforderungen Messbarkeitskriterien formuliert. 

\subsection{Vision und Ziele}
Die Bibliothek des \textit{\acrshort{MPI EA}} soll durch das System in die Lage versetzt werden, ihr Budget effizient und bedarfsgerecht zu planen.
Die Bibliotheksleitung und die Bibliotheksmitarbeiter:innen können sich relevante \textit{\acrlong{KPI}} durch das System mit
Datenvisualisierungen anzeigen lassen. Relevante \textit{\acrshort{KPI}} sind die Budget- und Umsatzübersichten, die Ausleihzahlen, das Bestandswachstum und die
Neuerwerbungen. 
Ausgewählte \textit{\acrshort{KPI}} werden an die Institutsleitung als Standardreport verteilt.

\subsection{Rahmenbedingungen}
Die Rahmenbedingungen legen die organisatorischen Anforderungen für das zu entwickelnde System fest. 
Darunter fallen die Anwendungsbereiche, die unterschiedlichen Zielgruppen und die technischen Anforderungen. 
\begingroup
\setlength{\tabcolsep}{10pt} % Default value: 6pt
\renewcommand{\arraystretch}{1.25} 
\begin{table}[h]
    \centering
    \begin{adjustbox}{max width=\textwidth}
    \begin{tabular}{lp{7.5cm}p{7.5cm}l}
       \toprule
       \textbf{ID}          & \textbf{Beschreibung} & \textbf{Messbarkeit} & \textbf{Priorisierung}\\
       \midrule
        %R1                                &Das System ist eine administrative Anwendung. & -  & M\\
        R1                                &Zielgruppen für das System sind die Bibliotheksleitung und die Bibliotheksmitarbeiter:innen. & -  & M\\
        R2                                &Die Institutsleitung ist die Zielgruppe für den Standardbericht, der aus dem System erzeugt wird. & -  & M\\
        R3                                &Das System wird als Desktop-Anwendung in einer Büroumgebung eingesetzt. & Das System wird nicht optimiert für mobile Endgeräte wie Smartphones oder Tablets. & M\\
        R4                                &Das System wird bedarfsorientiert gestartet und beendet. & -  & M\\
        R5                                &Der Betrieb des Systems läuft unbeaufsichtigt. & -  & M\\
        R6                                &Die Entwicklungsumgebung kann identisch mit der Produktivumgebung sein. & Vergleich der Entwicklungsumgebung mit der Produktivumgebung.  & C\\
        %R8                               &Das System läuft auf mindestens zwei Browsern (Chrome,Safari) und auf den Betriebssystemen macOs und Linux. & Testen des Systems mit mindestens zwei Browsern und beiden Betriebssystemen  & M\\
        R7                                &Die Softwareanforderungen sind für das System dokumentiert. & Überprüfen, ob entsprechendes Dokument vorhanden ist.  & M\\
        R8                                &Die Hardwareanforderungen sind für das System dokumentiert. & Überprüfen, ob entsprechendes Dokument vorhanden ist.  & M\\
        R9                               &Das System läuft in einem geschützten Bereich im internen Netzwerk des Institutes, der nur für das Bibliothekspersonal einsehbar ist. & Testen des Zugriffs von außerhalb des institutseigenen IP-Adressbereich und von bibliotheksfremden Mitarbeiter:innen des Instituts.& W\\
        R10                               &Das System wird auch von anderen Bibliotheken mit einer ähnlichen Datenlage eingesetzt. & Testen des Systems durch eine andere Bibliothek & W\\
       \bottomrule
    \end{tabular}
    \end{adjustbox}
    \caption
    \end{table}
\endgroup

% \textbf{R10} Das System ist eine administrative Anwendung.\\ 
% \textbf{R20} Zielgruppe für das System sind die Bibliotheksleitung und die Bibliotheksmitarbeiter:innen. (und für den StandardReport die Institutsleitung
% (Verwaltungsleitung / Geschäftsführung)).\\
% \textbf{R30} Das System wird als Desktop-Anwendung in einer Büroumgebung eingesetzt.\\
% \textbf{R40} Das System wird bedarfsorientiert gestartet und beendet.\\
% \textbf{R50} Die tägliche Betriebszeit des Systems wird bedarfsorientiert gesteuert.\\
% \textbf{R60} Der Betrieb des Systems soll unbeabsichtigt laufen.\\
% \textbf{R70} Die Entwicklungsumgebung kann identisch mit der Zielumgebung sein.\\
% \textbf{R80} Das System soll mindestens drei Browser (Chrome, Firefox) unterstützen und auf den Betriebssystemen MacOS und Linux laufen.\\
% \textbf{R85} Die Software-Requirements für das System sollen hinterlegt sein.\\
% \textbf{R90} Das System soll in einem geschützten Bereich im internen Netzwerk des Institutes, der nur für das Bibliothekspersonal einsehbar ist, laufen.\\
% \textbf{R100} Das System soll auch von anderen Bibliotheken mit einer ähnlichen Datenlage eingesetzt werden können.






%\subsection{Kontext und Überblick}


\subsection{Funktionale Anforderungen}
Im Folgenden werden die funktionalen Anforderungen an das System formuliert. Das System strukturiert sich in die Bereiche des \textit{\acrshort{ETL}-Prozesses} und der
Datenspeicherung, der Datenanalyse, der Datenpräsentation und des Standardberichtes.\\
Die \autoref{tab:funktionale Anforderungen I} beschreibt die Anforderungen an das System bezüglich des \textit{\acrshort{ETL}-Prozesses} und der Datenspeicherung.\\

\begingroup
\setlength{\tabcolsep}{10pt} % Default value: 6pt
\renewcommand{\arraystretch}{1.25}
\begin{table}[h]
    \centering
    \begin{adjustbox}{max width=\textwidth}
    \begin{tabular}{lp{15cm}c}
       \toprule
       \textbf{ID}          & \textbf{Beschreibung} &\textbf{Priorisierung}\\
       \midrule
        F1                              &Das System importiert automatisch die Daten von einem lokalem Verzeichnis. & M\\
        F2                              &Das System bereinigt automatisch die Daten von syntaktischen Fehlern und vollzieht Formatanpassungen (einheitliches Datumsformat, einheitliches Zeichenformat, einheitliches Tabellenformat). & M \\
        %F2                             &Das System bereinigt, harmonisiert, verdichtet (aggregiert) und reichert die Daten automatisch an. & M\\
        F3                              %&Das System bereinigt die Daten automatisch von semantischen Fehlern (fehlende Datenwerte). & M \\
        F4                              &Das System harmonisiert die Daten automatisch (Erkennen und Harmonisierung von unterschiedlichen Codierungen der Datenwerte, Erkennen und Zusammenführung von Synonymen). & M \\
        F5                              &Das System ergänzt die Daten mit zusätzlichen Daten aus anderen Datenquellen (z.B. RVK-Systematik). & M \\
        F6                             &Das System speichert automatisch die Daten an einem definierten Speicherort.  & M\\
        %F4                              &Das System verarbeitet die Daten mit Daten aus anderen Datenquellen.  & M\\
        F7                              &Das System speichert automatisch redundant die Daten für ein Backup an einem definierten Ort.  & S\\
        %F40                             &Das System verschiebt die importierten Daten für das Backup.  & c\\
        F8                              &Das System löscht nach dem Importprozess automatisch die Daten aus dem lokalen Verzeichnis.  & C\\
        F9                              &Das System dokumentiert zur Fehlererkennung den \textit{\acrshort{ETL}-} und den Speicherprozess in einer log-Datei. & W\\
    \bottomrule
    \end{tabular}
    \end{adjustbox}
    \caption
    \end{table}
\endgroup

\autoref{tab:funktionale Anforderungen II} legt die Anforderungen  für die Datenanalyse an das System fest.

\begingroup
\setlength{\tabcolsep}{10pt} % Default value: 6pt
\renewcommand{\arraystretch}{1.25} 
\begin{table}[h]
    \centering
    \begin{adjustbox}{max width=\textwidth}
    \begin{tabular}{lp{15cm}c}
       \toprule
       \textbf{ID}          & \textbf{Beschreibung} &\textbf{Priorisierung}\\
       \midrule
        F10                              &Das System bietet eine \acrfull{GUI} an.  & M\\
        F11                              &Das System analysiert die Daten automatisch mit deskriptiven und explorativen Methoden der Statistik nach Kriterien, die in den Anwendungsfällen formuliert sind.  & M\\
        F12                              &Das System filtert die Daten nach bestimmten Kriterien, die in den Anwendungsfällen formuliert sind.  & M\\ %Filterkriterien sind in Tabelle aufgeführt.
        F13                              &Das System bietet die Filterung der Daten über die \textit{\acrshort{GUI}} auf Basis von Filtern, die von den Benutzer:innen eingegeben werden, an.  & M\\
        F14                              &Das System analysiert die gefilterten Daten automatisch mit deskriptiven und explorativen Methoden der Statistik nach Kriterien, die in den Anwendungsfällen formuliert sind. & M\\                             
        \bottomrule
    \end{tabular}
    \end{adjustbox}
    \caption
    \end{table}
\endgroup


Die Anforderungen für die Datenpräsentation und die Erstellung des Standardberichtes sind in \autoref{tab:funktionale Anforderungen III} aufgelistet.
%\dots
\begingroup
\setlength{\tabcolsep}{10pt} % Default value: 6pt
\renewcommand{\arraystretch}{1.25} 
\begin{table}[h]
    \centering
    \begin{adjustbox}{max width=\textwidth}
    \begin{tabular}{lp{15cm}c}
       \toprule
       \textbf{ID}          & \textbf{Beschreibung} &\textbf{Priorisierung}\\
       \midrule
        F15                              &Das System bietet für die betreffenden Daten Datenvisualisierungen an. & M\\
        F16                              &Die angebotenen Datenvisualisierungen sind hauptsächlich Linien- und Balkendiagramme. & M\\
        F17                              &Das System bietet für die betreffenden Daten eine Auswahl an diesen Datenvisualisierungen an. & S\\
        F18                              &Das System erstellt bedarfsorientiert ein PDF-Dokument mit den relevanten \textit{\acrshort{KPI}}. & M\\
        F19                              &Das System speichert bedarfsorientiert Diagramme der relevanten \textit{\acrshort{KPI}} in einem platzsparenden Bildformat. & C\\
        F20                              &Das System greift auf Diagrammbilder und Texte für die Generierung des PDF-Dokumentes automatisch zu. &M\\
        F21                              &Das System öffnet das PDF-Dokument automatisch. & M\\
        \bottomrule
    \end{tabular}
    \end{adjustbox}
    \caption
    \end{table}
\endgroup


% Speicherort\\
% Wie sollen die Daten importiert werden?\\
% Von wo sollen die Daten importiert werden?\\
% Wie sollen die Daten gespeichert werden?\\
% Wo sollen die Daten gespeichert werden?\\
% Sollen Backups der importierten Daten gemacht werden?\\
% Soll es eine log-Datei geben?\\
% Antwort: zentraler Platz\\

% Auswertung der Daten\\
% Welche Daten sollen ausgewertet werden?\\



% Visualisierung der Daten\\
% Welche Visualisierungen sind für die Daten sinnvoll?\\
% Welche Visualisierungen sollen zum Einsatz kommen?\\
% Welche Annotationen sollen zur Anwendung kommen?\\
% Welche Farben sollen zur Anwendung kommen?\\



% Interaktivität\\
% Soll aus es die Möglichkeit geben aus den Visualisierungen auszuwählen?\\
% Soll es die Filterung der Daten zur Darstellung als Möglichkeit der Interaktivität geben?\\
% Welche Möglichkeiten der Interaktivität soll es geben (Filterung, Highlighting, Animation)\\
\clearpage
\subsection{Nicht-funktionale Anforderungen}
In der \autoref{tab: nfAnforderungen} sind die nicht-funktionalen Anforderungen, die Qualitätskriterien an das System
beschreiben, aufgelistet.
\begingroup
\setlength{\tabcolsep}{10pt} % Default value: 6pt
\renewcommand{\arraystretch}{1.25} 
\begin{table}[h]
    \centering
    \begin{adjustbox}{max width=\textwidth}
    \begin{tabular}{lp{7.5cm}p{7.5cm}c}
       \toprule
       \textbf{ID}          & \textbf{Beschreibung} & \textbf{Messbarkeit} & \textbf{Priorisierung}\\
       \midrule
        NF1                               &Das System ist portierbar auf eine andere Plattform. & Testen mit unterschiedlichen Systemen (OS, Browser). & M\\
        NF2                               &Das System ist leicht erlernbar.& Schulungsdauer weniger als 1 Stunde.  & M\\
        NF3                               &Der Zugriff auf das System erfolgt passwortgeschützt. & -  & W\\
        NF4                               &Das System ist modular aufgebaut. Die einzelnen Module sind testbar. & Module sind unabhängig voneinander testbar. & S\\
        %NF5                               &Das System verfügt zur Datenanalyse stets über die aktuellsten Daten. & Überprüfen auf aktuellste Daten. & M\\
        NF5                               &Zur Programmierung des Systems wird freie Software (Programmiersprachen) genutzt. & Überprüfen, ob Software freie Lizenz enthält.  & M\\
        NF6                               &Die eingesetzte Software ist weitverbreitet und geeignet für diese Aufgabe. & Überprüfen der Verbreitung und Eignung der Software. & M\\
        NF7                               &Das System ist einfach zu warten. & - & S\\
        NF8                               &Das System ist unter einer Open-Source-Lizenz zu entwickeln. & Vergabe einer freien Lizenz für das zu entwickelnde System.  & M\\
        %Q100                              &Das System ist zu 99,9 Prozent zuverlässig.  &- & S\\
        NF9                              &Die Reaktionszeit des Systems auf Benutzungsanfragen über die \textit{\acrshort{GUI}} beträgt weniger als 2 Sekunden. & Tests mit dem System.& S\\
        NF10                              &Auf das System kann in vollem Umfang von mehreren Endgeräten gleichzeitig zugegriffen. & Tests mit mehreren Endgeräten, die gleichzeitig auf System zugreifen.& S\\
        NF11                              &Das Layout des Dashboards ist strukturiert, übersichtlich und selbsterklärend. & Erfassen relevanter Informationen in weniger als 5 Sekunden.& M\\
        NF12                              &Die Entwicklung des Systems erfolgt mit modernen Technologien. & - & M\\
        NF13                              &Das Layout des Dashboards ist am Corporate Design des Institutes auszurichten. & -& W\\
        NF14                              &Die verschiedenen Diagrammtypen werden zielgerichtet eingesetzt. & -& M\\
        NF15                              &Optische Gestaltungsmerkmale und Farben der Datenvisualisierungen werden zur Verdeutlichung der Information eingesetzt. &- & M\\
        NF16                              &Optische Effekte oder Animationen der Datenvisualisierungen sind sparsam einzusetzen. & -& M\\
        NF17                              &Die Darstellung der Informationen ist auf die Zielgruppen zugeschnitten. & -& M\\
        NF18                              &Die Datenvisualisierung zielt auf die Beantwortung der spezifischen Anwendungsfälle. & -& M\\
        %NF20                              &Zahlen und Datenvisualisierungen werden im passenden Kontext präsentiert. & -& M\\
       \bottomrule
    \end{tabular}
    \end{adjustbox}
    \caption
    \end{table}
\endgroup
% Was sind nicht-funktionale Anforderungen?\\
% \textbf{Q10} Das System soll portierbar auf eine andere Plattform sein.\\
% \textbf{Q20} Das System ist leicht erlernbar.\\
% \textbf{Q30} Der Zugriff auf das System soll passwortgeschützt erfolgen.\\
% \textbf{Q40} Zum Testen der einzelnen Bestandteile soll das System modular aufgebaut sein.\\
% \textbf{Q50} Das System verfügt zur Analyse der Daten stets über die aktuellsten Daten\\
% \textbf{Q60} Zur Programmierung des Systems muss freie Software (Programmiersprachen) genutzt werden\\
% \textbf{Q70} Die eingesetzte Software ist weitverbreitet.\\
% \textbf{Q80} Das System ist einfach zu warten.\\
% \textbf{Q10} Das System ist unter einer Open-Source-Lizenz zu entwickeln.\\
% \textbf{Q11} Das System ist zu 99,9 Prozent zuverlässig.\\
% \textbf{Q12} Die Reaktionszeit des Systems auf Benutzungsanfragen beträgt <= 5 Sekunden
% \textbf{Q13} Auf das System kann in vollem Umfang von mehreren Endgeräten gleichzeitig zugegriffen werden. (Mehrbenutzerzugriff)



%\subsection{Abnahmekriterien}
\subsection{Anwendungsfälle}
\label{chap:four_one_five}
Im folgenden Abschnitt werden die Anwendungsfälle dargestellt, die das System beantworten soll. Für die Entwicklung des Proof-of-Concepts wurde eine Auswahl aus den bibliothekarischen Dienstleistungsbereichen getroffen. 
Dabei wird auf die in \autoref{tab:Statistische_Daten} aufgeführte Evaluationstypen referiert.
Jeder Anwendungsfall enthält den Namen des Anwendungsfalls (Titel), den bibliothekarischen Evaluationstyp, die beteiligten Akteur:innen, das zu erreichende Ziel und
den Inhalt. Zudem werden Vorbedingungen und die Anforderungen aufgeführt.
%das Ausnahmeverhalten 
Das Systemverhalten wird zu den jeweiligen Punkten ebenfalls beschrieben.


%\subsubsection{Anwendungsfall 1}
\noindent
\textit{Anwendungsfall 1}
%lorem et ipsum. lorem et ipsum. lorem et ipsum. lorem et ipsum. lorem et ipsum. lorem et ipsum. lorem et ipsum. lorem et ipsum.
\begingroup
\setlength{\tabcolsep}{10pt} % Default value: 6pt
\renewcommand{\arraystretch}{1.15} 
\begin{table}[h]
    \centering
    \begin{adjustbox}{max width=\textwidth}
    \begin{tabular}{lp{7.5cm}p{7.5cm}}
       \toprule
       \textbf{}          & \textbf{Beschreibung} &\textbf{Systemverhalten}\\
       \midrule
        Titel                            &Ausleihzahlen Bibliotheksbestand  & -\\
        Evaluationstyp                   &Nutzungsbezogen                   & -\\
        Akteur:innen                     &Bibliotheksleitung, Bibliotheksmitarbeiter:innen & -\\
        Ziel                             &Anzeige der Ausleihzahlen des Bestandes nach Jahr. & Das System generiert eine Anzeige mit den jeweiligen Parametern.\\
        Vorbedingungen                   &Die Bibliotheksleitung und die Bibliotheksmitarbeiter:innen haben Zugriff auf das System. & Das System ist im Betrieb.\\
        Inhalt                           &Die Bibliotheksleitung oder die Bibliotheksmitarbeiter:innen lassen sich auf der \textit{\acrshort{GUI}} die Ausleihzahlen pro Jahr anzeigen. & Das System filtert die betreffenden Datensätze nach Jahr. Das System zeigt diese Datensätze mit Datenvisualisierungen an.\\
                                         %& &Das System zeigt diese Datensätze mit Datenvisualisierungen an.\\
                                         & &Das System zeigt darüberhinaus die ausleihstärksten Titel absteigend nach Anzahl und Jahr an.\\
                                         & &Das System zeigt darüberhinaus die Verteilung der ausgeliehenen Titel in der \textit{\acrshort{RVK}}-Aufstellungssystematik pro Jahr an.\\
                                         & &Das System zeigt darüberhinaus die Anzahl der Ausleihen pro Jahr und für den Gesamtzeitraum\footnotemark im Vergleich zum Bestandswachstum pro Jahr und für den Gesamtzeitraum an.\\
                                         & &Das System zeigt darüberhinaus die Entwicklung bezüglich der Anzahl der ausgeliehenen Titel im Vergleich zum Gesamtbestand für den Gesamtzeitraum an.\\

        %Ausnahmeverhalten               &- &- \\
        Anforderungen                   &R1, R4, F2-F5, F10-F17, NF14-NF18& -\\ 
        \bottomrule
    \end{tabular}
    \end{adjustbox}
    \caption
    \end{table}
\footnotetext{Der Gesamtzeitraum bezieht sich hier und im Folgenden auf den angegebenen Zeitraum in \autoref{tab:Statistische_Daten}}

\endgroup


\newpage
%\subsubsection{Anwendungsfall 2}
\noindent
\textit{Anwendungsfall 2}

%lorem et ipsum. lorem et ipsum. lorem et ipsum. lorem et ipsum. lorem et ipsum. lorem et ipsum. lorem et ipsum. lorem et ipsum.

\begingroup
\setlength{\tabcolsep}{10pt} % Default value: 6pt
\renewcommand{\arraystretch}{1.25} 
\begin{table}[h]
    \centering
    \begin{adjustbox}{max width=\textwidth}
    \begin{tabular}{lp{7.0cm}p{7.0cm}}
       \toprule
       \textbf{}          & \textbf{Beschreibung} &\textbf{Systemverhalten}\\
       \midrule
        Titel                            &Ausleihzahlen bibliotheksinterne Lieferdienste& -\\
        Evaluationstyp                   &Nutzungsbezogen                   & -\\
        Akteur:innen                     &Bibliotheksleitung, Bibliotheksmitarbeiter:innen& -\\
        Ziel                             &Anzeige der Ausleihzahlen der bibliotheks-internen Lieferdienste.& Das System generiert eine Anzeige mit den jeweiligen Parametern.\\
        Vorbedingungen                   &Die Bibliotheksleitung und die Bibliotheksmitarbeiter:innen haben Zugriff auf das System.& Das System ist im Betrieb.\\
        Inhalt                          &Die Bibliotheksleitung oder die Bibliotheksmitarbeiter:innen wählen den gewünschten Zeitraum aus und lassen sich die Ausleihzahlen bibliotheksinterner Lieferdienste auf der \textit{\acrshort{GUI}} anzeigen. & Das System filtert die betreffenden Datensätze nach Gesamtzeitraum und Jahr. Das System zeigt diese Datensätze mit Datenvisualisierungen an.\\
                                        & &Das System zeigt die Nutzung der verschiedenen internen Lieferservices nach Gesamtzeitraum, Jahr und Monat an.\\
                                        & &Das System zeigt darüberhinaus die verschiedenen internen Lieferservices nach Nutzung der Institutsabteilungen an.\\
                                        & &Das System zeigt darüberhinaus die Entwicklung in der Nutzung der verschiedenen Lieferservices durch die Abteilungen nach Jahr und im Gesamtzeitraum an.\\
        %Ausnahmeverhalten               &- & -\\
        %Nachbedingungen                 &-& -\\
        Anforderungen                   &R1, R4, F2-F5 F10-F17, NF9, NF14-NF18& -\\
        \bottomrule
    \end{tabular}
    \end{adjustbox}
    \caption
    \end{table}
\endgroup

%\subsubsection{Anwendungsfall 3}
%lorem et ipsum. lorem et ipsum. lorem et ipsum. lorem et ipsum. lorem et ipsum. lorem et ipsum. lorem et ipsum. lorem et ipsum.
%\subsubsection{Anwendungsfall 2}
\newpage
\noindent
\textit{Anwendungsfall 3}

\begingroup
\setlength{\tabcolsep}{10pt} % Default value: 6pt
\renewcommand{\arraystretch}{1.25} 
\begin{table}[h]
    \centering
    \begin{adjustbox}{max width=\textwidth}
    \begin{tabular}{lp{7.0cm}p{7.0cm}}
       \toprule
       \textbf{}          & \textbf{Beschreibung} &\textbf{Systemverhalten}\\
       \midrule
        Titel                            &Lesesaalnutzung& -\\
        Evaluationstyp                   &Nutzungsbezogen                   & -\\
        Akteur:innen                     &Bibliotheksleitung, Bibliotheksmitarbeiter:innen& -\\
        Ziel                             &Anzeige der Nutzung des Lesesaals während der Öffnungszeiten.& Das System generiert eine Anzeige mit den jeweiligen Parametern.\\
        Vorbedingungen                   &Die Bibliotheksleitung und die Bibliotheksmitarbeiter:innen haben Zugriff auf das System.& Das System ist im Betrieb.\\
        Inhalt                          &Die Bibliotheksleitung oder die Bibliotheksmitarbeiter:innen lassen sich die Nutzung des Lesesaals auf der \textit{\acrshort{GUI}} anzeigen. & Das System filtert die betreffenden Datensätze nach Gesamtzeitraum und Jahr. Das System zeigt diese Datensätze mit Datenvisualisierungen an.\\
                                         %& &Das System zeigt diese Datensätze mit Datenvisualisierungen.\\
                                         & &Das System zeigt die Nutzung des Lesesaals nach Monat und Jahr, gruppiert in drei Zeitschichten an.\\
        %Ausnahmeverhalten               &- & -\\
        %Nachbedingungen                 &?& -\\

        Anforderungen                   &R1, R4, F2-F4, F10-F17, NF9, NF14-NF18& -\\
        \bottomrule
    \end{tabular}
    \end{adjustbox}
    \caption
    \end{table}
\endgroup


\newpage
\noindent
\textit{Anwendungsfall 4}

\begingroup
\setlength{\tabcolsep}{10pt} % Default value: 6pt
\renewcommand{\arraystretch}{1.25} 
\begin{table}[h]
    \centering
    \begin{adjustbox}{max width=\textwidth}
    \begin{tabular}{lp{7.0cm}p{7.0cm}}
       \toprule
       \textbf{}          & \textbf{Beschreibung} &\textbf{Systemverhalten}\\
       \midrule
        Titel                            &Neuerwerbungen& -\\
        Evaluationstyp                   &Sammlungsbezogen                   & -\\
        Akteur:innen                     &Bibliotheksleitung, Bibliotheksmitarbeiter:innen& -\\
        Ziel                             &Anzeige der Anzahl der Neuerwerbungen nach \textit{\acrshort{RVK}}-Systematikstelle pro Monat.& Das System generiert eine Anzeige mit den jeweiligen Parametern.\\
        Vorbedingungen                   &Die Bibliotheksleitung und die Bibliotheksmitarbeiter:innen haben Zugriff auf das System.& Das System ist im Betrieb.\\
        Inhalt                           &Die Bibliotheksleitung und die Bibliotheksmitarbeiter:innen wählen einen Monat auf der \textit{\acrshort{GUI}} aus.& Das System filtert die betreffenden Datensätze nach Monat. Das System zeigt diese Datensätze mit Datenvisualisierungen an.\\
                                        & &Das System zeigt die Anzahl der Neuerwerbungen nach Medientyp pro Monat nach \textit{\acrshort{RVK}}-Systematikstelle an.\\
                                        & &Das System zeigt darüberhinaus die Gesamtzahl der Neuerwerbungen nach Medientyp pro Monat an.\\
                                        & &Das System zeigt darüberhinaus die Anzahl der monatlichen Neuerwerbungen des laufenden Monats im Vergleich zu den Monaten vergangener Jahre an.\\
        %Ausnahmeverhalten               &- & -\\
        %Nachbedingungen                 &?& -\\

        Anforderungen                   &R1, R4, F2-F5, F10-F17, NF9, NF14-NF18& -\\
        \bottomrule
    \end{tabular}
    \end{adjustbox}
    \caption
    \end{table}
\endgroup


%\subsubsection{Anwendungsfall 5}
%lorem et ipsum. lorem et ipsum. lorem et ipsum. lorem et ipsum. lorem et ipsum. lorem et ipsum. lorem et ipsum. lorem et ipsum.
%\subsubsection{Anwendungsfall 2}
\newpage
\noindent
\textit{Anwendungsfall 5}

\begingroup
\setlength{\tabcolsep}{10pt} % Default value: 6pt
\renewcommand{\arraystretch}{1.25} 
\begin{table}[h]
    \centering
    \begin{adjustbox}{max width=\textwidth}
    \begin{tabular}{lp{7.0cm}p{7.0cm}}
       \toprule
       \textbf{}          & \textbf{Beschreibung} &\textbf{Systemverhalten}\\
       \midrule
        Titel                            &Bestandswachstum& -\\
        Evaluationstyp                   &Sammlungsbezogen                   & -\\
        Akteur:innen                     &Bibliotheksleitung, Bibliotheksmitarbeiter:innen& -\\
        Ziel                             &Anzeige des Wachstums des Bibliotheksbestandes nach einzelnen \textit{\acrshort{RVK}}-Systematikstellen.& Das System generiert eine Anzeige mit den jeweiligen Parametern.\\
        Vorbedingungen                   &Die Bibliotheksleitung und die Bibliotheksmitarbeiter:innen haben Zugriff auf das System.& Das System ist im Betrieb.\\
        Inhalt                           &Die Bibliotheksleitung und die Bibliotheksmitarbeiter:innen lassen sich auf der \textit{\acrshort{GUI}} das Bestandswachstum nach  \textit{\acrshort{RVK}}-Systematikstellen anzeigen.& Das System filtert die betreffenden Datensätze für die  \textit{\acrshort{RVK}}-Systematikstellen. Das System zeigt diese Datensätze mit Datenvisualisierungen an.\\
                                         & &Das System zeigt die Anzahl der Titel pro \textit{\acrshort{RVK}}-Systematikstelle an.\\
                                         & &Das System zeigt darüberhinaus die Gesamtzahl der Titel nach Jahren an.\\
                                         & &Das System zeigt darüberhinaus die Gesamtzahl der Titel und die Zahl der \textit{\acrshort{RVK}}-Systematikstellen an.\\
                                         & &Das System zeigt darüberhinaus die Anzahl der Titel nach Medienart für jede \textit{\acrshort{RVK}}-Systematikstelle pro Jahr an.\\
                                         & &Das System zeigt darüberhinaus die Entwicklung in der Verteilung der Titel in den \textit{\acrshort{RVK}}-Systematikstellen über den Gesamtzeitraum an.\\
        %Ausnahmeverhalten               &- & -\\
        %Nachbedingungen                 &?& -\\

        Anforderungen                   &R1, R4, F2-F5, F10-F17, NF9, NF14-NF18& -\\
        \bottomrule
    \end{tabular}
    \end{adjustbox}
    \caption
    \end{table}
\endgroup


\newpage
\noindent
\textit{Anwendungsfall 6}
%\subsubsection{Anwendungsfall 6}
%lorem et ipsum. lorem et ipsum. lorem et ipsum. lorem et ipsum. lorem et ipsum. lorem et ipsum. lorem et ipsum. lorem et ipsum.

\begingroup
\setlength{\tabcolsep}{10pt} % Default value: 6pt
\renewcommand{\arraystretch}{1.25} 
\begin{table}[h]
    \centering
    \begin{adjustbox}{max width=\textwidth}
    \begin{tabular}{lp{7.0cm}p{7.0cm}}
       \toprule
       \textbf{}          & \textbf{Beschreibung} &\textbf{Systemverhalten}\\
       \midrule
        Titel                            &Budget- und Umsatzübersicht& -\\
        Evaluationstyp                   &Sammlungsbezogen                   & -\\
        Akteur:innen                     &Bibliotheksleitung, Bibliotheksmitarbeiter:innen& -\\
        Ziel                             &Anzeige der Budget- und Umsatzübersicht für den Gesamtzeitraum und das laufende Jahr.& Das System generiert eine Anzeige mit den jeweiligen Parametern.\\
        Vorbedingungen                   &Die Bibliotheksleitung und die Bibliotheksmitarbeiter:innen haben Zugriff auf das System.& Das System ist im Betrieb.\\
        Inhalt                           &Die Bibliotheklsleitung und die Bibliotheksmitarbeiter:innen können sich den Lieferanten auswählen und den Gesamtumsatz und den Umsatz pro Jahr ansehen.& Das System filtert die betreffenden Datensätze für den Lieferanten. Das System zeigt diese Datensätze mit Datenvisualisierungen an.\\
                                         &Die Bibliotheksleitung und die Bibliotheksmitarbeiter:innen können sich die Budgetübersicht nach Kostenstelle auswählen und Gesamtbudget per Kostenstelle und pro Jahr ansehen. &Das System filtert die betreffenden Datensätze für die Kostenstellen. Das System zeigt diese Datensätze mit Datenvisualisierungen an.\\
                                         &Die Bibliotheksleitung und die Bibliotheksmitarbeiter:innen können sich den Verlauf des Budgets und der Umsätze über den Gesamtzeitraum und über das laufende Jahr anzeigen lassen. &Das System zeigt das Budget und den Umsatz für den Gesamtzeitraum und für das laufende Jahr an.\\
                                         & &Das System zeigt darüberhinaus Entwicklung für das Budget und den Umsatz über den Gesamtzeitraum an.\\
        %Ausnahmeverhalten               &- & -\\
        %Nachbedingungen                 &-& -\\

        Anforderungen                   &R1, R4, F2-F4, F10-F17, NF9, NF14-NF18& -\\
        \bottomrule
    \end{tabular}
    \end{adjustbox}
    \caption
    \end{table}
\endgroup


%\subsubsection{Anwendungsfall 7}
%lorem et ipsum. lorem et ipsum. lorem et ipsum. lorem et ipsum. lorem et ipsum. lorem et ipsum. lorem et ipsum. lorem et ipsum.

\newpage
\noindent
\textit{Anwendungsfall 7}

\begingroup
\setlength{\tabcolsep}{10pt} % Default value: 6pt
\renewcommand{\arraystretch}{1.25} 
\begin{table}[h]
    \centering
    \begin{adjustbox}{max width=\textwidth}
    \begin{tabular}{lp{7.0cm}p{7.0cm}}
       \toprule
       \textbf{}          & \textbf{Beschreibung} &\textbf{Systemverhalten}\\
       \midrule
        Titel                            &Standardbericht& -\\
        Evaluationstyp                   &-                   & -\\
        Akteur:innen                     &Bibliotheksleitung, Bibliotheksmitarbeiter:innen, Institutsleitung& -\\
        Ziel                             &Generierung eines Standardberichts mit den relevanten \textit{\acrshort{KPI}} der Bibliothek zur Vorlage bei der Institutsleitung. & Das System generiert eine Anzeige mit den jeweiligen Parametern.\\
        Vorbedingungen                   &Die Bibliotheksleitung und die Bibliotheksmitarbeiter:innen haben Zugriff auf das System.& \\
        Inhalt                           &Die Bibliotheksmitarbeiter:innen lösen ein Skript zur Generierung des Standardberichtes aus.& Das System greift auf Bilder und Texte an einem definierten Speicherort zu.\\
                                         & &Das System generiert aus diesen Bildern und Texten automatisch ein PDF-Dokument.\\
                                         & &Das System öffnet das PDF-Dokument automatisch.\\
                                         &Die Bibliotheksleitung oder die Bibliotheksmitarbeiter:innen speichern das PDF-Dokument an einem Speicherort. &Speicherung des PDF-Dokument systemunabhängig.\\
                                         &Die Bibliotheksleitung oder die Bibliotheksmitarbeiter:innen verteilen das PDF-Dokument an die Institutsleitung. &-\\

       
        %Ausnahmeverhalten               &- & -\\
        %Nachbedingungen                 &?& -\\

        Anforderungen                   &R2, F18-F21, NF14-NF18& -\\
        \bottomrule
    \end{tabular}
    \end{adjustbox}
    \caption
    \end{table}
\endgroup

%Was sind Anwendungsfälle (welche Daten aus den bibliothekarischen GG)? 
%\footnote{misto}