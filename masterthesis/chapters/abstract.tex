\begin{center}
    \textsc{Zusammenfassung}
  \end{center}
  %
  \noindent
  %
Aufgrund von ökonomischen Entwicklungen müssen Bibliotheken ihr Etat effizient und bedarfsgerecht einsetzen. 
Zudem werden Etatverhandlungen in Bibliotheken immer wichtiger. 
Das Ziel der vorliegenden Masterarbeit war es, ein Proof-of-Concept eines datengetriebenen Unterstützungssystems zur
Etatplanung- und Mittelallokation für die Bibliothek des Max-Planck-Institutes für empirische Ästhetik zu konzipieren und zu entwickeln.
Dafür wurden aus verschiedenen bibliothekarischen Bereichen Daten analysiert 
und ausgewertet. Das datengetriebene Unterstützungssystem ermöglicht wesentliche Key Performance Indicators wie Budget, 
Umsatz, Ausleihe, Bestandsentwicklung sowie die Lesesaalnutzung in einem Dashboard anschaulich anzeigen zu lassen. 
Damit kann die Bibliothek ihre Planung des Etats und den Einsatz der Mittelallokation effizienter und bedarfsgerechter gestalten
sowie Verhandlungen über den Etat sicher führen.

  

\begin{center}
  \textsc{Abstract}
  \end{center}

  \noindent
  %
  Due to economic developments, libraries must use their budgets efficiently and in line with demand. 
  In addition, budget negotiations in libraries are becoming more and more important. 
  The goal of this master thesis was to develop a proof-of-concept of a data-driven support system for
  budget planning and resource allocation for the library of the Max Planck Institute for Empirical Aesthetics.
  For this purpose, data from different library areas were analyzed and evaluated. The data-driven support system 
  displays key performance indicators such as budget, expenditures, circulation, collection development, and reading room usage in a dashboard. 
  This allows the library to plan its budget and allocate funds more efficiently and in line with its needs
  as well as conduct budget negotiations with confidence.
  %\lipsum[1]
