\begin{center}
    \textsc{Abstract}
  \end{center}
  %
  \noindent
  %
  Due to economic developments, libraries have to use their budgets efficiently and in line with its needs. 
  In addition, negotiations about the budgets in libraries are becoming increasingly important. 
  The goal of this master's thesis, was to design and develop a proof-of-concept of a data-driven support system 
  that enables the library of the Max Planck Institute for Empirical Aesthetics to use and plan its budget more efficiently.
  For this purpose, data from different library areas were analyzed and evaluated. 
  The data-driven support system shows essential key performance indicators such as budget, 
  expenditures, circulation, collection development, and reading room usage in a single dashboard. 
  This lays the foundation for the library to plan its budget and allocate resources more efficiently and in line with its needs.

  

\begin{center}
    \textsc{Zusammenfassung}
  \end{center}

  \noindent
  %
Aufgrund von ökonomischen Entwicklungen müssen Bibliotheken ihre Budgets effizient und bedarfsgerecht einsetzen. 
Zudem werden Verhandlungen über die Budgets in Bibliotheken immer wichtiger. 
Das Ziel der vorliegenden Masterarbeit, war es ein Proof-of-Concept eines datengetriebenen Unterstützungssystems 
zu konzipieren und zu entwickeln, dass die Bibliothek des Max-Planck-Institutes für empirische Ästhetik in die Lage versetzt,
ihr Budget effizienter zu planen und einzusetzten. Dafür wurden aus verschiedenen bibliothekarischen Bereichen Daten analysiert 
und ausgewertet. Das datengetriebene Unterstützungssystem ermöglicht wesentliche Key Performance Indicators wie Budget, 
Umsatz, Ausleihe, Bestandsentwicklung sowie die Lesesaalnutzung in einem Dashboard anzuzeigen. 
Damit ist ein Grundstein für die Bibliothek gelegt, die Planung des Budgets und der Einsatz Mittelallokation effizienter
und bedarfsgerechter zu gestalten.
  %\lipsum[1]
