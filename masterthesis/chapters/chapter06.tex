
\chapter{Schluss}
\label{chap:six}
%\section{Soll-Ist-Vergleich (Stand der Umsetzung)}
\section{Lessons learned}
\section{Welche Themen wurden nicht bearbeitet}
Counter-Statistiken, Anwendungsfall 2, 7

\section{Welche Themen sind im Anschluss denkbar}
Zusammenführen der Daten Ausleihe mit Bestand -> ein großes dataset
bash script für den import der Dateien
Daten aus weiteren Datenquellen
Integration von Counterstatistiken
Integration von RVK-Daten nicht auf der Ebene der Systematikstelle , sondern auf der Ebene
der Hauptgruppen -> da Bennennung auf Ebene der Systematikstelle zum Teil zu generisch ist. 
-> sich darein denken erfordert nochmal ein bisschen mehr Zeit.
Datenbank-Anbindung
Es wurden als Datengrundlage weiterhin nur die Daten berücksichtigt, 
    die für die Anwendungsfälle benötigt werden. Außen vor blieb zum Beispiel die Integration der \acrshort{COP 5}-Statistiken.

Refactoring -> wenn System produktiv gehen soll.
Zerklopfen der Module -> data\_prep falls noch mehr Methoden dazukommen  riesige Datei Auslagerung der Child classes in eigene Dateien
-> flat is always better
Wurde nicht alles im OO-Style umgesetz

Die Dateien zur Datenanreicherung müssen manuell gepflegt werden -> automatischer Prozess -> könnte noch ein bisschen vereinfacht werden,
programmiertechnisch

Proportionen der Diagramme mehr beachten, um visuell besser unterscheiden zu können -> Plotly-Vorgaben, Überarbeitung des Layouts,

Python Dateien besser aufteilen -> Data prep eine Datei mit bis zu knapp 1000 Zeilen Code
    Sollte Codebase noch wachsen -> in einzelne dateien - > habe ich aber trotzdem erstmal so gelassen
    keine eindeutige Regel -> sondern Konvention für und wieder, Viele Module der Python StandardBibliothek
    haben viele Klassen in einer Datei -> it's easier to Import