
\chapter{Schluss}
\label{chap:six}
%\section{Soll-Ist-Vergleich (Stand der Umsetzung)}

% \section{Welche Themen wurden nicht bearbeitet}
Anwendungsfall 2, 7
Buchservice-Daten (Artikel...)


% \section{Welche Themen sind im Anschluss denkbar}
Mit dem Proof-of-Concept Versprechen eingelöst das Programmierung eines Dashboards gelingt.

Counter-Statistiken
Es wurden als Datengrundlage weiterhin nur die Daten berücksichtigt, 
die für die Anwendungsfälle benötigt werden. Außen vor blieb zum Beispiel die Integration der \acrshort{COP 5}-Statistiken.
Zugriff auf elektronische Ressourcen Schon auch wichtiger Bereich innerhalb der MPG
Fehlermeldungen loging des erfolgreichen
\\
Auswertung Zeitschriften, aus denen die Artikel sind -> Aufschluß darüber, ob es sich lohnt Zeitschriften anzuschaffen.
Zusammenführen der Daten Ausleihe mit Bestand -> ein großes dataset
\\
%bash script für den import der Dateien
 RVK -> sprechende Benennungen einführen
 Datenbank-Anbindung
\\
Refactoring -> wenn System produktiv gehen soll.
Zerklopfen der Module -> data\_prep falls noch mehr Methoden dazukommen  riesige Datei Auslagerung der Child classes in eigene Dateien
-> flat is always better
Wurde nicht alles im OO-Style umgesetz
Python Dateien besser aufteilen -> Data prep eine Datei mit bis zu knapp 1000 Zeilen Code
    Sollte Codebase noch wachsen -> in einzelne dateien - > habe ich aber trotzdem erstmal so gelassen
    keine eindeutige Regel -> sondern Konvention für und wieder, Viele Module der Python StandardBibliothek
    haben viele Klassen in einer Datei -> it's easier to Import

Reducing complexity \cite[Vgl.][]{ousterhout_philosophy_2018}
\\
-> sich darein denken erfordert nochmal ein bisschen mehr Zeit.
Die Dateien zur Datenanreicherung müssen manuell gepflegt werden -> automatischer Prozess -> könnte noch ein bisschen vereinfacht werden,
programmiertechnisch

klarere Trennung zwischen den Teilsystemen, insbesondere die Datenanreicherung angeht

Proportionen der Diagramme mehr beachten, um visuell besser unterscheiden zu können -> Plotly-Vorgaben, Überarbeitung des Layouts,
weitere Möglichkeiten der statistischen Auswertung in Betracht ziehen. Korrelationen berechnen zwischen der Bestandsgröße und der Ausleihe
Anzeigen von Entwicklungen durch Trendlinien 

Das Dashboard-Layout überarbeiten. -> Responsive Design hinzufügen
Anordnung der Diagramme sowie das Anzeigen der Cards verbessern.


% \section{Lessons learned}
Viel Zeit für Datenanalyse draufging
war viel Daten aus allen möglichen Bereichen -> hoher Aufwand
relativ einfach Datenvisualisierungen zu erzeugen mit den Frameworks
gute Zeitplanung alles ist
Viele Dinge zu berücksichtigen gilt -> Frontend Gestaltung des Dashboardes -> ausbaufähig
Python und pandas mächtig und flexibel ->  bieten  viele Möglichkeiten an, die es manchmal 
dann ein bisschen zu verwirrend macht. Viele neue Dinge gelernt, während des Programmierens
Lernkurve merkt man
Zeiteinteilung -> bessere Aufwandseinschätzung weiß wie aufwendig die Sachen, gerade wenn man festestellt,
dass ein Fehler behoben werden muss, und anwelchen
Dass Fehler, die man Anfang macht -> dass die manchmal später zurück kommen -> Teilsysteme nicht so richtig trennen


