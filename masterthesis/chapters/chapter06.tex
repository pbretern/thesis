
\chapter{Fazit und Ausblick}
\label{chap:six}
Das Ziel, ein System für die Etatplanung und Mittelallokation zu entwickeln, an dem die
Bibliotheksleitung und die Mitarbeiter:innen relevante \textit{\acrlong{KPI}} wie Umsatz- und Budgetübersichten, 
Ausleihzahlen und Bestandsinformationen ablesen können, ist durch das vorliegende Proof-of-Concept umgesetzt wurden. Das System aggregiert relevante Daten, die statistisch mit geeigneten und
modernen Datenvisualisierungen analysiert werden.

Durch die ansprechende graphische Darstellung der Daten in Form eines Dashboards können die gewonnenen Informationen schnell erfasst werden.
Somit eignet sich das\\ Dashboard sehr gut als Instrument für das Monitoring und die Steuerung von wesentlichen Bibliotheksprozessen 
und kann darüber hinaus für die Kommunikation und Präsentation der Daten gegenüber Stakeholdern zum Beispiel in Etatverhandlungen gewinnbringend 
eingesetzt werden. Die Darstellung der Daten im Dashboard basiert auf umfassender und gewissenhafter Datenerhebung, 
sie ist funktional, indem eine große Anzahl von Datensätzen kompakt dargestellt wird, und sie ist 
aufschlussreich und aufklärend, indem Zusammenhänge und Entwicklungen ersichtlich werden. 
Eine attraktive und zeitgemäße Aufbereitung der Daten wurde angestrebt und umgesetzt. 
Die visuelle Präsentation der Daten beinhaltet deshalb alle wichtigen Eigenschaften von Datenvisualisierungen, 
wie sie in \autoref{chap:two_two} beschrieben wurden.

Das datengetriebene Unterstützungssystem, das in dieser Arbeit konzeptionell entworfen wurde, erfüllt drei Hauptaufgaben:
den Import von bibliothekarischen Daten aus heterogenen Datenquellen in ein einheitliches Dateiformat, die Auswertung der Daten mit deskriptiven Methoden der Statistik
wie die Darstellung der Lagemaße oder mit Visualisierungen der Daten in einem Dashboard.
Dieses System funktioniert für die vorliegenden bibliothekarischen Daten gut. 

In seiner Funktionalität und seiner Architektur ist das 
System an \textit{\acrlong{BI}} angelehnt, wie sie in  \autoref{chap:two_three} beschrieben wurden. 
Das System enthält alle wesentlichen Schichten eines \textit{\acrshort{BI}}-Systems. So werden Daten aus externen Datenquellen und
operativen Anwendungssystemen mithilfe von \textit{\acrshort{ETL}}-Prozessen integriert und aufbereitet.
Anschließend werden sie in einer Organisationsform, die einem Data Lake ähnelt, gespeichert. 
In einem analytischen und taktischen Dashboard erfolgt eine Bewertung der Daten auf mehreren Detailebenen. 
Bei dem vorliegenden System handelt es sich um einen integrierten, unternehmensspezifischen Ansatz, 
der betriebliche Entscheidungen IT-basiert unterstützt. Insofern kann es als \textit{\acrlong{BI}} begriffen werden.

Für die Beschaffung der bibliothekarischen Daten gibt es in der Bibliothek bereits Prozesse, auf die bei der Umsetzung des Proof-of-Concepts 
zurückgegriffen werden konnte. So werden die Umsatz- und Budgetdaten seit 2018 regelmäßig jeden Monat vom \textit{\acrshort{LBS}}-Team geliefert. Die benötigten Daten für die 
Jahre 2014 bis 2017 konnten - in einem anderen Format - unproblematisch vom \textit{\acrshort{LBS}}-Team nachgeliefert werden. Um die Daten für den Import passend zu machen, 
musste hier manuell trotzdem nochmal nachgebessert werden. Problematisch war die Datenbasis für die monatlichen Neuerwerbungen. Die Neuerwerbungsdaten wurden 
in der Vergangenheit nur sehr unregelmäßig aus dem \textit{\acrshort{CBS}} abgezogen, so dass dieser Datenbestand viele Lücken aufwies. Leider war es in Verbindung mit den Neuerwerbungen 
ebenfalls nicht möglich, einen gesamten Abzug des Bestandes für das datengetriebene Unterstützungssystem retrospektiv zu nutzen, da die wichtige Datuminformation für die 
Neuerwerbungen in dem Abzug fehlte. So wurden die Daten für den Zeitraum ab 2014 nochmals für jeden Monat aus dem \textit{\acrshort{CBS}} abgezogen. 

Vorkenntnisse der Daten existierten bereits aus früheren Auswertungen mit einem Tabellenkalkulationsprogramm, sodass eine solide Wissensbasis 
der einzelnen Daten bereits bestand. Dennoch wurde der Aufwand des gründlichen Sichtens der Daten unterschätzt, sodass bei 
der Implementation des Systems und der Arbeit mit den Daten Probleme auftraten. 
Problematisch war dies insbesondere bei den Ausleihdaten. Durch die Validierung der RFID-Etiketten in
den Medien am Selbstverbucher durch die Bibliothek wird eine größere Anzahl an Ausleihen erzeugt als eigentlich ausgeliehen wird. 
Dagegen wurde mit einer Programmmethode beim Import der Ausleihrohdaten vorgegangen, die unabhängig von den Jahrensangaben in den Daten die Ausleihanzahl der Medien um eins reduzierte. 
So kann es sein, dass ein Medium in fünf Jahren jeweils einmal ausgeliehen wurde, in den betreffenden Diagrammen aber überhaupt nicht erscheint. 
Hier gilt es die Methode im Programmcode zu erweitern, sodass die Jahresangabe in den Daten mit berücksichtigt werden kann. Darüberhinaus ist die konkrete bibliothekarische Praxis der Medienerschließung zu überdenken.
Wegen solcher Probleme sollte der zeitliche Aufwand der Voranalyse der Daten in Hinsicht auf die Integration neuer
Daten in das System nicht unterschätzt werden und ihr einen größtmöglichen Platz eingeräumt werden. 

Das bibliothekarischen Domänwissen in Bezug auf das Datenformat der Titel- und Lokaldaten konnte
für die Bestands- und Neuerwerbungsdaten zielführend eingesetzt werden. Unbekannte Statistiken wie die \acrlong{COP 5} mussten aufgrund des engen Projektzeitplans leider unberührt bleiben.
Eine erste Überlegung, nur mit Daten zu arbeiten, die schon in dem für das System bevorzugten CSV-Format vorlagen, 
zerschlug sich an der Heterogenität der Datenmodelle. Auch waren Versuche, von vornherein die Daten in dem CSV-Format abzubilden, nicht zufriedenstellend.
Für die Transformation wurde daraufhin das Teilsystem 1 entwickelt.
Aufgrund des Rückgriffs auf die pandas Bibliothek, die viele Funktionen für die verschiedenen Dateiformate anbietet, konnte der Import aber sichergestellt werden.
% bibliothekarisches Domainwissen

Aus der Voranalyse der Daten und aus früheren Datenauswertungen wurden dann die Anforderungen und die Anwendungsfälle abgeleitet.
Diese wurden vor dem Beginn der Systemprogrammierung formuliert. In einem iterativen Entwicklungprozess wurden die Anforderungen und Anwendungsfälle kontinuierlich re-evaluiert und 
entsprechend angepasst. Leider war dies aufgrund des engen Zeitrahmens des Projekts nicht immer in vollem Umfang möglich. So wurden die Teilsysteme nacheinander entwickelt und
erst am Ende dieses Entwicklungsprozesses wurde das erwünschte Ergebnis in Form des Dash-boards schließlich sichtbar. Vermutlich wäre es für den Verlauf der Systementwicklung sowie einer
Re-Evaluierung der Anforderungen und Anwendungsfälle besser gewesen, das System zunächst lediglich mit den Daten aus nur einem der vier bibliothekarischen Bereiche
aufzubauen. Danach hätte das System für die Daten aus den anderen Bereichen sukzessive erweitert werden können. 
So hätte man wahrscheinlich eine größere Chance gehabt, Fehler, die während der Systemprogrammierung entstanden und gelöst werden mussten, zu vermeiden.

% automatische Prozesse gegenüber manuelle Prozesse

% an unbekannte Daten wie Cop5 Statistiken nicht rangegangen -> da hier der zeitliche Aufwand zu groß gewesen wäre (Wie sind die Daten aufgebaut, was beschreiben sie,
% wie vollständig sind sie)
% -> nicht machbar in der Zeit, obwohl der Bibliotheken in der MPG Bedarfe gibt -> Anschlussprojekt

% dem Ermitteln der Anforderungen;
% Zusammentreffen des Entwicklers und Bibliothekars in einer Person
% bibliothekarischer Backgrounw´d wichtig beim Erkennen der Daten -> hat Einfluss auf die Entwicklung
% Fehlertoleranz -> welche Daten für die Auswertung können vernachlässigt werden,  wie wichtig ist das mitnehmen aller daten -> hat Auswirkungen auf
% Translation zwischen den statistischen Auswertungen die die Bibliothek schon gemacht und denen die automatisch erfolgen sollen.
% die Anforderungen, Translaionsprozess zwischen Anforderungen und Entwicklung zu minimieren, d.h. aber auch konkrete Anforderungen schreiben
% iterative Prozess hat aufgrund des engen zeitplans nicht so gut geklappt.
% Da die Teilsysteme nacheinander entwickelt wurden, hat es einer etwas größeren Vorstellungskraft gebraucht, um sich zum Beispiel mit den
% Diagrammen vorzustellen, erst am Schluss der Entwiicklung etwas zu sehen war
% Eigentlich wäre jetzt nochmal der Zeitpunkt gekommen, die Anforderungen und die Anwendungsfälle anzupassen und

% dem Umgang mit den verschiedenen Arten von Heterogenität [1] usw.)

% Es konnte mit der vorliegenden Arbeit demonstriert werden, dass Daten aus heterogenen Datenquellen in einem Dashboard dargestellt werden können.
% Dabei wurde ein System entwickelt, dass vom Datenimport über die Datenvorbereitung 
% bis hin zur Darstellung der vorhandenen Daten gut funktioniert. 
Eine Weiterentwicklung des Systems ist denkbar und wünschenswert. 
Neben der noch ausstehenden Umsetzung der Anwendungsfälle 2 und 7 betrifft die Weiterentwicklung alle Teilsysteme der Systemarchitektur
sowie deren technische Implementation. Vorstellbar wäre eine striktere Trennung der Teilsysteme. 
So könnte die Datenanreicherung ausschließlich im Teilsystem 1 erfolgen. Zur Zeit geschieht diese im Teilsystem 1 und im Teilsystem 2. 
Ebenso wäre es denkbar, die Implementation des Dashboards und die Erstellung der Diagramme im Teilsystem 3 zu trennen und diese in jeweils einzelne Teilsysteme
aufzulösen. Dadurch würde die Wartbarkeit des dahinterliegenden Programmcodes erhöht und der Programmcode könnte modularer strukturiert werden.
Eine andere Möglichkeit in diesem Zusammenhang wäre, das Teilsystem 3 durch Lösungen wie \textit{Tableau} oder \textit{apache superset} zu ersetzen.
Diese erstellen Dashboard-Applikationen mit einem Baukastensystem und können die Daten in verschiedenen Dateiformaten erwarten.\footnote{ Inwieweit die Ersetzung des Teilsystems 3 in die anderen Teilsysteme eingreifen würde, kann
hier nicht diskutiert werden.}

Um die Aktualität der Daten und der Darstellung garantieren zu können, könnte der Import der bibliothekarischen Daten automatisch erfolgen. So könnte auf das manuelle
Anstoßen des Imports verzichtet werden. Stattdessen würden die Daten mit einem CronJob importiert werden. Dabei wäre es wichtig, die Ergebnisse des Import-Prozesses in einer log-Datei zu protokollieren. 
Diese Ergebnisse können Meldungen des Systems über die Anzahl der zu importierenden Datensätze, über den erfolgreichen Import oder auch Fehlermeldungen sein.
Für das Protokollieren würde sich die Python Standardbibliothek Logging anbieten \cite[vgl.][]{sajip_logging_2021}.

Eine vorstellbare Weiterentwicklung wäre die Integration zusätzlicher Daten in das System wie die der Dokumentenlieferdienste oder die Nutzungsdaten elektronischer Ressourcen wie Online-Zeitschriften. 
Sowohl die Dokumentenlieferdienste als auch die Bereitstellung der elektronischen Ressourcen sind wichtige Informationsdienstleistungen der Bibliothek. 
Deren Darstellung würde den Informationsmehrwert des Dashboards erhöhen.
% Bei der Darstellung der Daten in den Diagrammen sollte nachgebessert werden. So zum Beispiel könnten Spitzen in den Nutzungszahlen des Lesesaals herausgerechnet oder die
% Servicegruppen-Zeiten in den Lesesaal-Diagrammen in die richtigen Proportionen gesetzt werden.


Zudem wäre es möglich, genauere oder zusätzliche Analysen - wie das Bereitstellen zusätzlicher Diagramme - auf den bereits vorhandenen Daten zu vollziehen, um deren Aussagekraft zu erhöhen.
Weitere Datenanalysen könnten nach zusätzlichen Aspekten oder durch fortgeschrittene statistische Methoden erfolgen. Als Beispiel wäre die Auswertung der Bestandsdaten nach den \textit{\acrshort{RVK}}-Benennungen
zu nennen. Auch der Zusammenhang zwischen Bestand und Ausleihe könnte in einer Datenvisualisierung zum Ausdruck gebracht werden.
Um den Entwicklungsverlauf der dargestellten Daten hervorzuheben, könnten Trendlinien zu den Diagrammen hinzugefügt werden. Zusätzlich könnten Prognosen
für die Zukunft aus den vorhandenen Daten berechnet und dargestellt werden. Diese Weiterentwicklungen wären insbesondere für die Umsatz- und Budgetdaten interessant.

% Die einzelnen Diagramme könnten ebenfalls überarbeitet werden. Die Größe der Diagramme müsste angepasst werden, um die Proportionen den angezeigten Zahlenbereichen anzugleichen. 
% Die Ausnahmebehandlung bei der Datenbearbeitung war eine Schwierigkeit, die im Laufe der Systementwicklung öfters aufkam. Diese Ausnahmen
% entstanden 



%Das Layout des Dashboards könnte ebenfalls überarbeitet werden. 
%Es müssten die Größe der Diagramme angepasst werden, um die Proportionen den angezeigten Zahlenbereichen anzugleichen.
Ferner wurden im datengetriebenen Unterstützungssystem die mitgelieferten Funktionalitäten von Plotly Express nicht zur Gänze ausgeschöpft.
Hier bedarf es noch diverser Feineinstellungen für das Layoutverhalten der Diagramme.
Diese Anpassungen beziehen sich zum Beispiel auf die Legenden oder die Hover-Informationen.
Denkbar und sicherlich sinnvoll wäre zudem eine Anpassung der Diagramm-Proportionen; gegebenenfalls mit einer entsprechenden Anpassung an die von ihnen darstellten Wertebereiche. 

%Ferner könnten weitere Interaktionen implementiert werden. Die Dash\_core\_components bieten noch andere interaktive Elemente wie zum Beispiel Slider für zum Beispiele zeitliche Einschränkungen. 

Ferner sollte der Programmcode in allen Teilsystemen überarbeitet werden. Die Überarbeitung wäre insbesondere notwendig für das \texttt{data\_prep}-Modul, 
das in einer Datei die Basis- und den Kindklassen enthält.\footnote{ Es ist in Python durchaus üblich, dass viele Klassen in einer Datei enthalten sind. 
So bestehen viele Module der Python Standardbibliothek aus einer Datei mit vielen Klassen.}
Spätestens bei der Hinzunahme zusätzlicher bibliothekarischer Daten für die Analyse und Darstellung sollte über eine Aufteilung der Klassen in weitere Dateien nachgedacht werden. 
Es wird dabei davon ausgegangen, dass der Code weiter wachsen würde. Die Erweiterung des Codes würde ebenso das Modul \texttt{data\_import} betreffen. 

Nichtsdestotrotz sollte der Programmcode aller Teilsysteme nach dem \acrfull{DRY}-Prinzip vereinfacht und optimiert werden.
Ebenso sollten die \textit{\acrshort{PEP8}}-Richtlinien stärker umgesetzt werden.
Die Vereinfachung und Optimierung des Codes betreffen insbesondere die Implementation der Diagrammlogik im Teilsystem 3. Da die Diagramm-Funktionen in den einzelnen Dashboard-Tab-Dateien mit ähnlichen Parametern arbeiten, 
ließen sich hier Funktionen für die einzelnen Diagrammtypen entwickeln. Ferner wäre es wünschenswert, wenn das Diagrammlayout (Hintergrundfarbe, Position des Diagrammtitels) 
von dem Erstellen der Diagramme separiert und zentral festgelegt werden könnte. 
Generell muss zwischen Lesbarkeit, kondensierter Komplexität und explizitem Code abgewogen werden. %\cite[vgl.][]{ousterhout_philosophy_2018}
Der Programmcode sollte überdies systematisch getestet werden, um ihn robuster und weniger fehleranfällig zu machen. 
Dies kann mit der Pythonbibliothek pytest \cite[vgl.][]{krekel_pytest_2021} geschehen. Für das systematische Testen blieb leider keine Zeit während der Bearbeitung.


Die grundsätzliche Idee, jedes der drei Teilsysteme isoliert zu bearbeiten und diese dann miteinander zu verknüpfen, 
konnte nur zufriedenstellend umgesetzt werden. So wurden vereinzelt Probleme eines Teilsystem durch die anderen Teilsysteme erzeugt. 
Im Teilsystem 3 traten Probleme auf, die durch das Teilsystem 2 verursacht wurden, das auf Grundlage der 
vom Teilsystem 1 importierten Daten falsche Filterungen vornahm. So konnten zum Beispiel die gefilterten Daten 
in den Diagrammen nicht richtig mit Monatslabeln dargestellt werden. Zur Lösung dieses Problems bedurfte es einerseits einer 
anderen Kodierung der Dateinamen und andererseits eines veränderten Filtermechanismus. Dieses Problem fiel aber erst bei der 
Implementierung des Teilsystems 3 auf. 

Auch die statistische Datenanalyse bereitete dahingehend Probleme, dass mit Ausnahmen in den Daten umgegangen werden musste, die erst spät im Bearbeitungsprozess aufgefallen sind. 
So zum Beispiel sind doppelt vorhandene Datensätze in den Neuerwerbungsdaten erst bei der Umsetzung des Teilsystems 3
aufgefallen. Da das Teilsystem 1 keinen Mechanismus für die Bereinigung dieser doppelten Datensätze vorsieht, können diese erst mit dem Teilsystem 2 entfernt werden.
Eine gründlichere Datenbereinigung und -analyse wäre deswegen im Vorfeld wünschenswert gewesen, war aber aufgrund des engen Zeitplans nicht möglich.


% Im Allgemeinen lässt sich sagen, dass es im Projekt eine hohe Übersetzungsleistung zwischen den Anforderungen beziehungsweise Anwendungen und der praktischen Umsetzung gegeben hat.
% Das war zum Teil sehr schwierig, da sich \dots

% Aufjdenfall sollten die Entwickler eine Basiskentnisse von Statistik und Mathematik besitzen.


% Bei der Darstellung der Daten in den Diagrammen sollte auch noch einmal nachgebessert werden. So zum Beispiel könnten Spitzen in den Nutzungszahlen des Lesesaals herausgerechnet oder die
% Servicegruppen-Zeiten in den Diagrammen in die richtigen Proportionen gesetzt werden.
Weitergehende statistische Kenntnisse konnten im Bearbeitungszeitraum nicht angeeignet werden. Deswegen verbergen sich hinter den Datenvisualisierungen einfache Berechnungen und Filterungen der Daten.
%Wegen der Korrektheit der Darstellung der Daten sollte hier auch noch einmal nachgebessert werden.

Trotz der fehlenden Umsetzung zweier Anwendungsfälle wurde die Zeit zur Bearbeitung des Projektes im Großen und Ganzen vernünftig eingeteilt, 
sodass der Zeitplan im Laufe der Bearbeitung wenig korrigiert werden musste. Aufgrund des engen Zeitplans und der zu erledigenden Aufgabenfülle
blieb aber fast keine Zeit, alternative Lösungsmöglichkeiten in Betracht zu ziehen. 
Das war insbesondere beim praktischen Teil der Arbeit der Fall. Gleichfalls war es das erste große Projekt mit Python und den Bibliotheken pandas, plotly und Dash.
Deswegen musste sich zunächst auch mit der Programmiersprache und den Bibliotheken am Anfang des Programmierprozesses vertraut gemacht werden.
% Python bietet sehr viele alternative Möglichkeiten an Probleme zu lösen und ist durch viele Eigenschaften nicht so streng wie andere Programmiersprachen. 
Mit mehr Vorwissen und Erfahrung hätte sicherlich etliches sauberer, fehlerfreier und eleganter programmiert werden können. 
Trotzdem war es ein sehr lehrreicher Prozess, Python und insbesondere die Bibliothek pandas für das Projekt anwenden zu können. 
Durch die Arbeit wurde somit ein größeres Verständnis von Python und pandas erzielt, das als sehr bereichernd empfunden wird.


Ungeachtet der Probleme und der potentiellen Weiterentwicklungen des datengetriebenen Unterstützungssystems ist das System für die Nutzung durch die Bibliotheksleitung und der Bibliotheksmitarbeiter:innen des \textit{\acrlong{MPI EA}} 
geeignet und kann für die Etatplanung und Mittelallokation eingesetzt werden.