\chapter{Diskussion der Umsetzung}
\label{chap:five}
\section{Design}
    \subsection{Systemarchitektur}
    Plotly Express\\
    Pandas\\
    Dash\\
    -> effizient und effektiv zu sein


Die Klassen im Import Module sind auf den Datenbestand aus fremden Quellen zugeschnitten.
(Budget Umsatz, Neuerwerbungslisten (Anwendungsfall 4 - 6))
Ziel: Die Daten ohne Verluste zu importieren. Dabei spielt für die spätere Anwendung im System, der mtl. Stand
ausgedrückt im Datum eine wichtige Rolle.
Es wurden als Datengrundlage nur die Daten berücksichtigt, die für die Anwendungsfälle benötigt werden.
Außen vor blieb zum Beispiel die Integration der Counter-Daten.
Die von der Bibliothek erhobenen Daten

Einbindung der RVK-Systematik - gewisse Anpassung durch die Bibliothek -> Hinzunahme von eigenen Systematikstellen,
die zum Teil andere Systematikstellen ersetzten (NT eigentlich Rechtswissenschaften, in der Institutsbibliothek Noten, MT eigentlich
Gesundheitswissenschaften, in der Institutsbibliothek Methoden)

    \begin{lstlisting}[language=Python, caption=Python example]
        import pandas as pd
        import plotly.express as px
        import matplotlib.pyplot as plt

        
    \end{lstlisting}

    \subsection{Teilsysteme}

\section{Implementierung}
    \subsection{Umgesetzte Anforderungen}
    \subsection{Funktionsweise}

\section{Bewertung}