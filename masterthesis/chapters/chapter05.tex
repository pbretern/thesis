\chapter{Diskussion der Umsetzung}
\label{chap:five}
Im folgendem Kapitel wird das Design des Prototypen besprochen und die Teilsysteme vorgestellt und skizziert wie die Teilsysteme
funktionieren. Dabei wird exemplarisch anhand von einzelnen Methoden der Klassen oder Funktionen beschrieben wie das funktioniert
Im Quellcode selber wird ausführlich jede Methode beschrieben.
\section{Design}
    \subsection{Technische Details der Implementierung}
    Das Proof-of-Concepts wurde mittels der Programmiersprache Python in der Versionsnummer 3.7.9 umgesetzt.
    Python ist eine höhere Programmiersprache. Sie ist weitverbreitet \cite[vgl.][]{loukides_where_2021} und besitzt
    eine einfache Syntax. Besonderheiten von Python sind der Verzicht auf geschweifte Klammern und Punktionen nach Anweisungen.
    Die Anweisungen sind dagegen durch Einrückungen strukturiert und nicht durch öffnende und schließende Klammern
    getrennt. Des Weiteren zeichnet sich Python durch eine dynamische Typisierung aus und ist dadurch für Skripte als auch 
    für die schnelle Entwicklung von Anwendungen geeignet. Python erlaubt die Aufteilung von Programmen in Modulen, die in anderen Python-Programmen wiederverwendet werden können
    \cite[vgl.][]{python_6_2021}.
    Python ist für alle wichtigen Plattformen frei verfügbar und besitzt eine große Dokumentation \cite[vgl.][]{python_welcome_2021}.
    Wie andere Programmiersprachen besitz Python eine umfangreiche Standardbibliothek. 
    Zusätzlich dazu gibt es noch eine Vielzahl von Modulen, Programmen und Werkzeugen von Drittanbietern \cite[vgl.][]{python_pypi_2021}.
    
    
    Genutzt wurde für das Projekt insbesondere die Python-Bibliothek pandas, die eine Open-Source-Bibliothek für Datenanalyse und 
    Datenmanipulation darstellt \cite[vgl.][]{pandas_pandas_2021}. 
    Wichtige Konzepte dieser Bibliothek sind der pandas Dataframes und die pandas Series, die pandas Objekte sind. 
    Der pandas Dataframe ist eine zweidimensionale tabellarische Datenstruktur mit beschrifteten Achsen (Spalten und Reihen).
    Eine pandas Series ist ein eindimensionales Array. \autoref{fig:pandas Dataframe und pandas Series} zeigt anhand der Umsatzdaten die ersten fünf Reihen
    eines pandas Dataframe und eine pandas Series desselben Dataframes. Zugriff auf die Series erfolgt über
    den Spaltenkopf 'Lieferant Abk.' der als Schlüssel zur Series dient. Sowohl der Dataframe als auch die Series haben einen Index, 
    über den der Zugriff auf die einzelnen Werte funktioniert. 
    
    
    \begin{figure}[h]
        \centering
            \includegraphics[width=7.5cm, height=3.0cm]{dataframe_example}
            \hspace{1cm}
            \includegraphics[width=5.0cm, height=3.0cm]{series_example}
            \caption{pandas Dataframe und pandas Series}
            \label{fig:pandas Dataframe und pandas Series}
    \end{figure}
    
    Auf dem Dataframe, der als sogenannter Container für die Series dient, können verschiedene Operationen der Datenanalyse und 
    -manipulation erfolgen. Ähnlich wie bei SQL-Abfragen einer Datenbank lassen sich so verschiedene Funktionen wie \texttt{sort()}- oder \texttt{groupby()}-Funktionen in Zusammenhang
    mit Aggregatfunktionen wie \texttt{mean()}, \texttt{median()} oder \texttt{sum()} auf den Series durchführen.
    Auch gibt es eine Vielzahl von Funktionen, um die Daten rasch zu explorieren. Die Werte der Zellen können verschiedene Datentypen
    annehmen. Zum Beispiel können die Datumswerte der Spalte 'Datum' in \texttt{datetimeformat-Objekte} umgewandelt werden.

    In Verbindung mit pandas ist Python neben \textsf{R} und Matlab im wissenschaftlichen Kontext für Data-Science-Projekte sehr beliebt.

    
    Für die Erstellung der Datenvisualisierungen wurde hauptsächlich die Bibliothek Plotly Express genutzt. Plotly Express ist eine Weiterentwicklung von der 
    Bibliothek Plotly Graphic Objects, welche eine einfachere Syntax besitzt bei fast annäherender Feature-Gleichheit. \footnote{Manko von Plotly Express ist es, dass es noch 
    nicht mit multi-indexed pandas Dataframes umgehen kann, was die maßgerechte Manipulation der Dataframes für die Diagramme komplexer werden lässt.}
    Mit Plotly Express lassen sich rasch und mit wenigen Zeilen Python Code Diagramme erzeugen, die über eine basic Interactivity verfügen. Das macht es so stark im Gegensatz
    zu dem mächtigen Matplotlib.  
    Das Dashboard wurde mit dem Framework Dash entwickelt, mit welcher die Diagramme von Plotly Express leicht eingebunden werden können. 
    Sowohl Plotly Express als auch Dash werden von derselben Firma entwickelt. Dash baut
    auf Flask, Plotly.js und React.js auf. Das Framework ermöglicht, die Erstellung interaktiver Webapplikationen 
    in reinem Python mit Wissen von HTML und CSS ohne Kenntnisse von Javascript. \autoref{tab:Software-Requirements} zeigt einen 
    kurzen Überblick über die Versionsnummern der genutzten Programmiersprache und der Frameworks sowie deren Open-Source
    Lizenzen.
    
    \begingroup
        \setlength{\tabcolsep}{4pt} % Default value: 6pt
        \renewcommand{\arraystretch}{1.5}
        %\resizebox{\textwidth}{!}{
        \begin{table}[h]
            \centering
            \begin{adjustbox}{max width=\textwidth}
            \Huge
            \begin{tabular}{lccl}
              %\begin{tabular}{p{3cm}p{5cm}p{1cm}p{1.5cm}p{2cm}p{4cm}}
               \toprule
               \textbf{Name}             &{Version}    &\textbf{Lizenz}                        & \textbf{Webseite}\\
               \midrule     
                    Python               &3.7.9         &Open Source (PSF)                     & \url{https://docs.python.org/3.7/}\\
                    pandas               &1.1.2         &3-Clause-BSD-License                  & \url{https://pandas.pydata.org/pandas-docs/version/1.1.2/}\\
                    Plotly Express       &0.4.0         &MIT-License                           & \url{https://plotly.com/python/}\\
                    Dash                 &1.16.3        &MIT-License                           & \url{https://dash.plotly.com/}\\


                \bottomrule
            \end{tabular}
            \end{adjustbox}
            \caption
            \label{tab:Software-Requirements}
            }
             \end{table}
        \endgroup
    
     
    \subsection{Systemarchitektur}
    
    Das System teilt sich in vier Teilsysteme auf. \autoref{tab:Teilsysteme} zeigt die vier Teilsysteme mit einer Kurzbeschreibung der Hauptaufgabe.
    Objekt-orientiert programmiert wurden nur Teile des Systems. Diese Teile sind die Bereiche des Datenimports und der Datenbearbeitung. 
    %Wohingegen die Bereiche der Darstellung mit dem Dashframework nicht mehr objektorientiert werden konnte.
    
       \begingroup
            \setlength{\tabcolsep}{4pt} % Default value: 6pt
            \renewcommand{\arraystretch}{1.5}
            %\resizebox{\textwidth}{!}{
            \begin{table}[h]
                \centering
                \begin{adjustbox}{max width=\textwidth}
                \Huge
                \begin{tabular}{lccl}
                  %\begin{tabular}{p{3cm}p{5cm}p{1cm}p{1.5cm}p{2cm}p{4cm}}
                   \toprule
                   \textbf{Teilsystem}             &{Hauptaufgabe} \\
                   \midrule     
                        Import               &Import und Transformation der Daten aus heterogenen Datenquellen.\\
                        Datenbearbeitung     &Aufbereitung der Daten für die graphische Darstellung im Dashboard.\\
                        Darstellung          &Bereitstellung der Daten und Anzeige der Daten im Dashboard.\\
                        Standardbericht      &Export ausgewählter Datenvisualisierungen und Darstellung in Berichtsform.\\

                    \bottomrule
                \end{tabular}
                \end{adjustbox}
                \caption
                \label{tab:Teilsysteme}
                }
                 \end{table}
            \endgroup
    



    

% Urls zu den als Fußnoten dazufügen.

% Herausfiltern der Datensätze zum Beispiel aus den Neuerwerbungslisten, hinter denen keine physische Entsprechung zum einen steht (mehrteilige Ressouurce auf Gesaamtitel?ebene, Datensätze Schriftenreihen)

    
    
%     grobes Bild der Systemarchitekturmit Klassen -> UML-Diagramm mit Bereichen
    
    
%     Plotly Express\\
%     pandas\\
%     Dash\\
%     -> effizient und effektiv zu sein




%     \begin{lstlisting}[language=Python, caption=Python example]
%         import pandas as pd
%         import plotly.express as px

        
%     \end{lstlisting}

    \subsection{Teilsysteme}
    
    \textit{Import}\\
    Das Teilsystem Import ist verantwortlich für den Import der Daten im Rohformat aus vordefinierten Importverzeichnissen in vordefinierte Zielverzeichnisse. Das Ziel
    ist einerseits die Daten ohne Informationsverlust zu importieren und und andererseits diese mit notwendigen Daten anzureichern. Schließlich werden die Daten
    in ein einheitliches csv-Format umgewandelt und abgespeichert. Mit dem Teilsystem soll eine einheitliche Datengrundlage für die spätere Bearbeitung
    und Darstellung garantiert werden.
    % Die Daten werden jeweils in ein pandas Dataframe geladen. Das Dataframe wird dann verschiedentlich bearbeitet, bevor es zum Schluss als csv-Datei in dem jeweiligen
    % Zielverzeichnis abgespeichert wird.
    Das Teilsystem Import besteht aus vier Klassen. Die \autoref{fig:classes import} zeigt die einzelnen Klassen mit ihren Methoden des 
    Teilsystems. Instantiiert werden die einzelnen Klassen für die jeweiligen konkreten bibliothekarischen Daten. So gibt es für Budget, Umsatz, Ausleihe usw.
    jeweils eine eigene Instanz und die für die Daten angemessenen Methoden der Klassen.
    
    \begin{figure}[H]
        \centering
            \includegraphics[width=14cm, height=2.5cm]{classes_imp}
            \caption{Klassendiagramm - Teilsystem Import}
            \label{fig:classes import}
    \end{figure}

    Die Klassen im Teilsystem Import sind auf den Daten aus fremden Quellen zugeschnitten
    (Budget Umsatz, Neuerwerbungslisten, Ausleihe (Anwendungsfall 2, 4-6)). Dennoch wurden mit ihnen auch die anderen Daten
    wie Lesesaalnutzung bearbeitet. 

    Den Ablauf des Teilsystems zeigt schematisch die \autoref{fig:flow import}.

    \begin{figure}[H]
        \centering
            \includegraphics[width=12cm, height=2.5cm]{flow_imp}
            \caption{Datenfluss - Teilsystem Import}
            \label{fig:flow import}
    \end{figure}

    
    (1) Für den ersten Schritt sind die Klassen \texttt{FileValidation} und \texttt{FileImport} verantwortlich.
    Die Dateien werden aus einem vorher definierten lokalen Verzeichnis in ein pandas Dataframe exportiert. 
    Dabei wird mit Methoden der Klasse \texttt{FileValidation} sichergestellt, dass sowohl das Verzeichnis als auch die Dateien existieren. 
    Des Weiteren wird sichergestellt, dass die Dateinamen einem definierten semantischen Format (z.B. YYYY\_MM\_DD) und 
    einem Format (zum Beispiel csv, xlsx) entsprechen. Da die Daten unterschiedlich aufgebaut und in unterschiedlichen Dateiformaten vorliegen 
    werden beim Import in den Dataframe jeweils verschiedene Methoden angewandt.
    Beim Ladeprozess in den pandas Dataframe wird mit den Methoden \texttt{load\_txt\_to\_df} oder \texttt{load\_tsv\_to\_df} der Klasse \texttt{FileImport} 
    zum Zeitpunkt des Imports der Datei in den Dataframe der Dateiname extrahiert und in eine neue Spalte des Dataframe als String 
    im Datumsformat YYYY-MM-DD gespeichert. Die neu entstandene Spalte ist für die spätere Auswertung und Darstellung der Daten im Dashboard wichtig, 
    da anhand dieser Spalte die Daten nach dem Datum ausgewertet werden können.
    \footnote{Dieses Verfahren wird bei den Daten ausgeführt, die einer zusätzlichen Datumsspalte bedürfen.} 
    Verantwortlich für die Umwandlung des Dateinamens in einen String ist die in \texttt{utils.py} ausgelagerte Funktion \texttt{date\_from\_filename},
    die den Dateinamen als Parameter entgegennimmt.
    \footnote{In der Datei \texttt{utils.py} sind noch andere Funktionen als stand-alone-functions gruppiert, da diese das Objekt \texttt{self} nicht verändern.} 
    
    (2) Das geladene pandas Dataframe wird im zweiten Schritt durch die Klasse \texttt{CleanPreProcDf} aufgenommen und durch verschiedene Methoden dieser Klasse
    manipuliert. Beispielhaft ist hier die Methode \texttt{create\_new\_column\_for\_rvk\_benennung} zu nennen, die aus der Spalte der Signatur 
    der Neuerwerbungen die Hauptklassen der \textit{\acrlong{RVK}} extrahiert. Dieser Methode liegt eine csv-Datei
    der \textit{\acrshort{RVK}} zu Grunde, die mit einem Skript aus der \textit{\acrshort{RVK}}-XML-Datei entstanden ist. Ergänzt wurde diese noch um eigene Hauptklassen 
    der Institutsbibliothek, die an die in der Form an die \textit{\acrshort{RVK}} angelehnt sind. Des Weiteren gibt es in der Klasse Methoden die für die Entfernung von Rows
    mit bestimmten syntaktischen Zeichen wie einem Bindestrich verantwortlich oder die unnötige Leerzeichen Spaltenköpfen löschen und mit einem Leerzeichen ersetzen.
     
    (3) Nachdem Transformationsprozess wird das veränderte pandas Dataframe als csv-Datei in einem vorher definierten Zielordner gespeichert. 
    Verantwortlich ist dabei die Klasse \texttt{SaveDFToCSV} mit den zwei Methoden \texttt{add\_df\_existing\_csv\_file create\_new\_csv\_file\_df}. 
    Die je nach dem mit dem Dataframe eine neue csv-Datei erstellen oder an eine bereits vorhandene csv-Datei den Dataframe anhängen.
    
    \autoref{fig:umsatzuebersicht_csv} zeigt anhand der Umsatz-txt-Dateien die Transformation vom Rohformat in ein einheitliches csv-Format. 
    Im Transformationsprozess wurde neben der Datumsspalte noch eine Spalte für die abgekürzten Lieferantennamen geschaffen. Zur besseren Darstellung
    wird die csv-Datei in einem Tabellenformat angezeigt.

    \begin{figure}[h]
        \centering
            \includegraphics[width=6.5cm, height=7.0cm]{umsatzuebersicht_mtl}
            \includegraphics[width=6.5cm, height=7.0cm]{umsatz_csv_bsp}
            \caption{Monatliche Umsatzübersicht nach Ablauf Teilsystem Import}
            \label{fig:umsatzuebersicht_csv}
    \end{figure}
    

    
    \textit{Datenbearbeitung}\\
    \textit{Darstellung}\\
    \textit{Standardbericht}\\
    
    

\section{Implementierung}

  

    \subsection{Umgesetzte Anforderungen}
    Folgende Must-Anforderungen wurden erfüllt.
    R1, R2, R3, R4, R5, R7
    F1, F4, F5, F12
    NF5, NF6, 
    \subsection{Funktionsweise}
    \textit{Dashboard}
    Das Layout des Dashboards besteht aus drei Tabs. Diese wurden nach den Bereichen der Bibliothek aufgeteilt. Diese Aufteilung erschien als sinnvoll,
    da auf grafische Oberfläche nicht überladen werden sollte.
    Nach dem der Webserver gestartet wurde, landet man im Tab1 und bekommt die dort verankerten Inhalte zu sehen.
    Tabs sind als Reiter oben auf der Webseite angesiedelt.
    Drauf klicken und man gelangt auf den Inhalt desa Tabs (Seite)
    
    Fast allen Diagrammen wohnt inne, dass man aus der Legende auswählen kann welche Elemente angezeigt bzw. nicht angezeigt werden sollen.
    mitunter auch hover elemente, wenn über die einzelnen Balken oder Linien mit der Maus gefahre 
    
    Tab1 - Umsatz und Budget
    
    Bild Tab1
    Was ist dargestellt:
    
    zwei Diagramme mit 
        Anzeige des Gesamtumsatz nach Jahren und nach Lieferanten ->  horizontalestapeltes Balkendiagramm
        umsatzstärkste Lieferanten (Top 7) Verteilung in Prozent (Kreisdiagramm)
        
    zwei Diagramme mit interaktiver Auswahl eines Lieferanten über Dropdownmenü
        Anzeige des Gesamtumsatzes nach Jahren für de ausgewählten Lieferanten (Liniendiagramm)
        Anzeige des Umsatzes pro Monat für das laufende Jahr (Balkendiagramm)
    
    zwei Diagramme mit
        Anzeige des Gesamtbudgets nach Jahren und nach Kostenstellen 
        Anzeige kostenintesive (Top 7) Kostenstellen
        
    
    vier Karten mit
        Zahl des Gesamtumsatzes
        Zahl des Umsatzes im Jahr
        Zahl des Umsatzes im laufenden Jahrs für den einzelnen Lieferanten (gekoppelt mit Dropdown-Auswahl)
        Durchschnittlicher Umsatz des Lieferanten (gekoppelt mit Dropdown-Auswahl)
    
    
    
    
    
    Tab2 - Lesesaal und Ausleihe
    
    Bild Tab2
    
    zwei Diagramme mit
        Anzeige der Nutzer:innen nach Jahren und nach Service-Zeiten (gruppiertes Balkendiagramm)
        Anzeige der Nutzer:innen nach Monaten und nach Service-Zeiten für das laufende Jahr.
        
     ein Diagramm mit
        Anzeige der Ausleihanzahl nach Jahren (horizontales Balkendiagramm)
    
    Tab3 - Bestand
    
    Bild Tab3

    
     



\section{Bewertung}
zur Datenlage 
Ausleihdaten nur auf Zuruf, mtl. Daten bekommen wir erst seit 2021 zu geschickt.
kumulative Daten bei Umsatz/Budget
fehlen der UmsatzDaten für Datenbanken müsste nochmal händisch nachgetragen werden (ca. 20.000 Euro)
Fehlen der Buchservice-Daten (Subito, ...) -> keine Zeit
nur Daten die aus dem CBS/LBS  stammen
fehlen Daten der Online-Zeitschriften

dennoch:
gelungen, die vorhandenen Daten soweit zu importieren, dass mit diesen über das Dataframes Manipulationen und Berechnungen
durchgeführt werden können. Ebenso ist es gelungen eine grafische Oberfläche für die Diagramme anzubieten. Diese Diagramme
sind mit Plotly interaktiv. Es konnte auch ein bisschen interaktivität selber programmiert werden.

Es konnte gezeigt werden, das das gelingt\dots

Den größten Teil der Zeit hat die Datenanalyse gekostet (-> ähnlich Data Science Circle). 
pandas sehr mächtig und aber auch nicht trivial ist. Ziemlich gutes Tool für die Datenanaylyse darstellt.


