\chapter{Diskussion der Umsetzung}
\label{chap:five}
\section{Design}
Enschieden wurde sich für Umsetzung  des Proof-of-Concepts für Python als Programmiersprache.
Python - > weitverbreitet, steht unter der freien Lizenz https://opensource.org/osd
insbesondere im wissenschaftlichen Bereich für dataScience-Projekte ist Python mit den Bibliotheken Pandas, NumPy, Matplotlib
weit verbreitet. 
Plotly, Plotly Express ->
Für die Entwicklung des Dashboards wurde sich für das Framework Dash entschieden, das zusammen mit der Plotly.Express Library für Datenvisualisierungen  Werkzeuge in Python only bereitstellt.  Dash ist Wrapper-Klasse unter dessen Haube javascript, html, ... verarbeitet wird. Auskennen muss man sich insbesondere mit Javascript nicht, da dies Dash erledigt.
Eine Besonderheit von Dash sind die callbacks, die es ermöglichen die Ausgabe von Daten in Abhängigkeit von der Eingabe dynamisch generieren zu lassen.
Plotly express ist eine Weiterentwicklung von Plotly und stellt eine Wrapper-Klasse um Plotly dar, die es mit einfacherer Syntax fast die gleichen Ergebnisse hinsichtlich Datenvisualisierungen realisiert.
In dem Proof-of-concept wird die Pandas Library stark genutzt (beim Import und der Datenverarbeitung)
short table with heavily used software
Python und die drei Frameworks stehen unter verschiedenen Open-Source-Lizenzen wie die MIT-Lizenz, BSD oder die der Python SoftewareFoundation.

\begingroup
\setlength{\tabcolsep}{4pt} % Default value: 6pt
\renewcommand{\arraystretch}{1.5}
%\resizebox{\textwidth}{!}{
\begin{table}[h]
    \centering
    \begin{adjustbox}{max width=\textwidth}
    \Huge
    \begin{tabular}{lccl}
      %\begin{tabular}{p{3cm}p{5cm}p{1cm}p{1.5cm}p{2cm}p{4cm}}
       \toprule
       \textbf{Name}             &{Version}    &\textbf{Lizenz}                        & \textbf{Webseite}\\
       \midrule     
            Python               &3.8.5         &Open Source (PSF)                     & \url{https://docs.python.org/3.5/}\\
            Pandas               &1.1.2         &3-Clause-BSD-License                  & \url{https://pandas.pydata.org/pandas-docs/version/1.1.2/}\\
            Plotly Express       &0.4.0         &MIT-License                           & \url{https://plotly.com/python/}\\
            Dash                 &1.16.3        &MIT-License                           & \url{https://dash.plotly.com/}\\
            

        \bottomrule
    \end{tabular}
    \end{adjustbox}
    \caption
    \label{tab:Software-Requirements}
    }
     \end{table}
\endgroup

% Urls zu den als Fußnoten dazufügen.


Dash -> 
Wrapper-Klasse 
Organisiert sind die Klassen in sogenannten Modulen, die als solche importiert werden können und man dann auf die Klassen und deren Methoden 
über den Klassen- und Methodennamen Zugriff hat.
Alle Methoden, die kein Objekt/ Zustand (self) erfordern, wurden in standalone-Funktionen getan. Diese werden als Helper-Funktionen in eigene Dateien getan,
wenn sich der Aufgabenkontext der Klassen mit den Funktionen überschneidet.
Objekt-orientiert programmiert wurden ebenso nur Teile des Systems. Diese Teile sind der Bereich des Datenimports, der Datenberechnung. Wohingegen die Bereiche der Darstellung mit dem Dashframework eher programmiert wurden. 

Herausfiltern der Datensätze zum Beispiel aus den Neuerwerbungslisten, hinter denen keine physische Entsprechung zum einen steht (mehrteilige Ressouurce auf Gesaamtitel?ebene, Datensätze Schriftenreihen)

    \subsection{Systemarchitektur}
    
    grobes Bild der Systemarchitekturmit Klassen -> UML-Diagramm mit Bereichen
    
    
    Plotly Express\\
    Pandas\\
    Dash\\
    -> effizient und effektiv zu sein


Die Klassen im Import Module sind auf den Datenbestand aus fremden Quellen zugeschnitten.
(Budget Umsatz, Neuerwerbungslisten (Anwendungsfall 4 - 6))
Ziel: Die Daten ohne Verluste zu importieren. Dabei spielt für die spätere Anwendung im System, der mtl. Stand
ausgedrückt im Datum eine wichtige Rolle.
Es wurden als Datengrundlage nur die Daten berücksichtigt, die für die Anwendungsfälle benötigt werden.
Außen vor blieb zum Beispiel die Integration der Counter-Daten.
Die von der Bibliothek erhobenen Daten

Einbindung der RVK-Systematik - gewisse Anpassung durch die Bibliothek -> Hinzunahme von eigenen Systematikstellen,
die zum Teil andere Systematikstellen ersetzten (NT eigentlich Rechtswissenschaften, in der Institutsbibliothek Noten, MT eigentlich
Gesundheitswissenschaften, in der Institutsbibliothek Methoden)

    \begin{lstlisting}[language=Python, caption=Python example]
        import pandas as pd
        import plotly.express as px

        
    \end{lstlisting}

    \subsection{Teilsysteme}
    
    Kurzbeschreibung / Ziel:\\
    Wo zu finden:\\
    Datengrundlage:\\
    Klassenbeschreibung:\\
    Prozess:\\
    
    Klassenbeschreibung: mit Methoden und evtl. Funktionen
    Das System splittet sich in XX Teilsysteme auf: Import-System, Datentransformtion und Datenhaltung, Diagramme, Dash, PDF-System. Dazu gibt es noch Skripte,
    die der Datenextraktion dienen. Diese sind zu finden in dem Ordner XX. Notwendige andere Dateien, die die Auswahll der angezeigten daten steuern sind zu finden im Verzeichnis: XX
    
    Abbildung: UML-Diagramm mit Bereichen
    
    Import-System\\
    Kurzbeschreibung / Ziel: Import der Daten aus einem lokalen Folder in einen anderen. 
    Die Daten ohne Verluste zu importieren. Dabei spielt für die spätere Anwendung im System, der mtl. Stand ausgedrückt im Datum eine wichtige Rolle.\\
    Wo zu finden: Modul XX im Folder \\
    Datengrundlage: Es wurden als Datengrundlage nur die Daten berücksichtigt, die für die Anwendungsfälle benötigt werden.
    Außen vor blieb zum Beispiel die Integration der Counter-Daten.
    Budget, Umsatz, Ausleihdaten, Neuerwerbungslisten in txt, tsv-Formaten.
    
    Klassenbeschreibung:
    Prozess: Die Klassen im Import Module sind auf den Datenbestand aus fremden Quellen zugeschnitten.
    (Budget Umsatz, Neuerwerbungslisten (Anwendungsfall 4 - 6))
    Add: Datum bei Budget, Umsatz-Daten, Ausleihdaten, evtl. RVK-Systematikstelle, das wichtig ist für spätere Auswertung und grafische Darstellung
    im Dashboard
    
    
 
    
    
    Import der Daten aus einem lokalen Folder in einen anderen. 
    Währenddessen Bearbeitung der Daten und Basic Cleaning
    Umsetzung:
    Add: Datum bei Budget, Umsatz-Daten, Ausleihdaten, evtl. RVK-Systematikstelle 
    Die Klassen im Import Module sind auf den Datenbestand aus fremden Quellen zugeschnitten.
    (Budget Umsatz, Neuerwerbungslisten (Anwendungsfall 4 - 6))
    Ziel: Die Daten ohne Verluste zu importieren. Dabei spielt für die spätere Anwendung im System, der mtl. Stand
    ausgedrückt im Datum eine wichtige Rolle.
    Es wurden als Datengrundlage nur die Daten berücksichtigt, die für die Anwendungsfälle benötigt werden.
    Außen vor blieb zum Beispiel die Integration der Counter-Daten.
    Die von der Bibliothek erhobenen Daten. Die Übersicht zeigen folgende Methoden der einzelnen Klassen.
    git
    
    
    

\section{Implementierung}
    \subsection{Umgesetzte Anforderungen}
    Folgende Must-Anforderungen wurden erfüllt.
    R1, R2, R3, R4, R5, R7
    F1, F4, F5, F12
    NF5, NF6, 
    \subsection{Funktionsweise}
    Das Layout des Dashboards besteht aus drei Tabs. Diese wurden nach den XXX Berreichen der Bibliothek aufgeteilt. Diese Aufteilung eerschien als sinnvoll,
    da hier nichts überfrachtet wird.

    Beim Import werden zum beispiel txt-Dateien für das Budget und den Umsatz nach Lieferanten importiert von einem lokalen Verzeichnis
    in ein Zielverzeichnis importiert. Nach Abschluß des Prozesses liegen die Daten im csv-Format vor und sehen so aus wie in Abbildung
    old -> new
    figure




\section{Bewertung}
