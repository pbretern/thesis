\documentclass{mimosis}
%\usepackage[utf8]{inputenc}
%\usepackage[ngerman]{babel}
\usepackage{metalogo}

%%%%%%%%%%%%%%%%%%%%%%%%%%%%%%%%%%%%%%%%%%%%%%%%%%%%%%%%%%%%%%%%%%%%%%%%
% Some of my favourite personal adjustments
%%%%%%%%%%%%%%%%%%%%%%%%%%%%%%%%%%%%%%%%%%%%%%%%%%%%%%%%%%%%%%%%%%%%%%%%
%
% These are the adjustments that I consider necessary for typesetting
% a nice thesis. However, they are *not* included in the template, as
% I do not want to force you to use them.

% This ensures that I am able to typeset bold font in table while still aligning the numbers
% correctly.
\usepackage{etoolbox}

\usepackage[binary-units=true]{siunitx}
\DeclareSIUnit\px{px}

\sisetup{%
  detect-all           = true,
  detect-family        = true,
  detect-mode          = true,
  detect-shape         = true,
  detect-weight        = true,
  detect-inline-weight = math,
}

%%%%%%%%%%%%%%%%%%%%%%%%%%%%%%%%%%%%%%%%%%%%%%%%%%%%%%%%%%%%%%%%%%%%%%%%
% Hyperlinks & bookmarks
%%%%%%%%%%%%%%%%%%%%%%%%%%%%%%%%%%%%%%%%%%%%%%%%%%%%%%%%%%%%%%%%%%%%%%%%

\usepackage[%
  colorlinks = true,
  citecolor  = RoyalBlue,
  linkcolor  = RoyalBlue,
  urlcolor   = RoyalBlue,
  unicode,
  ]{hyperref}

\usepackage{bookmark}

%%%%%%%%%%%%%%%%%%%%%%%%%%%%%%%%%%%%%%%%%%%%%%%%%%%%%%%%%%%%%%%%%%%%%%%%
% Bibliography
%%%%%%%%%%%%%%%%%%%%%%%%%%%%%%%%%%%%%%%%%%%%%%%%%%%%%%%%%%%%%%%%%%%%%%%%
%
% I like the bibliography to be extremely plain, showing only a numeric
% identifier and citing everything in simple brackets. The first names,
% if present, will be initialized. DOIs and URLs will be preserved.

\usepackage[%
  autocite     = plain,
  backend      = biber,
  doi          = true,
  url          = true,
  giveninits   = true,
  hyperref     = true,
  maxbibnames  = 99,
  maxcitenames = 99,
  sortcites    = true,
  style        = numeric,
  ]{biblatex}

%%%%%%%%%%%%%%%%%%%%%%%%%%%%%%%%%%%%%%%%%%%%%%%%%%%%%%%%%%%%%%%%%%%%%%%%
% Some adjustments to make the bibliography more clean
%%%%%%%%%%%%%%%%%%%%%%%%%%%%%%%%%%%%%%%%%%%%%%%%%%%%%%%%%%%%%%%%%%%%%%%%
%
% The subsequent commands do the following:
%  - Removing the month field from the bibliography
%  - Fixing the Oxford commma
%  - Suppress the "in" for journal articles
%  - Remove the parentheses of the year in an article
%  - Delimit volume and issue of an article by a colon ":" instead of
%    a dot ""
%  - Use commas to separate the location of publishers from their name
%  - Remove the abbreviation for technical reports
%  - Display the label of bibliographic entries without brackets in the
%    bibliography
%  - Ensure that DOIs are followed by a non-breakable space
%  - Use hair spaces between initials of authors
%  - Make the font size of citations smaller
%  - Fixing ordinal numbers (1st, 2nd, 3rd, and so) on by using
%    superscripts

% Remove the month field from the bibliography. It does not serve a good
% purpose, I guess. And often, it cannot be used because the journals
% have some crazy issue policies.
\AtEveryBibitem{\clearfield{month}}
\AtEveryCitekey{\clearfield{month}}

% Fixing the Oxford comma. Not sure whether this is the proper solution.
% More information is available under [1] and [2].
%
% [1] http://tex.stackexchange.com/questions/97712/biblatex-apa-style-is-missing-a-comma-in-the-references-why
% [2] http://tex.stackexchange.com/questions/44048/use-et-al-in-biblatex-custom-style
%
\AtBeginBibliography{%
  \renewcommand*{\finalnamedelim}{%
    \ifthenelse{\value{listcount} > 2}{%
      \addcomma
      \addspace
      \bibstring{and}%
    }{%
      \addspace
      \bibstring{and}%
    }
  }
}

% Suppress "in" for journal articles. This is unnecessary in my opinion
% because the journal title is typeset in italics anyway.
\renewbibmacro{in:}{%
  \ifentrytype{article}
  {%
  }%
  % else
  {%
    \printtext{\bibstring{in}\intitlepunct}%
  }%
}

% Remove the parentheses for the year in an article. This removes a lot
% of undesired parentheses in the bibliography, thereby improving the
% readability. Moreover, it makes the look of the bibliography more
% consistent.
\renewbibmacro*{issue+date}{%
  \setunit{\addcomma\space}
    \iffieldundef{issue}
      {\usebibmacro{date}}
      {\printfield{issue}%
       \setunit*{\addspace}%
       \usebibmacro{date}}%
  \newunit}

% Delimit the volume and the number of an article by a colon instead of
% by a dot, which I consider to be more readable.
\renewbibmacro*{volume+number+eid}{%
  \printfield{volume}%
  \setunit*{\addcolon}%
  \printfield{number}%
  \setunit{\addcomma\space}%
  \printfield{eid}%
}

% Do not use a colon for the publisher location. Instead, connect
% publisher, location, and date via commas.
\renewbibmacro*{publisher+location+date}{%
  \printlist{publisher}%
  \setunit*{\addcomma\space}%
  \printlist{location}%
  \setunit*{\addcomma\space}%
  \usebibmacro{date}%
  \newunit%
}

% Ditto for other entry types.
\renewbibmacro*{organization+location+date}{%
  \printlist{location}%
  \setunit*{\addcomma\space}%
  \printlist{organization}%
  \setunit*{\addcomma\space}%
  \usebibmacro{date}%
  \newunit%
}

% Display the label of a bibliographic entry in bare style, without any
% brackets. I like this more than the default.
%
% Note that this is *really* the proper and official way of doing this.
\DeclareFieldFormat{labelnumberwidth}{#1\adddot}

% Ensure that DOIs are followed by a non-breakable space.
\DeclareFieldFormat{doi}{%
  \mkbibacro{DOI}\addcolon\addnbspace
    \ifhyperref
      {\href{http://dx.doi.org/#1}{\nolinkurl{#1}}}
      %
      {\nolinkurl{#1}}
}

% Use proper hair spaces between initials as suggested by Bringhurst and
% others.
\renewcommand*\bibinitdelim {\addnbthinspace}
\renewcommand*\bibnamedelima{\addnbthinspace}
\renewcommand*\bibnamedelimb{\addnbthinspace}
\renewcommand*\bibnamedelimi{\addnbthinspace}

% Make the font size of citations smaller. Depending on your selected
% font, you might not need this.
\renewcommand*{\citesetup}{%
  \biburlsetup
  \small
}

\DeclareLanguageMapping{english}{english-mimosis}

\addbibresource{Bibliography/thesis.bib}

%%%%%%%%%%%%%%%%%%%%%%%%%%%%%%%%%%%%%%%%%%%%%%%%%%%%%%%%%%%%%%%%%%%%%%%%
% Fonts
%%%%%%%%%%%%%%%%%%%%%%%%%%%%%%%%%%%%%%%%%%%%%%%%%%%%%%%%%%%%%%%%%%%%%%%%

\ifxetexorluatex
  \setmainfont{Minion Pro}
\else
  \usepackage[lf]{ebgaramond}
  \usepackage[oldstyle,scale=0.7]{sourcecodepro}
  \singlespacing
\fi

\renewcommand{\th}{\textsuperscript{\textup{th}}\xspace}

\makeindex
\makeglossaries  % loading glossary from the glossary-directory
\loadglsentries{Gloss_Acr/glossary}

%%%%%%%%%%%%%%%%%%%%%%%%%%%%%%%%%%%%%%%%%%%%%%%%%%%%%%%%%%%%%%%%%%%%%%%%
% Incipit
%%%%%%%%%%%%%%%%%%%%%%%%%%%%%%%%%%%%%%%%%%%%%%%%%%%%%%%%%%%%%%%%%%%%%%%%

%\title{\texttt{latex-mimosis}}
\title{Konzeption und Entwicklung eines\\ datengetriebenen Unterstützungssystems\\ für
Etatplanung und Mittelallokation\\ einer hybriden
Spezialbibliothek}
\author{Peter Breternitz}


\begin{document}


\frontmatter
  \begin{titlepage}
    \vspace*{5cm}
    \makeatletter
    \begin{center}
      \begin{Large}
        \@title
      \end{Large}\\[0.1cm]
      %
      %\begin{Large}
        %\@subtitle
      %\end{Large}\\
      %
      \vspace{0.25 cm}
      \emph{von}\\
      \vspace{0.25 cm}
      \@author
      %
      %\vfill
      %Masterarbeit zur Erlangung des 
      %akademischen Grades\\
      %\emph{Master}\\
      %an der \\
      %\textsc{Technischen Hochschule Wildau}
    \end{center}
    \makeatother
  \end{titlepage}
  
  \newpage
  \null
  \thispagestyle{empty}
  \newpage
  \begin{center}
    \textsc{Zusammenfassung}
  \end{center}
  %
  \noindent
  %
Aufgrund von ökonomischen Entwicklungen müssen Bibliotheken ihr Etat effizient und bedarfsgerecht einsetzen. 
Zudem werden Etatverhandlungen in Bibliotheken immer wichtiger. 
Das Ziel der vorliegenden Masterarbeit war es, ein Proof-of-Concept eines datengetriebenen Unterstützungssystems zur
Etatplanung- und Mittelallokation für die Bibliothek des Max-Planck-Institutes für empirische Ästhetik zu konzipieren und zu entwickeln.
Dafür wurden aus verschiedenen bibliothekarischen Bereichen Daten analysiert 
und ausgewertet. Das datengetriebene Unterstützungssystem ermöglicht die anschauliche Anzeige von wesentlichen Key Performance Indicators wie Budget, 
Umsatz, Ausleihe, Bestandsentwicklung sowie die Lesesaalnutzung in einem Dashboard.
Damit kann die Bibliothek ihre Planung des Etats und den Einsatz der Mittelallokation effizienter und bedarfsgerechter gestalten
sowie Verhandlungen über den Etat sicher führen.

  

\begin{center}
  \textsc{Abstract}
  \end{center}

  \noindent
  %
  Due to economic developments, libraries must use their budgets efficiently and in line with demand. 
  In addition, budget negotiations in libraries are becoming more and more important. 
  The goal of this master thesis was to develop a proof-of-concept of a data-driven support system for
  budget planning and resource allocation for the library of the Max Planck Institute for Empirical Aesthetics.
  For this purpose, data from different library areas were analyzed and evaluated. The data-driven support system 
  displays key performance indicators such as budget, expenditures, circulation, collection development, and reading room usage in a dashboard. 
  This allows the library to plan its budget and allocate funds more efficiently and in line with its needs
  as well as conduct budget negotiations with confidence.
  %\lipsum[1]


  \tableofcontents

\mainmatter

  %%%%%%%%%%%%%%%%%%%%%%%%%%%%%%%%%%%%%%%%%%%%%%%%%%%%%%%%%%%%%%%%%%%%%%%%
\chapter{Einleitung}
%%%%%%%%%%%%%%%%%%%%%%%%%%%%%%%%%%%%%%%%%%%%%%%%%%%%%%%%%%%%%%%%%%%%%%%%

\begin{center}
  \begin{minipage}{0.5\textwidth}
    \begin{small}
      In which the reasons for creating this package are laid bare for the
      whole world to see and we encounter some usage guidelines.
    \end{small}
  \end{minipage}
  \vspace{0.5cm}
\end{center}

\noindent This package contains a minimal, modern template for writing your
thesis. While originally meant to be used for a Ph.\,D.\ thesis, you can
equally well use it for your honour thesis, bachelor thesis, and so
on---some adjustments may be necessary, though.

%%%%%%%%%%%%%%%%%%%%%%%%%%%%%%%%%%%%%%%%%%%%%%%%%%%%%%%%%%%%%%%%%%%%%%%%
\section{Why?}
%%%%%%%%%%%%%%%%%%%%%%%%%%%%%%%%%%%%%%%%%%%%%%%%%%%%%%%%%%%%%%%%%%%%%%%%

I was not satisfied with the available templates for \LaTeX{} and wanted
to heed the style advice given by people such as Robert
Bringhurst~\cite{Bringhurst12} or Edward R.\
Tufte~\cite{Tufte90,Tufte01}. While there \emph{are} some packages out
there that attempt to emulate these styles, I found them to be either
too bloated, too playful, or too constraining. This template attempts to
produce a beautiful look without having to resort to any sort of hacks.
I hope you like it.

%%%%%%%%%%%%%%%%%%%%%%%%%%%%%%%%%%%%%%%%%%%%%%%%%%%%%%%%%%%%%%%%%%%%%%%%
\section{How?}
%%%%%%%%%%%%%%%%%%%%%%%%%%%%%%%%%%%%%%%%%%%%%%%%%%%%%%%%%%%%%%%%%%%%%%%%

The package tries to be easy to use. If you are satisfied with the
default settings, just add
%
\begin{verbatim}
\documentclass{mimosis}
\end{verbatim}
%
at the beginning of your document. This is sufficient to use the class.
It is possible to build your document using either \LaTeX|, \XeLaTeX, or
\LuaLaTeX. I personally prefer one of the latter two because they make
it easier to select proper fonts.

%%%%%%%%%%%%%%%%%%%%%%%%%%%%%%%%%%%%%%%%%%%%%%%%%%%%%%%%%%%%%%%%%%%%%%%%
\section{Features}
%%%%%%%%%%%%%%%%%%%%%%%%%%%%%%%%%%%%%%%%%%%%%%%%%%%%%%%%%%%%%%%%%%%%%%%%

%%%%%%%%%%%%%%%%%%%%%%%%%%%%%%%%%%%%%%%%%%%%%%%%%%%%%%%%%%%%%%%%%%%%%%%%
\begin{table}
  \centering
  \begin{tabular}{ll}
    \toprule
    \textbf{Package}      & \textbf{Purpose}\\
    \midrule
      \texttt{amsmath}          & Basic mathematical typography\\
      \texttt{amsthm}           & Basic mathematical environments for proofs etc.\\
      \texttt{booktabs}         & Typographically light rules for tables\\
      \texttt{bookmarks}        & Bookmarks in the resulting PDF\\
      \texttt{dsfont}           & Double-stroke font for mathematical concepts\\
      \texttt{graphicx}         & Graphics\\
      \texttt{hyperref}         & Hyperlinks\\
      \texttt{multirow}         & Permits table content to span multiple rows or columns\\ 
      \texttt{paralist}         & Paragraph~(`in-line') lists and compact enumerations\\
      \texttt{scrlayer-scrpage} & Page headings\\
      \texttt{setspace}         & Line spacing\\
      \texttt{siunitx}          & Proper typesetting of units\\
      \texttt{subcaption} & Proper sub-captions for figures\\
    \bottomrule
  \end{tabular}
  \caption{%
    A list of the most relevant packages required~(and automatically imported) by this template.
  }
  \label{tab:Packages}
\end{table}
%%%%%%%%%%%%%%%%%%%%%%%%%%%%%%%%%%%%%%%%%%%%%%%%%%%%%%%%%%%%%%%%%%%%%%%%

The template automatically imports numerous convenience packages that
aid in your typesetting process. \autoref{tab:Packages} lists the
most important ones. Let's briefly discuss some examples below. Please
refer to the source code for more demonstrations.

%%%%%%%%%%%%%%%%%%%%%%%%%%%%%%%%%%%%%%%%%%%%%%%%%%%%%%%%%%%%%%%%%%%%%%%%
\subsection{Typesetting mathematics}
%%%%%%%%%%%%%%%%%%%%%%%%%%%%%%%%%%%%%%%%%%%%%%%%%%%%%%%%%%%%%%%%%%%%%%%%

This template uses \verb|amsmath| and \verb|amssymb|, which are the
de-facto standard for typesetting mathematics. Use numbered equations
using the \verb|equation| environment.
%
If you want to show multiple equations and align them, use the
\verb|align| environment:
%
\begin{align}
    V &:= \{ 1, 2, \dots \}\\
    E &:= \big\{ \left(u,v\right) \mid \dist\left(p_u, p_v\right) \leq \epsilon \big\}
\end{align}
%
Define new mathematical operators using \verb|\DeclareMathOperator|.
Some operators are already pre-defined by the template, such as the
distance between two objects. Please see the template for some examples. 
%
Moreover, this template contains a correct differential operator. Use \verb|\diff| to typeset the differential of integrals:
%
\begin{equation}
  f(u) := \int_{v \in \domain}\dist(u,v)\diff{v}
\end{equation}
%
You can see that, as a courtesy towards most mathematicians, this
template gives you the possibility to refer to the real numbers~$\real$
and the domain~$\domain$ of some function. Take a look at the source for
more examples. By the way, the template comes with spacing fixes for the
automated placement of brackets.

%%%%%%%%%%%%%%%%%%%%%%%%%%%%%%%%%%%%%%%%%%%%%%%%%%%%%%%%%%%%%%%%%%%%%%%%
\subsection{Typesetting text}
%%%%%%%%%%%%%%%%%%%%%%%%%%%%%%%%%%%%%%%%%%%%%%%%%%%%%%%%%%%%%%%%%%%%%%%%

Along with the standard environments, this template offers
\verb|paralist| for lists within paragraphs.
%
Here's a quick example: The American constitution speaks, among others, of
%
\begin{inparaenum}[(i)]
  \item life
  \item liberty
  \item the pursuit of happiness.
\end{inparaenum}
%
These should be added in equal measure to your own conduct. To typeset
units correctly, use the \verb|siunitx| package. For example, you might
want to restrict your daily intake of liberty to \SI{750}{\milli\gram}.

Likewise, as a small pet peeve of mine, I offer specific operators for \emph{ordinals}. Use \verb|\th| to typeset things like July~4\th correctly. Or, if you are referring to the 2\nd edition of a book, please use \verb|\nd|. Likewise, if you came in 3\rd in a marathon, use \verb|\rd|. This is my 1\st rule.

%%%%%%%%%%%%%%%%%%%%%%%%%%%%%%%%%%%%%%%%%%%%%%%%%%%%%%%%%%%%%%%%%%%%%%%%
\section{Changing things}
%%%%%%%%%%%%%%%%%%%%%%%%%%%%%%%%%%%%%%%%%%%%%%%%%%%%%%%%%%%%%%%%%%%%%%%%

Since this class heavily relies on the \verb|scrbook| class, you can use
\emph{their} styling commands in order to change the look of things. For
example, if you want to change the text in sections to \textbf{bold} you
can just use
%
\begin{verbatim}
  \setkomafont{sectioning}{\normalfont\bfseries}
\end{verbatim}
%
at the end of the document preamble---you don't have to modify the class
file for this. Please consult the source code for more information.


  \chapter{Einführung}
%Das große Problem\\
Als Ende des Jahres 2019 der Ausbruch der Covid-19-Pandemie begann, entwickelte die 
Johns Hopkins University ein Dashboard als Antwort auf die anhaltende Unsicherheit im Bereich der öffentlichen Gesundheit. 
Dieses visualisiert seit je die gemeldeten Fälle weltweit. Es wurde entwickelt, um Forschern, Gesundheitsbehörden und der breiten Öffentlichkeit 
ein benutzerfreundliches Instrument an die Hand zu geben, mit dem sich der Ausbruch verfolgen lässt. 
Zu dessen Datenquellen gehören unter anderem die Informationen der Weltgesundheitsorganisation, staatliche und nationale
Gesundheitsämter. Die Daten wurden aggregiert und verdichtet. So visualisiert das Dashboard mit einer Landkarte und Punkten den Ausbruch. 
Dazu gibt es Zahlen der bestätigten COVID-19-Fälle, der Todesfälle und  der Genesungen für alle betroffenen Länder\cite[Vgl.][533]{dong_interactive_2020}.

Das faszinierende an dem Dashboard ist, das es gelungen ist, alle relevanten Zahlen auf einer Seite darzustellen
und somit der interessierten Öffentlichkeit schnell einen Überblick zu verschaffen.

\section{Problemstellung}
%Das kleine Problem\\
Ausgehend von ökonomischen, informationstechnologischen und marktpolitischen Einschnitten in den
vergangenen Jahrzehnten, sind Bibliotheken dazu veranlasst, ihr Budget hinsichtlich der Informationsbedarfe
ihrer Nutzer:innen behutsamer zu planen und sich in zunehmenden Maße gegenüber ihren Unterhaltsträgern zu rechtfertigen.
Die Relevanz von bibliothekarischen Statistiken ist in diesem Zusammenhang größer geworden.
Deswegen ist es wichtig, Daten aus den bibliothekarischen Bereichen zu aggregieren, zu erheben und statistisch
auszuwerten, um auf Basis der daraus erzielten Erkenntnisse handeln zu können. 
Die Transparenz von statistischen Daten sorgt für eine bessere Grundlage in den Verhandlungen mit den Stakeholdern
einer Bibliothek. Zudem wird durch sie der Einsatz des Bibliotheksbudgets zielgerichteter auf die Bedürfnisse der Nutzer:innen zugeschnitten.
Dazu ist es zweckmäßig, alle anfallenden Daten für die Budgetplanung und Mittelallokation einer Bibliothek zentral zu sammeln und mit geeigneten 
statistischen Methoden und Verfahren langfristig auszuwerten. Um den Wert dieser aus den Daten gewonnenen Information elegant den Stakeholdern zu kommunizieren und zu präsentieren,
können geeignete Verfahren der Datenvisualisierung zum Einsatz kommen. Die technische Realisierung kann durch gewöhnliche Tabellenkalkulationsprogramme umgesetzt werden.
Um den mitunter hohen Zeitaufwand einerseits zu minimieren und den Automatisierungsgrad hinsichtlich der Aggregation und Auswertung der bibliothekarischen Daten 
andererseits zu erhöhen, können aber auch andere technische Umsetzungen eingesetzt werden. In Bereichen der Wirtschaft kommen sogenannte Business-Intelligence-Systeme zum Einsatz,
die die Entscheidungsfindung auf Grundlage von Unternehmensdaten IT-basiert unterstützen.
%kann geschehen mit herkömmlichen Anwendungen wie Tabellenkalkulationsprogrammen oder mit anderen BI-Systemen...

%Was ist der Markt?\\
%--------------------
Es gibt bereits eine Vielzahl kommerzieller Lösungen für den Bibliotheksbereich, die auf Business-Intelligence-Systemen basieren.
Zu nennen wären \textit{AlmaAnalytics} für das Next-Generation-Library-System \textit{Alma} von \textit{ExLibris}\footnote{\url{https://www.exlibrisgroup.com/products/alma-library-services-platform/alma-analytics}
Stand: 26.05.2020}, \textit{BibControl} von \textit{OCLC}\footnote{\url{https://www.oclc.org/de/bibcontrol.html} Stand: 26.05.2020},
\textit{CollectionHq} von \textit{Baker \& Taylor}\footnote{\url{https://www.collectionhq.com/} Stand: 26.05.2020} oder \textit{Libinsight} von \textit{SpringShare}\footnote{\url{https://springshare.com/libinsight/} Stand: 26.05.2020}.
Darüber hinaus gibt es Business-Intelligence-Applikationen, die von Bibliotheken für Reporting, Datenanalyse und Datenvisualisierung adaptiert werden,
wie zum Beispiel \textit{Tableau} von der Firma \textit{Tableau Software},
\textit{Crystal Reports} von \textit{SAP} oder Microsoft BI.
Diese Applikationen sind entweder an bestimmte Bibliothekssysteme zurückgebunden, limitiert in ihren
Funktionen\cite{golas_statistische_2018} oder zu generisch.
Überdies wird sowohl von \textit{HeBis} bzw. von der
Lokal-Bibliothekssystembetreuung als auch von der \textit{mpdl} keine Applikation
in dieser Richtung angeboten.
Ebenso ist ungewiss, wann die Ablösung des schon betagten \textit{CBS/LBS} hin zu
einem neuen Next-Generation-Library-System im \textit{HeBis-Verbund} stattfinden wird und ob
es ein Modul zur statistischen Datenerhebung liefern wird.

%Wem hilft es?\\
%--------------
Die Notwendigkeit für diese Applikation ist durch das
Fehlen eines zentralen Nachweisortes für bibliothekarische
Statistiken in der Bibliothek gegeben. Da die Bibliothek zudem verschiedene Recherche-Systeme den
Wissenschaftler:innen anbietet, wäre eine Engführung der statistischen
Datenerhebung auf eine Plattform begrüßenswert.
Des Weiteren ist das Erfordernis, bibliothekarische Geschäftsprozesse zu evaluieren und die
Servicedienstleistungen bezüglich der Ziele der Institution noch weiter zu
optimieren, von großer Relevanz. 
Die zu entstehende Applikation könnte hierbei helfen, systematisches Controlling einzuführen und das
Bibliotheksmanagement weiter zu professionalisieren.
%predictive analysis\\


%Warum jetzt?\\
%--------------

Mit dem Ende der Konsolidierungsphase der
Bibliothek, die im Zuge des \textit{Max-Planck-Institutes für empirische
Ästhetik} 2014 gegründet wurde, tritt sie ein in eine Phase, in der ab dem Jahr
2021 Budgetplanungen eine größere Rolle spielen werden.

%Ist das Problem lösbarerer geworden?\\
-------------------\\
Ein System jenseits von Excel zu entwickeln ist heute einfacher geworden, da es einerseits einen Markt für
solche Anwendungen, der im Schatten von DataScience wächst. So muss man kein ausgwachsener FullStack-Entwickler,
um eine Anwendung zu programmieren,
%Wodurch?\\
sondern man kann zurück greifen auf ausgereiftere und mächtige Frameworks
wenig aufwendig Daten statistisch auszuwerten, eher Problem, dass es so viele heterogene Daten aus verschiedenen Datenquellen
gibt die aufbereitet werden müssen. Dashboards gibt es zwar schon lange, sind aber jetzt auch einfacher geworden zu gestalten.
%Was ist der Trend?\\
Ad-hoc Realtime Data - Weg bewegen von schwierig aufsetzen DWH hinzu Data Lakes, die Daten auswerten, wenn sie benötigt werden


\section{Ziel der Arbeit}
%Ihr Ziel der Arbeit in zwei Sätzen.\\
Das Ziel der Arbeit ist die Schaffung eines Dashboards für Budgetplanung in Bibliotheken. 
In Anlehnung an \acrfull{BI}-Systeme soll ein System als proof-of-concept entstehen,
mit dem systematisch die relevanten Daten einer hybriden Spezialbibliothek aggregiert, statistisch
analysiert und mit geeigneten und modernen Datenvisualisierungen ausgegeben werden sollen.
Um künftigen Anforderungen gewachsen zu sein, soll sie
modulbasiert programmiert werden und dadurch leicht erweiterbar und eventuell von
anderen Bibliotheken nachnutzbar sein.

Warum genau dieses Problem?\\
Ist Ihr Beitrag völlig neu, oder nur ein Baustein?\\
Ist Ihr Problem schwer zu lösen oder „straight forward“?\\
eher schwieriger zu lösen, da generischer Ansatz gewählt werden soll -> abstrahieren ein bisschen von konkreter Implementierung
Eher Forschung oder eher Anwendung?\\
eher anwendung
Wenn Sie ein System bauen...\\
Welche Anfragen / Aufgaben wollen Sie beantworten / lösen können?\\
Entwicklung eines Workflows für die Aktualisierung der Daten...
Framework für template design für Daten, die eingespeist werden sollen
Welche Kernfunktionalität soll Ihr System haben?\\
Automatisierte Prozesse bei der Auswertung mit statistischen Verfahren, Import der Daten soll fast vollständig automatisiert sein
Interaktiviät -> Multidemensionalität, Eingrenzung der Zeiträume, Auswählen welche Auswertungen nach Medienart ... Domainwissen
Auswahl aus mehreren Visualisierungen
Bereitstellung der wichtigsten KPI's in einem PDF

Was ist ein typischer (Bedienungs-) Prozess für Ihr System?\\
1) montaliche Budget und Umsatzzahlen kommen als email, werden automatisch in csv umgewandelt -> werden an große csv / datenbank geschrieben
mit der veränderten Datenlage entstehen Veränderungen in den Zahlen -> in Visualisierungen
2) Abfrage des Systems nach Neuerwerbungen mit Systemstellen in der RVK, vierteljahrlich, halbjahrlich, jährlich -> Kontrolle wie und in welchen Systematikgruppen
der Bestand wächst -> RVK-Systematik-Tabelle
Wer nutzt Ihr System, Ihren Algorithmus?\\
Nutzen sollen das System Bibliotheksmitarbeiter:innen und Bibliotheksleitung, auch zum Einspeisen der daten -> Fallback, Fehler abfangen beim Import
Wodurch ist dieses Nutzungsverhalten gekennzeichnet?\\
wenig bis keine Kenntnisse -> soll leicht nutzbar sein ohne große Vorkenntnisse durch automatisierte Prozesse zur Gewinnung der Ergebnisse konzentriert werden.
nur Datei in Ordner schieben
Import/ Export  soll automatisch geschehen.

Anleihen von BI-Systemen in der Architektur, Anhaltspunkte 

Ausleihzahlen nach Jahr, Quartal, Monat
Bestand -> Bestandssegmente(Klass.Gruppen) -> Bestandssegment (einezelne Klassen) -> Einzeltitel => graphisch darstellen
=> Welche Bestandsgruppe am besten geht / schlechtesten

Budgewt



Mit diesen automatisch angefertigten statistischen Datenanalysen sollen zukünftige
Entscheidungen im Bibliotheksmanagement wie Erwerbungspolitik, Budgetplanung und
Mittelallokation hinsichtlich der weiteren Entwicklung der
Servicedienstleistungen evidenzbasiert und datengetrieben unterstützt werden.

Darüber hinaus soll die Applikation  eine Funktion beinhalten, ausgewählte
Resultate automatisiert als \textit{factsheet} zu exportieren, um diese
als Rechenschaftsbericht gegenüber Stakeholdern der Bibliothek präsentieren zu können.

\section{Verwandte Arbeiten}

% Welche Vorarbeiten gibt es schon?\\
% Wo und Wann sind die Vorarbeiten entstanden?\\
% Welche Ziele haben die Vorarbeiten verfolgt?\\
% Auf welche Schwierigkeiten sind sie gestossen?

Ein gutes Beispiel für ein datengetriebenes Unterstützungssystem findet sich in
der Literatur bei Spielberg, der sich mit dem Thema der Bestandspflege an der
\textit{Universitätsbibliothek Essen} befasst und eine Applikation (weiter-)entwickelt hat, die
die Fachreferent:innen bei der Aussonderung und Erwerbung von Medien
unterstützt.\cite{spielberg_eike_t_fachref-assistent_nodate}
Ebenso finden sich in der Fachliteratur Ansätze, die vorrangig anhand einzelner
Fragestellungen hinsichtlich der Bestandsentwicklung\cite{hughes_long-term_2016} oder anderer
bibliothekarischer Servicedienstleistungen\cite{kutlay_shiny_2020, knievel_use_2006,meyer_using_2018} verschiedene statistische Analysen
vollzogen und diese visualisiert haben.
Eine Ausnahme bildet die Entwicklung eines Dashboards an der \textit{New York
University Health Sciences Libraries}, das versucht, möglichst viele Metriken
aus bibliothekarischen Dienstleistungen aufzunehmen.\cite{morton-owens_trends_2012}
Fast alle Projekte sind an größeren
Universitätsbibliotheken mit ganz unterschiedlichen softwaretechnischen
Herangehensweisen\cite{finch_using_2016, wiegand_visualizing_2013} und Zielen\cite{phetteplace_effectively_2012} entstanden.

Dennoch fehlen in der gesichteten Literatur Teile, die sich mit der Budgetierung
befassen und Auskunft über Mittelallokation geben.

Zudem fehlt ein Beispiel in der Literatur, das holistisch alle relevanten Daten, die in den
Geschäftsgängen und Servicedienstleistungen insbesondere einer Spezialbibliothek entstehen,
aggregiert, auf diesen Daten automatisch statistische Analysen ausführt und diese mit modernen Visualisierungstechniken
interaktiv darstellt.

Anwendung kommen deskriptive Statistik, es geht vielmehr darum Daten zusammen zu tragen, als diese zu explorieren 
Abgrenzung deskriptive Statistik / explorative Datenanalyse

\section{Gliederung der Arbeit}
%Wann lesen wir was und warum?\\
Im \autoref{chap:two} werden die theoretischen Grundlagen für die folgenden Kapitel gelegt. Das Kapitel befasst
sich mit den Themen Bibliothek und Statistik, Datenvisualisierung und Business-Intelligence-Systemen. Dabei wird herausgestellt, wie wichtig Statistik
im bibliothekarischen Bereich sind, was Datenvisualisierungen sind und warum sie eingesetzt werden sollen und welche Anleihen Business-Intelligence-Systeme 
für das zu entstehende System liefern können. \autoref{chap:three}
wird die Bibliothek vorgestellt und darauf eingegangen welche bibliothekarischen Statistiken bereits erhoben wurden.
Nachdem die Ausgangssituation bestimmt wurde, wird mit der Anforderungsanalyse im \autoref{chap:four} die Konzeption einer Lösung vorgestellt.
Im \autoref{chap:five} wird die Umsetzung diskutiert. Bevor das System bewertet wird, wird das Design und die Implementierung vorgestellt.
Das Fazit wird im \autoref{chap:six} mit dem Stand der Umsetzung, den lessons learned und einem Ausblick auf Themen, die noch bearbeitet werden könnten, gezogen.

% This ensures that the subsequent sections are being included as root
% items in the bookmark structure of your PDF reader.
\bookmarksetup{startatroot}
\backmatter

  \begingroup
    \addcontentsline{toc}{chapter}{Tabellenverzeichnis}
    \listoftables
    \newpage
    \addcontentsline{toc}{chapter}{Abbildungsverzeichnis}
    \listoffigures
    \newpage
    \let\clearpage\relax
    \glsaddall
    \printglossary[type=\acronymtype]
    \newpage
    \printglossary
    \printindex
    \newpage
    \printbibliography
    \newpage
    \chapter{Selbständigkeitserklärung}
Ich versichere, dass die vorliegende Arbeit von mir selbständig und ohne unerlaubte Hilfe angefertigt worden ist. 
Ich habe alle Stellen, die wörtlich oder sinngemäß aus 
Veröffentlichungen entnommen sind, durch Zitate bzw. Literaturhinweise als solche kenntlich gemacht.

\vspace{4cm}
\noindent
\begin{tabular}{ll}
    \makebox[2.5in]{\hrulefill} & \makebox[2.5in]{\hrulefill}\\
    Ort, Datum & Unterschrift\\[8ex]
\end{tabular}

  \endgroup

  %\printindex
  %\printbibliography
 




\end{document}